\begin{landscape}
    \begin{longtable}{p{0.1\linewidth}p{0.1\linewidth}p{0.15\linewidth}p{0.20\linewidth}p{0.1\linewidth}p{0.3\linewidth}}
        \caption{Σύγκριση τεχνολογιών που παρουσιάζονται στην βιβλιογραφία}
        \label{table:comparison}\\
        
        Τεχνολογία & Ερευνητές & Σκοπός Έρευνας & Μέθοδοι & Τύπος Έρευνας & Κύρια Πορίσματα - Συμπεράσματα \\
        \hline

        LoRA WAN & Guozheng Zhao, Kaiqiang Lin, Tong Hao \cite{zhao_feasibility_2023} 
    & Ανάπτυξη προσομοίωσης για να αξιολογηθεί η σκοπιμότητα της υπόγειας παρακολούθησης χρησιμοποιώντας
        υπόγειο ασύρματο δίκτυο αισθητήρων (WUSN) που βασίζεται στο LoRaWAN & 
        Ο προσομοιωτής έκανε 20 προσομοιώσεις με διαφορετικές ρυθμίσεις δικτύου και υπολόγισε πολλές
        μετρήσεις για κάθε προσομοίωση, όπως: Goodput (bps), αναλογία σφάλματος καθυστέρησης (DER)
        και απόδοση ελέγχου ισχύος (EPP). Ο προσομοιωτής εντόπισε επίσης πιθανές συγκρούσεις μεταξύ
        συσκευών στο ασύρματο δίκτυο αισθητήρων χρησιμοποιώντας τέσσερις διαφορετικές λειτουργίες:
        σύγκρουση συχνότητας, Sf(sps), ισχύος, χρονική σύγκρουση
        & Πείραμα μέσω προσομοίωσης 
        & Μέσω πειραμάτων προσομοίωσης, αυτοί οι παράγοντες διερευνήθηκαν ποσοτικά και τα αποτελέσματα
        έδειξαν ότι τα WUSN που βασίζονται στο LoRaWAN επέτρεψαν βαθύτερη και ευρύτερη υπόγεια
        παρακολούθηση με καλή απόδοση δικτύου σε σύγκριση με τα παραδοσιακά WUSN.\\
        \hline
        UHF & Milan Rak and Pavel Pechac \cite{rak_uhf_2007} 
        & Διερεύνηση της διάδοσης ραδιοκυμάτων σε σπήλαια για συχνότητες, 446 MHz και 860 MHz. &
        Τα συνεχή κύματα δημιουργήθηκαν από φορητούς πομπούς εξοπλισμένους με οριζόντια πανκατευθυντικές 
        μονοπολικές κεραίες που πολώθηκαν κατακόρυφα. Η κεραία λήψης μετακινήθηκε κατά μήκος του
        προφίλ της συλλογής αναζητώντας το μέγιστο επίπεδο σήματος, εξαλείφοντας τα τοπικά ελάχιστα
        που προκαλούνται από τη διάδοση πολλαπλών διαδρομών. & Πειράματα και Μαθηματική Ανάλυση &
        Η μετρηθείσα ειδική εξασθένηση στις πρώτες τέσσερις θέσεις ήταν σε καλή συμφωνία με τις 
        θεωρητικές προβλέψεις αλλά στην τοποθεσία Ε, τα αποτελέσματα διέφεραν σημαντικά. 
        Αυτή η απόκλιση αποδόθηκε σε λανθασμένη εκτίμηση των περιβαλλοντικών παραμέτρων.
        Παρατηρήθηκε συνδυασμός πολλαπλών τρόπων λειτουργίας. Αυτό αποδόθηκε στην αυξημένη απόσταση,
        η οποία προκάλεσε σημαντική διασπορά. Η σκέδαση είχε ως αποτέλεσμα την εκπόλωση των κυμάτων
        και παρόμοιες απώλειες διαδρομής και για τις δύο ορθογώνιες πολώσεις. Η μελέτη 
        διαπίστωσε ότι το διάσπαρτο πεδίο συνέβαλε σημαντικά στη διάδοση του σήματος στις στροφές.\\
        
        ZigBee (802.15.4g) & M.I. Martínez-Garrido, R. Fort \cite{martinez-garrido_experimental_2016} &
        Αξιολόγηση μιας πλατφόρμας ασύρματης επικοινωνίας για χρήση σε διάφορα περιβάλλοντα, 
        συμπεριλαμβανομένων εσωτερικών και εξωτερικών που σχεδιάστηκε για να ανιχνεύει αυτόματα νέες
        κινήσεις και να ενσωματώνει τα αντίστοιχα δεδομένα.&
        Οι παράμετροι που μετρώνται μπορούν να χωριστούν σε δύο
        κύριες κατηγορίες: αφενός, τις μετρήσεις που λαμβάνονται από τους ίδιους τους αισθητήρες 
        (σε αυτή τη μελέτη θερμοκρασία και σχετική υγρασία), και αφετέρου την ποιότητα των επικοινωνιών
        και την απαίτηση ενέργειας.& Πειράματα &
        Ποσοτικοποίησε την επίδραση των εμποδίων στη διαδρομή μεταξύ της κίνησης και του σταθμού βάσης, 
        την ενέργεια που καταναλώνεται από τα τερματικά, μετρούμενη ως προς την ισχύ της μπαταρίας. 
        Ο πιο επιζήμιος παράγοντας βρέθηκε ότι ήταν η πλήρης απόφραξη των κεραιών εκπομπής από 
        παρακείμενα μεταλλικά υλικά.\\
        \hline
        WiFi & Pablo F. Miaja, Fermin Navarro-Medina, Daniel G. Aller, Germán León,
        Alejandro Camanzo, Carlos Manuel Suarez, Francisco G. Alonso, Diego Nodar,
        Francesco Sauro, Massimo Bandecchi, Loredana Bessone, Fernando Aguado-Agelet, Manuel Arias
        & Χρήση ενός πρωτοκόλλου IEEE 802.11 για τη δημιουργία ραδιοδέσμης μεταξύ των σημείων πρόσβασης (CH)
        και κινητών τερματικών (CE).
        & Οποιοδήποτε από τα κινητά τερματικά (CE) μπορεί να λειτουργήσει ως «κινητά» σημεία πρόσβασης 
        που διευκολύνουν την επικοινωνία απευθείας οπτικής επαφής (LOS). &
        Μηχανική συστημάτων που βασίζονται σε μοντέλα &
        Το κύριο πλεονέκτημα του 802.11ah έγκειται στη χαμηλότερη συχνότητα που χρησιμοποιείται,
        κάτω από 1 GHz, που ικανοποιεί απαιτήσεις μεγάλης εμβέλειας ή ισοδύναμα μειωμένες ανάγκες
        ισχύος μετάδοσης, καθώς και βελτιωμένη διείσδυση εμποδίων, σε βάρος της διαθέσιμης απόδοσης.\\
        
        ELF (Extremely Low Frequency)& Jarred S. Glickstein, Jifu Liang, Seungdeog Choi, Arjuna Madanayake,
        and Soumyajit Mandal \cite{glickstein_power-efficient_2020}
        & Σχεδιασμός και τη δοκιμή ενός αποτελεσματικού συστήματος μετάδοσης δεδομένων.&
        Η πρωτότυπη μηχανική κεραία δοκιμάστηκε για να εξεταστεί η πρακτική αποτελεσματικότητά της.
        &Θεωρητική Ανάλυση \& Πειραματικός Έλεγχος&
        Ένας πομπός ELF που κατασκευάστηκε χρησιμοποιώντας ένα μηχανικά περιστρεφόμενο μαγνητικό
        δίπολο αξιολογήθηκε για να αποδειχθεί η χρήση του για ασύρματες επικοινωνίες μη οπτικής επαφής
        σε αγώγιμο μεσο.\\
        \hline
        WiFi & William Walsh, Jay Gao \cite{walsh_communications_2018}& Εξέταση ενός πρόχειρου μοντέλου
        μιας σπηλιάς και να συζήτηση σχετικά γενικών ηλεκτρομαγνητικών φαινομένων. &
        Τέσσερις περιοχές εξετάστηκαν κατά τη διάρκεια της μελέτης: το περιβάλλον επικοινωνίας στα
        πέντε μέτρα πίσω από το εμπόδιο, το περιβάλλον επικοινωνίας περίπου 18 μέτρα μπροστά από 
        το εμπόδιο, το σήμα κοντά στα τοιχώματα του σπηλαίου και το σήμα ως εισχωρεί σε μια 
        πλευρική δίοδο που αποκλείεται από τη θέα στο πίσω μέρος της σπηλιάς.&
        Μοντελοποίηση σε Η/Υ & Λαμβάνεται καλό σήμα WiFi σε όλο το σπήλαιο μήκους 100 μέτρων,
        αλλά σημειώνεται ότι η γεωμετρία κάθε σπηλαίου πρέπει να εξετάζεται ξεχωριστά. 
        Η επιλεγμένη γεωμετρία μπορεί να είναι αισιόδοξη καθώς συλλαμβάνει μεγάλη ποσότητα 
        της μεταδιδόμενης ενέργειας και παρέχει πολλές ευκαιρίες διασποράς.\\
        \hline
        Ασαφή Λογική \& BPSK (350MHz)& Muhammed Enes Bayrakdar \cite{bayrakdar_rule_2019}
        & Ανάπτυξη μιας μεθόδου που διασφαλίζει τη μετάδοση δεδομένων χωρίς απώλειες, 
        ελαχιστοποίηση της κατανάλωσης ενέργειας σε υπόγεια δίκτυα αισθητήρων, 
        βελτιστοποίηση της επιλογής των σταθμών συλλογής για τη βελτίωση της απόδοσης του δικτύου. 
        & Ασαφή (Fuzzy) λογική χρησιμοποιείται για τις λειτουργίες επιλογής βάσει κανόνων για τον
        προσδιορισμό της καταλληλότερης τοποθεσίας του σταθμού συλλογής. Βασικές παράμετροι όπως
        η καθυστέρηση, η απόδοση και η κατανάλωση ενέργειας διερευνώνται.
        & Simulation Modeling (Matlab \& Riverbed) &
        Το προτεινόμενο σχέδιο επιλέγει αποτελεσματικά τον καταλληλότερο σταθμό συλλογής, 
        ελαχιστοποιώντας έτσι την κατανάλωση ενέργειας. Τα αποτελέσματα της προσομοίωσης δείχνουν
        σημαντικές βελτιώσεις στη μέση καθυστέρηση και στη μέγιστη απόδοση. Η μελέτη επιβεβαιώνει
        ότι η προτεινόμενη προσέγγιση βασισμένη σε κανόνες, χρησιμοποιώντας ασαφή λογική για την 
        επιλογή σταθμών συλλογής, ενισχύει την απόδοση και την αποδοτικότητα των ασύρματων υπόγειων δικτύων αισθητήρων. \\

        BPSK (315-700MHz)& Muhammed Enes Bayrakdar \cite{bayrakdar_smart_2019}&
        Στόχος είναι η παροχή μιας αξιόπιστης μεθόδου για την ανίχνευση επιβλαβών υπόγειων 
        παρασίτων εντόμων που απειλούν την ανάπτυξη των λαχανικών&
        Ένα μαθηματικό μοντέλο περιγράφει τη δομή του δικτύου αισθητήρων και την επικοινωνία 
        τους με τους σταθμούς συλλογής. Το μοντέλο προσομοίωσης χρησιμοποιείται για να καταδείξει
        την αποτελεσματικότητα της προτεινόμενης τεχνικής σε ελεγχόμενο περιβάλλον.&
        Μαθηματική Ανάλυση \& Προσομοίωση σε Λογισμικό (Riverbed)&
        Η τεχνική που χρησιμοποιεί κατάλληλους ακουστικούς υπόγειους αισθητήρες εντοπίζει
        αποτελεσματικά τα παράσιτα με βάση το επίπεδο θορύβου που παράγουν. 
        Στη συνέχεια, οι πληροφορίες μεταδίδονται στο σταθμό συλλογής για περαιτέρω ενέργειες.
        Οι αισθητήρες εισέρχονται σε κατάσταση αναστολής λειτουργίας όταν δεν υπάρχει σημαντική
        υπόγεια δραστηριότητα.Τέλος τα δεδομένα μεταδίδονται στον σταθμό συλλογής είτε απευθείας
        είτε μέσω άλλων κόμβων (ad-hoc), εξασφαλίζοντας στιβαρή και ευέλικτη επικοινωνία ακόμη 
        και σε διαφορετικά βάθη.\\
        \hline
        WASP (Wireless Ad-hoc System for Positioning)& Mark Hedley, Ian Gipps \cite{hedley_accurate_2013}&
        Βελτίωση της ασφάλειας και της παραγωγικότητας των υπόγειων εργασιών εξόρυξης με την ακριβή
        παρακολούθηση των οχημάτων σε επιχειρήσεις σπηλαίων μπλοκ.&
        Πλακέτες κυκλωμάτων WASP για μέτρηση επικοινωνίας και ώρας άφιξης (TOA), κόμβους 
        τοποθετημένους σε οχήματα, κόμβους αγκύρωσης σε σταθερές θέσεις και υπολογιστές για 
        επεξεργασία και μετάδοση δεδομένων. Αλγόριθμος για τον υπολογισμό του εύρους που βασίζεται
        σε μετρήσεις TOA και ένας αλγόριθμος παρακολούθησης σχεδιασμένος να λειτουργεί με χαμηλή
        πυκνότητα κόμβων αναφοράς.&
        Μαθηματική Μοντελοποίηση \& Πειραματική Υλοποίηση&
        Το σύστημα απέδειξε την ικανότητα ακριβούς παρακολούθησης οχημάτων ορυχείων και διεργασιών
        εξόρυξης μεταλλεύματος σε ένα επιχειρησιακό υπόγειο ορυχείο. Κατάφερε να διατηρήσει 
        αξιόπιστη λειτουργία για μεγάλο χρονικό διάστημα (ένα έτος), αποδεικνύοντας την στιβαρότητα
        και την αποτελεσματικότητά του σε δύσκολα υπόγεια περιβάλλοντα. Τόνισε τα σημαντικά οφέλη
        της ασύρματης παρακολούθησης για τη βελτίωση της ασφάλειας των ανθρακωρύχων και τη βελτίωση
        της λειτουργικής απόδοσης με τη μείωση των συγκρούσεων και την καλύτερη παρακολούθηση
        της παραγωγής.\\

        
    \end{longtable}
\end{landscape}

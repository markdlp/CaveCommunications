\section{\textsf{Συζήτηση}}
        Σε θεωρητικό επίπεδο εχουν προταθεί τεχνολογίες extremely low frequency
        (elf) οι οποίες είναι ήδη σε χρήση σε υποβρύχια επειδή τα ραδιοκύματα ELF μπορούν να
        διεισδύσουν στο θαλασσινό νερό σε πολύ μεγαλύτερο βάθος από τα ραδιοκύματα υψηλότερης
        συχνότητας, τα οποία απορροφώνται από το νερό. Τα κύματα ELF (3 - 30Hz) έχουν πολύ
        μεγάλα μήκη κύματος, που κυμαίνονται από εκατοντάδες έως χιλιάδες χιλιόμετρα, και
        παράγονται από μια μεγάλη κεραία που ονομάζεται «δίπολο εδάφους» ή «δίπολο γης». Οι
        κεραίες που χρησιμοποιούνται για την επικοινωνία ELF έχουν συνήθως μήκος πολλών
        χιλιομέτρων και συχνά βρίσκονται σε απομακρυσμένες περιοχές με χαμηλά επίπεδα
        ανθρωπογενών παρεμβολών.

        Ωστόσο απ'οτι φάνηκε στην έρευνα για LoRaWAN \cite{zhao_feasibility_2023}, μεγάλες 
        αποστάσεις μπορούν να καλυφθούν αλλά συγκεκριμένες περιπτώσεις όπως η γεωργία ακριβείας.
        Αυτό διότι το LoRaWAN είναι μια τεχνολογία που ελαχιστοποιεί την κατανάλωση ενέργειας και
        μεγιστοποιεί στην εμβέλεια, με αντάλλαγμα τον μειωμένο ρυθμό δεδομένων, δηλαδή τον χαμηλό
        ρυθμό διάδοσης (ενδεικτικά κάθε 30 λεπτά σε βάθος 40 εκατοστών με VWC=40\%). Με άλλα
        λόγια το LoRaWAN είναι μια πολύ αξιόπιστη επιλογή, που ωστόσο υστερεί σε περιπτώσεις 
        χρήσης που χρειάζεται γρήγορη επικοινωνία σε υπόγειο περιβάλλον βαθύτερο των τριών μέτρων.

        Αξίζει επίσης να σημειωθεί οτι τα δίκτυα πέμπτης γενιάς προσφέρουν υψηλή ρυθμοαπόδοση,
        χαμηλή καθυστέρηση και μεγάλο εύρος ζώνης - πράγμα χρήσιμο για περιβάλλοντα ορυχείων.
        Ωστόσο κρίνεται σημαντικό να γίνει η διαφοροποιηση των ορυχείων με των συνήθων σπηλαίων,
        ερειπίων καθώς τα ορυχεία έχουν λιγότερο ακανόνιστη μορφολογία και έτσι μεγαλύτερο βαθμό
        ελευθερίας και εκμετάλλευσης των χαρακτηριστικών από την τηλεμετρία. Αντίστοιχα στην έρευνα
        για την ανάκλαση ακτίνων σε τραχείς επιφάνειες \cite{soo_investigation_2018} δημιουργήθηκε
        ένα τυχαίο προφίλ τραχιάς επιφάνειας που μοιάζει με φυσικές συνθήκες σπηλαίων, εξετάστηκε
        η επίδραση της περιεκτικότητας σε νερό στο βράχο στη διάδοση ηλεκτρομαγνητικών κυμάτων και
        τέλος εισήλθε ο μέσος συντελεστής ανάκλασης για να ληφθούν υπόψη τα φαινόμενα σκέδασης.

        Όπως φάνηκε από την ανάλυση των IoT σε ορυχεία \cite{ming_study_2019} η λαμβανόμενη ισχύς
        σήματος μπορεί να χρησιμοποιηθεί για την παρακολούθηση της προόδου σε ένα ορυχείο, ακόμη 
        η καμπύλη προσαρμογής προέκυψε πειραματικά για την υλοποιηση του νευρωνικού δικτύου.

        Αντίθετα από την έρευνα για ασύρματες επικοινωνίες σε σπήλαια \cite{yavuz_-cave_2009},
        φάνηκε οτι λόγω των περιοριστικών συνθηκών ραδιοσυχνοτήτων εντός των σπηλαίων καθιστούν
        ανεπαρκείς τις ασύρματες λύσεις. Επίσης το ICWCS χειρίζεται επί του παρόντος μόνο μία
        ενεργή φωνητική επικοινωνία από σημείο σε σημείο. Επιπρόσθετα η υπηρεσία καταλόγου
        εκτελείται συνήθως έξω από το σπήλαιο, αλλά οι κόμβοι κορμού χρησιμοποιούν μηχανισμό
        προσωρινής αποθήκευσης για να επιταχύνουν τις λειτουργίες καταλόγου. Τέλος πρέπει να
        εκτελεστούν αρκετά βήματα, συμπεριλαμβανομένης της ανάπτυξης και της αρχικής εγκατάστασης,
        προτού τεθεί σε λειτουργία το ICWCS. Όλα τα παραπάνω φανερώνουν κάποιους από τους
        περιορισμούς των ασύρματων δικτύων επικοινωνίας σε υπόγεια περιβάλλοντα.

        Οι σταλαγμίτες και οι σταλακτίτες στα περάσματα των σπηλαίων χρειάζονται περαιτέρω
        εξερεύνηση όπως φάνηκε στην \cite{soo_propagation_2018}.
        Τα ασύρματα σήματα στα σπήλαια είναι ευεργετικά για την επιστημονική έρευνα και τη 
        διαχείριση του τουρισμού

        Στην πλήρη ανάλυση ραδιοσυχνοτήτων \cite{pingenot_full_2005} εγινε υπολογιστική μελέτη
        διάδοσης και εξασθένησης σήματος σε περιβάλλον σπηλαίων με απώλειες. Οι εξισώσεις Maxwell
        πλήρους κύματος επιλύθηκαν απευθείας στο πεδίο του χρόνου. Στατιστικά δεδομένα που παράγονται
        για τη φασματική πυκνότητα ισχύος και τη φάση του ηλεκτρικού πεδίου συλλέχθηκαν.

        Στην μέτρηση μοντέλου ανώμαλου εδάφους για πιθανές εφαρμογές σε σπήλαια \cite{soo_measurement_2019}
        κατασκευάστηκε ένα μοντέλο ανώμαλου εδάφους για πιθανές εφαρμογές σε σπήλαια. Η μέτρηση πεδίου
        πραγματοποιήθηκε στα 2,4 GHz και φάνηκε ότι προσομοίωση ιχνηλάτησης ακτίνων με συντελεστή 
        σκέδασης ταιριάζει καλύτερα με το μετρημένο αποτέλεσμα.

    \subsection{}
        Σε πειραματικό επίπεδο έχουν υλοποιηθεί και δοκιμαστεί ορισμένα πρότυπα επικοινωνίας
        τα οποία βασίζονται σε κάποιες από τις αρχές που έχουν προταθεί παραπάνω. Πιο
        συγκεκριμένα το HeyPhone είναι ένα φορητό ραδιόφωνο που χρησιμοποιεί μια κεραία βρόχου
        με διάμετρο ένα μέτρο και έχει την ικανότητα να διαπεράσει το έδαφος σε βάθος μέχρι
        και 500 περίπου μέτρα, χρησιμοποιεί διαμόρφωση 87KHz Single Side Band (SSB).

    \subsection{}
        Ακόμη μια τεχνολογία που χρησιμοποιείται ευρέως στην βιομηχανία είναι το τηλέφωνο
        ενός καλωδίου (Single Wire Telephone). Οι συσκευές είναι πολύ απλές, χρησιμοποιώντας
        ένα μόνο op-amp τόσο για να στείλουν όσο και να λάβουν ένα ηχητικό σήμα (φωνή) κατά
        μήκος ενός μόνο μονωμένου καλωδίου, χρησιμοποιώντας τη χωρητικότητα σύζευξης χειριστών
        στη γείωση για την επιστροφή. Ο δέκτης έχει πολύ υψηλή αντίσταση εισόδου. Η ζήτηση
        ισχύος είναι πολύ χαμηλή, με πολλές ώρες (ή ημέρες) λειτουργίας από μπαταρία 9 volt.
        Απαιτείται πολύ μικρή σύζευξη στο έδαφος στο τέλος της λήψης. Το άκρο μετάδοσης οδηγεί
        κυρίως στην κατανεμημένη χωρητικότητα του καλωδίου στη γείωση. Τα πολύ μεγάλα καλώδια
        (χιλιόμετρα) θα απαιτήσουν αρκετά καλή σύζευξη στη γείωση στο άκρο μετάδοσης.
        Προφανώς, οι αμφίδρομες επικοινωνίες πάνω από ένα πολύ μακρύ σύρμα θα απαιτήσουν
        αρκετά καλούς λόγους και στα δύο άκρα, όπως ένα μικρό πείρο σε βρωμιά ή ένα γυμνό
        σύρμα στο νερό.
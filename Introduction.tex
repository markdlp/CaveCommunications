\section{\textsf{Εισαγωγή}}
% what has not been done in the sector, like there are technologies in lace but not
% for a specific matter discussed. \textit{\textbf{Ιστορικό}}: 
    Από την εμφάνιση των πρώτων κινητών επικοινωνιών, πολλοί εξερευνητές, όπως και
    πολλοί διασώστες επιχειρούν να τις χρησιμοποιούν προς όφελός τους· πράγμα το οποίο
    αποδεικνύεται δύσκολο λόγω της πρόκλησης που παρουσιάζεται κατά την διάδοση
    ηλεκτρομαγνητικών κυμάτων σε υπόγεια περιβάλλοντα και γενικότερα σε περιβάλλοντα
    με περίπλοκη μορφολογία. Αξίζει ακόμη να σημειωθεί οτι ενώ δεν υπάρχει πολύ
    βιβλιογραφία για συνήθεις σπήλαια και ερείπια, έχει γίνει αρκετή έρευνα για
    ορυχεία ανθρακα, παρότι έχουν διαφορετικά χαρακτηριστικά.

% \maintxt{
    Η τρέχουσα γενιά ασύρματων δικτύων, όπως το 4G και το 5G, έχουν κατασκευαστεί
    βασισμένα σε αρχές υψηλού εύρους ζώνης, χαμηλής καθυστέρησης και υψηλής
    διαθεσιμότητας για να ανταποκριθούν στις απαιτήσεις μεγάλου αριθμού πελατών. Αυτά
    τα δίκτυα έχουν σχεδιαστεί για να υποστηρίζουν μεταφορά δεδομένων υψηλής
    ταχύτητας, επικοινωνία χαμηλής καθυστέρησης και εφαρμογές σε πραγματικό χρόνο.
% }
% \maintxt{ %\textit{\textbf{Στόχοι}}: 
    Ωστόσο, οι στόχοι των υπόγειων ασύρματων δικτύων επικοινωνίας είναι διαφορετικοί
    από εκείνους των υπαρχόντων ασύρματων δικτύων. Λόγω του μικρού αριθμού πελατών
    (κινητών τερματικών), του μικρού εύρους ζώνης και της υψηλής ανοχής καθυστέρησης,
    αυτά τα δίκτυα θα πρέπει να σχεδιάζονται με διαφορετικές αρχές κατά νου. Σε
    υπόγεια περιβάλλοντα, τα ασύρματα δίκτυα πρέπει να είναι αξιόπιστα, αποτελεσματικά
    και ανεκτικά σε υψηλή εξασθένηση και απώλεια σήματος.
% }
% \maintxt{ %\textit{\textbf{Σκοπιμότητα}}: 
    Αυτή η εργασία παρουσιάζει μια μελέτη σκοπιμότητας για τη χρήση υπόγειων ασύρματων
    δικτύων επικοινωνίας σε περιβάλλοντα σπηλαίων. Η μελέτη περιλαμβάνει την ανάλυση
    των υπαρχόντων τεχνολογιών που έχουν προταθεί και υλοποιηθεί, τα χαρακτηριστικά
    διάδοσης του σήματος σε υπόγεια περιβάλλοντα καθώς και το σχεδιασμό πρωτοκόλλων
    επικοινωνίας για την επίτευξη αξιόπιστης και αποτελεσματικής επικοινωνίας.
% }

    \subsection{\textsf{Το πρόβλημα}} 
        Το μεγαλύτερο πρόβλημα στις υπόγειες
        επικοινωνίες εγγυάται στο υψηλό pathloss λόγο της πυκνότητας του μέσου διαδοσης. Με
        άλλα λόγια το μεγαλύτερο μέρος της έρευνας που έχει λάβει τόπο στις ασύρματες
        επικοινωνίες λαμβάνει ώς μέσο διάδοσης τον κενό χώρος (δηλ. τον αέρα), αυτό διότι
        υπάρχει προβλεψιμότητα στην μοντελοποίηση και επειδή επιτρέπει διάδοση σε
        υψηλότερες συχνότητες μεγαλύτερο εύρος ζώνης και άλλα.\\
        Όταν ωστόσο όταν το αντικείμενο της μελέτης διαφέρει από την μεγιστοποίηση της
        διαδιδόμενης πληροφορίας αλλά εγγυάται στην αξιοπιστία και στην μεγιστοποιηση της
        θωράκισης σε μέσα με υψηλή απώλεια σήματος, προκύπτει ανάγκη για έρευνα σημάτων σε
        διαφορετικές συχνότητες και κωδικοποιήσεις.\\
        Ο σκοπός της παρούσας μελέτης είναι η -- σε περιβάλλον σπηλαίου , συντρίμμια
        φυσικές καταστροφές και επικοινωνία με το εξωτερικό περιβάλλον ή/και με προσωπικό
        διάσωσης.

        Η προτεινόμενη λύση με κινητούς αναμεταδότες που προτείνεται είναι πολύ χρήσιμη
        για επιχειρήσεις διάσωσης σε περιπτώσεις φυσικών καταστροφών ή/και επικοινωνίας σε
        υπόγειο βάθος.

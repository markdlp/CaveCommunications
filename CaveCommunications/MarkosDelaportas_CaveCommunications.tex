\documentclass[12pt]{article}
\usepackage{geometry}
\geometry{ b4paper, total={220mm,320mm}, left=20mm, top=15mm }

%\usepackage[parfill]{parskip}  % Activate to begin paragraphs with an empty line rather
%than an indent

\usepackage{mathtools}
\usepackage{blindtext}
\usepackage{multicol}
\usepackage{listings}
\usepackage{tikz}

\usepackage{fancyhdr, lastpage, setspace}

\usepackage{pgfplots}
\pgfplotsset{compat=1.18}

\usepackage{hyperref}
\hypersetup{
    bookmarksopen=true,
    colorlinks=true,
    linkcolor=black,
    filecolor=magenta,      
    urlcolor=black,
    pdftitle={Cave Communications}
}

\usepackage{graphicx}
\usepackage{amssymb}
\usepackage{enumitem}
\usepackage{amsmath}

\usepackage[acronym]{glossaries}
\usepackage{fontspec}
\usepackage{microtype}

\usepackage{polyglossia}
\usepackage[ backend=biber, style=ieee, sorting=ynt ]{biblatex}
\addbibresource{ref.bib}

%\setdefaultlanguage{greek} \setotherlanguages{english}

% \setmainlanguage{greek} \setotherlanguage{english}
% \setTransitionsForGreek{\selectlanguage{greek}}{\selectlanguage{english}}

% \defaultfontfeatures{Mapping=tex-text}


%\setmainfont[Kerning=On,Mapping=tex-text]{Linux Libertine}
\setmainfont{Georgia}
\setsansfont{Segoe UI Variable}
\setmonofont{Corbel}

%\newfontfamily\greekfont[Script=Greek]{Linux Libertine}
%\newfontfamily\greekfontsans[Script=Greek]{Fira Code}

%\DeclareTextFontCommand{\maintxt}{\greekfont}
%\DeclareTextFontCommand{\titletxt}{\greekfontsans}

\renewcommand{\thesection}{\Roman{section}}
\renewcommand{\thesubsection}{\thesection.\Roman{subsection}}

\onehalfspacing

\lhead{Μάρκος Δελαπόρτας} \rhead{ece01316@uowm.gr} \cfoot{Page \thepage\ of
\pageref{lastpage}}

\title{ \textsf{ Cave Communications}\\
    \textsf{Δίκτυα Νέας Γενιάς \& Επικοινωνίες}\\
    \textsf{\Large Τμήμα Ηλεκτρολόγων Μηχανικών \& Μηχανικών Υπολογιστών}\\
    \textsf{\large Πανεπιστήμιο Δυτικής Μακεδονίας}
} \author{\textsf{Μάρκος Δελαπόρτας} \footnote{E-mail: ece01316@uowm.gr}}
\date{\textsf{Ιανουάριος 2024}}

\begin{document}
\maketitle
\tableofcontents


\begin{multicols*}{2}
    \scriptsize \textbf{ \textit{Abstract} -- Τα ασύρματα δίκτυα επικοινωνίας αισθητήρων
        έχουν γίνει πανταχού παρόντα, τόσο στην καθημερινή ζωή, όσο και σε πολλές
        βιομηχανίες. Ωστόσο η αποτελεσματικότητά τους μπορεί να παρεμποδιστεί σε ορισμένα
        περιβάλλοντα όπως για παράδειγμα στην εξερεύνηση σπηλαίων, στην διάσωση ανθρώπων
        από κατεστραμμένα κτήρια ή στον τομέα της γεωργίας ακριβείας όπου τα σπαρτά
        βρίσκονται κάτω από την επιφάνεια του εδάφους. Με άλλα λόγια υπάρχει ανάγκη για
        ασύρματη επικοινωνία σε υπόγειο περιβάλλον. Στο παρόν άρθρο εξετάζονται διάφορες
        τεχνικές ασύρματης επικοινωνίας, οι περιορισμοί και οι στερήσεις που έχουν γίνει
        προκειμένου να μοντελοποιηθεί σε περιβάλλον σπηλαίων από διάφορες τεχνολογίες που
        έχουν προταθεί. Καθώς επίσης και πως με αλλαγές στην ισχύ και την συχνότητα μπορεί
        να μειωθεί η απώλεια μονοπατιού και η ισχύς του σήματος λήψης. }
    

    \section{\normalsize \textsf{Εισαγωγή}}
    % what has not been done in the sector, like there are technologies in lace but not
    % for a specific matter discussed. \textit{\textbf{Ιστορικό}}: 
        Από την εμφάνιση των πρώτων κινητών επικοινωνιών, πολλοί εξερευνητές, όπως και
        πολλοί διασώστες επιχειρούν να τις χρησιμοποιούν προς όφελός τους· πράγμα το οποίο
        αποδεικνύεται δύσκολο λόγω της πρόκλησης που παρουσιάζεται κατά την διάδοση
        ηλεκτρομαγνητικών κυμάτων σε υπόγεια περιβάλλοντα και γενικότερα σε περιβάλλοντα
        με περίπλοκη μορφολογία. Αξίζει ακόμη να σημειωθεί οτι ενώ δεν υπάρχει πολύ
        βιβλιογραφία για συνήθεις σπήλαια και ερείπια, έχει γίνει αρκετή έρευνα για
        ορυχεία ανθρακα, παρότι έχουν διαφορετικά χαρακτηριστικά.
    
    % \maintxt{
        Η τρέχουσα γενιά ασύρματων δικτύων, όπως το 4G και το 5G, έχουν κατασκευαστεί
        βασισμένα σε αρχές υψηλού εύρους ζώνης, χαμηλής καθυστέρησης και υψηλής
        διαθεσιμότητας για να ανταποκριθούν στις απαιτήσεις μεγάλου αριθμού πελατών. Αυτά
        τα δίκτυα έχουν σχεδιαστεί για να υποστηρίζουν μεταφορά δεδομένων υψηλής
        ταχύτητας, επικοινωνία χαμηλής καθυστέρησης και εφαρμογές σε πραγματικό χρόνο.
    % }
    % \maintxt{ %\textit{\textbf{Στόχοι}}: 
        Ωστόσο, οι στόχοι των υπόγειων ασύρματων δικτύων επικοινωνίας είναι διαφορετικοί
        από εκείνους των υπαρχόντων ασύρματων δικτύων. Λόγω του μικρού αριθμού πελατών
        (κινητών τερματικών), του μικρού εύρους ζώνης και της υψηλής ανοχής καθυστέρησης,
        αυτά τα δίκτυα θα πρέπει να σχεδιάζονται με διαφορετικές αρχές κατά νου. Σε
        υπόγεια περιβάλλοντα, τα ασύρματα δίκτυα πρέπει να είναι αξιόπιστα, αποτελεσματικά
        και ανεκτικά σε υψηλή εξασθένηση και απώλεια σήματος.
    % }
    % \maintxt{ %\textit{\textbf{Σκοπιμότητα}}: 
        Αυτή η εργασία παρουσιάζει μια μελέτη σκοπιμότητας για τη χρήση υπόγειων ασύρματων
        δικτύων επικοινωνίας σε περιβάλλοντα σπηλαίων. Η μελέτη περιλαμβάνει την ανάλυση
        των υπαρχόντων τεχνολογιών που έχουν προταθεί και υλοποιηθεί, τα χαρακτηριστικά
        διάδοσης του σήματος σε υπόγεια περιβάλλοντα καθώς και το σχεδιασμό πρωτοκόλλων
        επικοινωνίας για την επίτευξη αξιόπιστης και αποτελεσματικής επικοινωνίας.
    % }

    \section{\normalsize \textsf{Το πρόβλημα}} Το μεγαλύτερο πρόβλημα στις υπόγειες
        επικοινωνίες εγγυάται στο υψηλό pathloss λόγο της πυκνότητας του μέσου διαδοσης. Με
        άλλα λόγια το μεγαλύτερο μέρος της έρευνας που έχει λάβει τόπο στις ασύρματες
        επικοινωνίες λαμβάνει ώς μέσο διάδοσης τον κενό χώρος (δηλ. τον αέρα), αυτό διότι
        υπάρχει προβλεψιμότητα στην μοντελοποίηση και επειδή επιτρέπει διάδοση σε
        υψηλότερες συχνότητες μεγαλύτερο εύρος ζώνης και άλλα.\\
        Όταν ωστόσο όταν το αντικείμενο της μελέτης διαφέρει από την μεγιστοποίηση της
        διαδιδόμενης πληροφορίας αλλά εγγυάται στην αξιοπιστία και στην μεγιστοποιηση της
        θωράκισης σε μέσα με υψηλή απώλεια σήματος, προκύπτει ανάγκη για έρευνα σημάτων σε
        διαφορετικές συχνότητες και κωδικοποιήσεις.\\
        Ο σκοπός της παρούσας μελέτης είναι η -- σε περιβάλλον σπηλαίου , συντρίμμια
        φυσικές καταστροφές και επικοινωνία με το εξωτερικό περιβάλλον ή/και με προσωπικό
        διάσωσης.

        Η προτεινόμενη λύση με κινητούς αναμεταδότες που προτείνεται είναι πολύ χρήσιμη
        για επιχειρήσεις διάσωσης σε περιπτώσεις φυσικών καταστροφών ή/και επικοινωνίας σε
        υπόγειο βάθος.

    \section{\normalsize \textsf{Σχετικές Έρευνες}} Στην βιβλιογραφία έχουν συνταχθεί
        αρκετές εργασίες σχετικά με την ασύρματη επικοινωνία σε σπήλαια και υπόγεια
        περιβάλλοντα. Ωστόσο πρέπει να σημειωθεί οτι δεδομένου του εύρους όλων των 
        σεναρίων/περιπτώσεων που ενέχει η έρευνα υπόγειων επικοινωνιών στην βιβλιογραφία
        μπορεί να βρεθεί μια πληθώρα από μοντέλα που διαφέρουν σχετικά με τα προβλήματα που
        καλούνται να επιλύσουν. Με άλλα λόγια ενώ παρατίθενται διάφορες έρευνες για το 
        συγκεκριμένο ζήτημα κρίνεται απαραίτητο να ομαδοποιηθούν με βάση ορισμένες παραμέτρους
        ώς προς το πρόβλημα και τα φυσικά χαρακτηριστικά του συστήματος.
        
        Στην έρευνα των Guozheng Zhao, Kaiqiang Lin, Tong Hao \cite*{zhao_feasibility_2023}
        χρησιμοποιήθηκαν τερματικά που ξεκινούν να δέχονται κατερχόμενη σύνδεση αμέσως μετά 
        από μετάδοση και για ένα ορισμένο χρονικό διάστημα σε ένα LoRa-WAN δίκτυο.
        Για την μοντελοποίηση του καναλιού Above Ground to Underground (AG2UG) \& 
        Underground to Above Ground (UG2AG) λήφθηκε υπ'όψη και εξασθένηση λόγω πλευρικών
        κυμάτων, έτσι η λαμβανόμενη ισχύς υπολογίζεται από τον τύπο:\\
        \begin{equation} \label{eq:1}
            P_r = P_t + G_t + G_r + [L_{ug}(d_{ug}) + aL_{ag}(d_{ag}) + bL_{surface}(d_{surface}) + L_R - 10\log\chi^2]    
        \end{equation} 
        
        Και ανάλογα με τις φυσικές παραμέτρους της τοπολογίας (π.χ. αποστάσεις μεταξύ
        τερματικών, αποστάσεις μεταξύ κόμβων και τερματικών, κ.α.) επιλέγονται οι 
        σταθερές εξασθένησης a \& b, συγκεκριμένα εξαρτώνται από την διηλεκτρική 
        σταθερά του εδάφους.


        Πιο συγκεκριμένα ο Muhammed Enes Bayrakdar έδειξε έναν τρόπο για
        συσκευές του δικτύου των πραγμάτων να επικοινωνούν σε ένα δίκτυο πλέγματος
        \cite{bayrakdar_rule_2019}. Οι Mark Hedley και Ian Gipps ανέδειξαν έναν τρόπο για
        ακριβή προσδιορισμό θέσης σε υπόγεια ορυχεία \cite{hedley_accurate_2013}. Οι
        Manoja D. Weiss και Kevin Moore ανέδειξαν έναν τρόπο για αυτόνομη κινητή
        τηλεπικοινωνία και ασύρματη πρόσδεση σε τούνελ \cite{weiss_autonomous_2009}. Οι
        William Walsh και Jay Gao ανέλυσαν την δυνατότητα χρήσης wifi σε περιβάλλον
        σπηλαίου \cite{walsh_communications_2018}. Ο Philip Branch συνέταξε την
        διπλωματική του για κυψελωδη δίκτυα πέμπτης γενιάς σε υπόγεια ορυχεία βαρύτιτας
        \cite{branch_fifth_2021}. Οι M.I. Martínez-Garrido και R. Fort ανέπτυξαν υλισμικό
        για πειράματα σε ανθρωπογενής και φυσική κληρονομιά
        \cite{martinez-garrido_experimental_2016}.

        Ο βασικός σκοπός του In-Cave Wireless
        Sensor Networks (ICWCS) είναι να παρέχει ένα αξιόπιστο κανάλι φωνητικής επικοινωνίας
        μέσω ενός δικτύου πολυμέσων μεταξύ των μελών μιας ομάδας σε μια σπηλιά. Αποτελείται
        από δύο διαφορετικούς τύπους κόμβων. 
        \begin{itemize}
            \item «κόμβους κορμού» που είναι τοποθετημένο στα τοιχώματα του σπηλαίου.
            \item «Κινητοί κόμβοι», οι οποίοι έχουν παρόμοια λειτουργικότητα όπως τα κινητά
            τηλέφωνα, φέρονται από τα μέλη της ομάδας.
        \end{itemize}

        Οι κόμβοι κορμού είναι ακίνητοι
        κόμβοι και ο σχεδιασμός τους βασίζεται στον ασύρματο αισθητήρα γενικής
        χρήσης «VF1A» \cite*{walsh_communications_2018}.
        Αυτοί οι κόμβοι είναι υπεύθυνοι για διατήρηση της συνδεσιμότητας κορμού μέσα 
        στο σπήλαιο, τις δραστηριότητες
        εγγραφής και περιαγωγής κινητών κόμβων, δρομολόγηση πληροφοριών κλήσεων μεταξύ
        κινητών κόμβων και μεταφορά των δεδομένων ελέγχου και φωνής μέσω του ασύρματου
        δικτύου.

    \subsection{\normalsize \textsf{Συγκριτική ανάλυση}}
        Οι παράμετροι με βάση τις οποίες θα συγκριθούν οι έρευνες είναι οι εξής:
            \begin{itemize}
                \item Pathloss
                \item Received Signal Power
                \item Data Extraction Rate (DER)
                \item Goodput
                \item Energy per Packet (E/P)
                \item Volumetric Water Content (VWC) of the soil
                \item Transmit Power
                \item Spreading Factor
                \item Coding Rate (the proportion of the useful
                parts of the data stream)
                \item Bandwidth
            \end{itemize}

        Ακόμη είναι σημαντικό στην μέτρηση της απόδοσης να επικρατεί κοινός
        σχεδιασμός δικτύου σε όλα τα πειράματα προκειμένου να αποφευχθεί προκατάληψη
        των μετρήσεων αν σε περίπτωση σύγκρουσης των πακέτων.Όπως στην έρευνα για την
        σκοπιμότητα των LoRaWAN σε υπόγεια περιβάλλοντα \cite*{zhao_feasibility_2023}
        οι δέκτες όταν λαμβάνουν ισχύ (\ref{eq:1}) μεγαλύτερη από την ισχύ ευαισθησίας,
        αυτόματα αναγνωρίζουν το πακέτο ώς χαμένο καθώς έχει γίνει σύγκρουση. Ωστόσο έχει
        αναπτυχθεί αλγόριθμος για την αναγνώριση αν όντως έχει γίνει σύγκρουση.
        Οι παράμετροι φυσικού επιπέδου στους δέκτες (gateways) να είναι:
        \begin{itemize}
            \item Ισχύς Μετάδοσης: 14dBm
            \item Παράγοντας Εξάπλωσης: 12
            \item Coding Rate: 4/8 (50\%)
            \item Εύρος Ζώνης: 125kHz
        \end{itemize}

        Ως παράμετρο προσομοίωσης έχει οριστεί το Goodput το οποίο είναι ένα μέγεθος
        που περιγράφει την ταχύτητα των δεδομένων από ένα τερματικό στον λήπτη και 
        ορίζεται ώς: 
        \begin{equation}\label{eq:2}
            Goodput = \frac{N_{arrived} * PL}{T_{simulation}}
        \end{equation}\\
        Όπου PL είναι το μήκος του πακέτου.
        Ακόμη μια παράμετρος που χρησιμοποιήθηκε είναι το DER, δηλαδή τον ρυθμό με τον οποίο \dots
        Τέλος η παράμετρος με την οποία αξιολογήθηκε το μοντέλο είναι η ενέργεια ανά πακέτο. 



        

    \section{\normalsize \textbf{Συμπεράσματα}}
        Συμπερασματικά, η εργασία καταδεικνύει τη σκοπιμότητα των
        υπόγειων ασύρματων δικτύων επικοινωνίας σε υπόγεια περιβάλλοντα. Αυτά τα δίκτυα
        μπορούν να υποστηρίξουν μικρό αριθμό πελατών, με μικρά εύρη ζώνης και υψηλή ανοχή
        καθυστέρησης, και μπορούν να σχεδιαστούν ώστε να είναι αξιόπιστα και αποτελεσματικά σε
        τέτοια απαιτητικά περιβάλλοντα. Η μελέτη παρέχει ένα σημείο εκκίνησης για μελλοντική
        έρευνα σε αυτόν τον τομέα.

        Από την έρευνα για την σκοπιμότητα των LoRaWAN σε υπόγεια περιβάλλοντα 
        \cite{zhao_feasibility_2023} οι συντάκτες παραθέτουν τα αποτελέσματα τις προσομοίωσης
        και τις διαφορές που έχουν στο σύστημα οι μεταβλητές περιβάλλοντος. Πιο συγκεκριμένα
        φαίνεται ότι το LoRaWAN είναι πιο συμπαγές σε σχέση με άλλα υπόγεια ασύρματα συστήματα
        αισθητήρων (WUSNs). Πιο συγκεκριμένα φάνηκε ότι παρά ενδεχόμενες κακές συνθήκες 
        περιβάλλοντος (π.χ. Volumetric Water Content = 50\%)

    \section{\normalsize  \textsf{Συζήτηση}}
        Σε θεωρητικό επίπεδο εχουν προταθεί τεχνολογίες extremely low frequency
        (elf) οι οποίες είναι ήδη σε χρήση σε υποβρύχια επειδή τα ραδιοκύματα ELF μπορούν να
        διεισδύσουν στο θαλασσινό νερό σε πολύ μεγαλύτερο βάθος από τα ραδιοκύματα υψηλότερης
        συχνότητας, τα οποία απορροφώνται από το νερό. Τα κύματα ELF (3 - 30Hz) έχουν πολύ
        μεγάλα μήκη κύματος, που κυμαίνονται από εκατοντάδες έως χιλιάδες χιλιόμετρα, και
        παράγονται από μια μεγάλη κεραία που ονομάζεται «δίπολο εδάφους» ή «δίπολο γης». Οι
        κεραίες που χρησιμοποιούνται για την επικοινωνία ELF έχουν συνήθως μήκος πολλών
        χιλιομέτρων και συχνά βρίσκονται σε απομακρυσμένες περιοχές με χαμηλά επίπεδα
        ανθρωπογενών παρεμβολών.

        Ωστόσο απ'οτι φάνηκε στην έρευνα για LoRaWAN \cite*{zhao_feasibility_2023}, μεγάλες 
        αποστάσεις μπορούν να καλυφθούν αλλά συγκεκριμένες περιπτώσεις όπως η γεωργία ακριβείας.
        Αυτό διότι το LoRaWAN είναι μια τεχνολογία που ελαχιστοποιεί την κατανάλωση ενέργειας και
        μεγιστοποιεί στην εμβέλεια, με αντάλλαγμα τον μειωμένο ρυθμό δεδομένων, δηλαδή τον χαμηλό
        ρυθμό διάδοσης (ενδεικτικά κάθε 30 λεπτά σε βάθος 40 εκατοστών με VWC=40\%). Με άλλα
        λόγια το LoRaWAN είναι μια πολύ αξιόπιστη επιλογή, που ωστόσο υστερεί σε περιπτώσεις 
        χρήσης που χρειάζεται γρήγορη επικοινωνία σε υπόγειο περιβάλλον βαθύτερο των τριών μέτρων.

    \subsection{}
        Σε πειραματικό επίπεδο έχουν υλοποιηθεί και δοκιμαστεί ορισμένα πρότυπα επικοινωνίας
        τα οποία βασίζονται σε κάποιες από τις αρχές που έχουν προταθεί παραπάνω. Πιο
        συγκεκριμένα το HeyPhone είναι ένα φορητό ραδιόφωνο που χρησιμοποιεί μια κεραία βρόχου
        με διάμετρο ένα μέτρο και έχει την ικανότητα να διαπεράσει το έδαφος σε βάθος μέχρι
        και 500 περίπου μέτρα, χρησιμοποιεί διαμόρφωση 87KHz Single Side Band (SSB).

    \subsection{}
        Ακόμη μια τεχνολογία που χρησιμοποιείται ευρέως στην βιομηχανία είναι το τηλέφωνο
        ενός καλωδίου (Single Wire Telephone). Οι συσκευές είναι πολύ απλές, χρησιμοποιώντας
        ένα μόνο op-amp τόσο για να στείλουν όσο και να λάβουν ένα ηχητικό σήμα (φωνή) κατά
        μήκος ενός μόνο μονωμένου καλωδίου, χρησιμοποιώντας τη χωρητικότητα σύζευξης χειριστών
        στη γείωση για την επιστροφή. Ο δέκτης έχει πολύ υψηλή αντίσταση εισόδου. Η ζήτηση
        ισχύος είναι πολύ χαμηλή, με πολλές ώρες (ή ημέρες) λειτουργίας από μπαταρία 9 volt.
        Απαιτείται πολύ μικρή σύζευξη στο έδαφος στο τέλος της λήψης. Το άκρο μετάδοσης οδηγεί
        κυρίως στην κατανεμημένη χωρητικότητα του καλωδίου στη γείωση. Τα πολύ μεγάλα καλώδια
        (χιλιόμετρα) θα απαιτήσουν αρκετά καλή σύζευξη στη γείωση στο άκρο μετάδοσης.
        Προφανώς, οι αμφίδρομες επικοινωνίες πάνω από ένα πολύ μακρύ σύρμα θα απαιτήσουν
        αρκετά καλούς λόγους και στα δύο άκρα, όπως ένα μικρό πείρο σε βρωμιά ή ένα γυμνό
        σύρμα στο νερό.

    \printbibliography
\end{multicols*}
\end{document}

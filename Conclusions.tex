\section{\textsf{Συμπεράσματα}}
        Συμπερασματικά, η εργασία καταδεικνύει τη σκοπιμότητα των
        υπόγειων ασύρματων δικτύων επικοινωνίας σε υπόγεια περιβάλλοντα. Αυτά τα δίκτυα
        μπορούν να υποστηρίξουν μικρό αριθμό πελατών, με μικρά εύρη ζώνης και υψηλή ανοχή
        καθυστέρησης, και μπορούν να σχεδιαστούν ώστε να είναι αξιόπιστα και αποτελεσματικά σε
        τέτοια απαιτητικά περιβάλλοντα. Η μελέτη παρέχει ένα σημείο εκκίνησης για μελλοντική
        έρευνα σε αυτόν τον τομέα.

        Από την έρευνα για την σκοπιμότητα των LoRaWAN σε υπόγεια περιβάλλοντα 
        \cite{zhao_feasibility_2023} οι συντάκτες παραθέτουν τα αποτελέσματα τις προσομοίωσης
        και τις διαφορές που έχουν στο σύστημα οι μεταβλητές περιβάλλοντος. Πιο συγκεκριμένα
        φαίνεται ότι το LoRaWAN είναι πιο συμπαγές σε σχέση με άλλα υπόγεια ασύρματα συστήματα
        αισθητήρων (WUSNs). Πιο συγκεκριμένα φάνηκε ότι παρά ενδεχόμενες κακές συνθήκες 
        περιβάλλοντος (π.χ. Volumetric Water Content = 50\%) το σύστημα λειτουργούσε αξιόπιστα.

        Στην έρευνα για τα UHF \cite{rak_uhf_2007} οι συγγραφείς παρουσίασαν βασικές οδηγίες
        για την εκτίμηση της ειδικής εξασθένησης και της μέγιστης εμβέλειας για ασύρματες
        επικοινωνίες σε υπόγειες γκαλερί διαφόρων προφίλ και τύπων τοίχων. Πρότειναν επίσης 
        ότι τα αποτελέσματά τους θα μπορούσαν να χρησιμοποιηθούν ως πειραματική βάση για 
        περαιτέρω θεωρητικές εργασίες.

        Στα αποτελέσματα της διπλωματικής εργασίας για τα δίκτυα πέμπτης γενιάς \cite{branch_fifth_2021}
        φαίνεται η υπεροχή των 5G σε σύγκριση με το wifi αλλά και πως μπορεί να συμπληρώσει
        την παραδοσιακή VHF ραδιοφωνία σε Peer to Peer (P2P) επιτρέποντας την μετάδοση βίντεο.
        Τέλος αναφέρει ότι το περιβάλλον διάδοσης των ορυχείων είναι περίπλοκο με κάποιες ενδείξεις
        οτι η διάδοση είναι λιγότερο δριμεία οταν το μονοπάτι περιλαμβάνει άξονες εξαγωγής με
        επένδυση από χάλυβα.

        Η έρευνα που μοντελοποίησε δίκτυα IoT σε ορυχεία \cite{ming_study_2019} ανέδειξε οτι 
        προκειμένου να υλοποιηθεί αποδοτικά η επικοινωνία με μέσο διάδοσης την πέτρα, δε μπορεί
        να υπερβαίνει τα έξι μέτρα. Ακόμη φάνηκε οτι ένα νευρωνικό δίκτυο προς τα πίσω διάδοσης
        επιστρέφει πιο ακριβή αποτελέσματα από απλό τριγωνικό εντοπισμό.

        Όπως φάνηκε από την έρευνα για ασύρματη διάδοση σε κυκλικό τούνελ \cite{dudley_wireless_2005}
        13 λειτουργίες είναι πάνω από την συχνότητα αποκοπής για τέλεια αγώγιμα τοιχώματα. Ακόμη
        χρησιμοποιούνται 16 λειτουργίες για τη διασφάλιση της σύγκλισης στην περίπτωση τοίχου 
        με απώλειες.

        Τα αποτελέσματα της προσομοίωσης ανίχνευσης ακτίνων \cite{soo_investigation_2018} δείχνουν
        τη διαφορά στην ανάκλαση του εδάφους μεταξύ λείων και τραχιών επιφανειών. Αλλά επίσης και
        πως ο τροποποιημένος, με συντελεστή εξασθένησης, συντελεστής ανάκλασης Fresnel χρησιμοποιείται
        για τραχιές επιφάνειες και πώς η ισχύς του σήματος επηρεάζεται από την απόσταση και είναι 
        χαμηλότερη σε ανώμαλες συνθήκες επιφάνειας.

        Όπως και σε άλλες έρευνες οι κινητοί κόμβοι έχουν παρόμοια λειτουργία με τα κινητά τερματικά
        ενός κυψελωτού δικτύου. Πιο συγκεκριμένα στην έρευνα για ασύρματο δίκτυο επικοινωνίας σε
        σπήλαια \cite{yavuz_-cave_2009} υλοποιείται ένα αξιόπιστο δίκτυο φωνητικής επικοινωνίας
        χρησιμοποιώντας ασύρματους κόμβους αισθητήρων πολυμέσων σε εσωτερικά περιβάλλοντα με δύο
        τύπους ασύρματων κόμβων αισθητήρων: κόμβοι κορμού και κινητοί κόμβοι. Η τρέχουσα υλοποίηση
        χειρίζεται μια ενεργή φωνητική επικοινωνία από σημείο σε σημείο.

        Πιο πειραματικά σε ένα σπήλαιο που μετατράπηκε σε κελάρι \cite{soo_propagation_2018}
        παρατηρήθηκε ότι τα σήματα χαμηλότερης συχνότητας είναι πιο πρακτικά για ασύρματη
        επικοινωνία σε σπηλιές. Πιο συγκεκριμένα η κάθετη συν-πόλωση (VV) έχει την καλύτερη
        λαμβανόμενη ισχύ στα περισσότερα σενάρια.
        Είναι σημαντικό να γίνει ο διαχωρισμός μεταξύ σπηλαίων τουριστικού σκοπού και των σπηλαίων
        εξερευνητικού/ορειβατικού σκοπού. Αρχικά επειδή οι τουριστικές σπηλιές έχουν κίνηση ανθρώπων
        και μεγαλύτερες διαστάσεις σε σύγκριση με τις άγριες σπηλιές \cite{soo_propagation_2018}.
        Ακόμη η επιφάνεια του εδάφους των τουριστικών σπηλαίων είναι πιο λεία ενώ αντίθετα τα φυσικά
        περάσματα των σπηλαίων έχουν τραχιές επιφάνειες και ανομοιομορφίες διαστάσεων.

        Τα αποτελέσματα μέτρησης πεδίου που λαμβάνονται από το εσωτερικό του κελαριού του
        Jeff \cite{soo_propagation_2018} σε τρεις συχνότητες είναι τα εξής: η ισχύς του σήματος
        μειώνεται καθώς ο δέκτης απομακρύνεται από τον πομπό, το σήμα 900 MHz μπορεί να διαδοθεί
        ισχυρότερα για μεγαλύτερη απόσταση ενώ το σήμα 5,8 GHz εξασθενεί πιο γρήγορα και φτάνει στο 
        επίπεδο θορύβου.

        Στην πλήρη ανάλυση ραδιοσυχνοτήτων \cite{pingenot_full_2005} συλλέχθηκαν στατιστικά στοιχεία
        για τις ιδιότητες διάδοσης και εξασθένησης του περιβάλλοντος του σπηλαίου. Ακόμη η φασματική
        πυκνότητα ισχύος και η φάση των συντελεστών του διανυσματικού ηλεκτρικού πεδίου. Δεν βρέθηκε
        κάποια σημαντική διαφοροποίηση στα αποτελέσματα σε όλο το φάσμα. Για το πρωτεύον διαδιδόμενο
        πεδίο ($E_z$) το διάσπαρτο πεδίο συμπληρώνει μηδενικά (nulls) που στην ομαλή περίπτωση 
        δημιουργούνται από καταστροφικές παρεμβολές. Τέλος η διασπορά πόλωσης αυξάνει την ενέργεια
        των $E_x$ και $E_y$ περαιτέρω μέσα στο σπήλαιο, πράγμα αναμενόμενο.

        Η μέτρηση μοντέλου ανώμαλου εδάφους για πιθανές εφαρμογές σε σπήλαια \cite{soo_measurement_2019}
        ανέδειξε ότι η ισχύς του σήματος πέφτει καθώς ο δέκτης απομακρύνεται από τον πομπό και πιο
        συγκεκριμένα η δραστική πτώση του σήματος σε τοποθεσία μεταξύ 18 και 22 m μετά το σήμα LOS
        που στρίβει σε γωνία. Ακόμη η διαδρομή σήματος NLOS μπορεί να ταξιδέψει μόνο μέσω ανάκλασης,
        περίθλασης και διάθλασης. Τα προσομοιωμένα και μετρημένα αποτελέσματα δείχνουν καλύτερη συμφωνία
        με τον παράγοντα σκέδασης, δηλαδή βελτιωμένη τυπική απόκλιση από 6,84 dB σε 4,48 dB με συντελεστή
        σκέδασης.
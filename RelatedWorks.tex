\section{\textsf{Σχετικές Έρευνες}} Στην βιβλιογραφία έχουν συνταχθεί
        αρκετές εργασίες σχετικά με την ασύρματη επικοινωνία σε σπήλαια και υπόγεια
        περιβάλλοντα. Ωστόσο πρέπει να σημειωθεί οτι δεδομένου του εύρους όλων των 
        σεναρίων/περιπτώσεων που ενέχει η έρευνα υπόγειων επικοινωνιών στην βιβλιογραφία
        μπορεί να βρεθεί μια πληθώρα από μοντέλα που διαφέρουν σχετικά με τα προβλήματα που
        καλούνται να επιλύσουν. Με άλλα λόγια ενώ παρατίθενται διάφορες έρευνες για το 
        συγκεκριμένο ζήτημα κρίνεται απαραίτητο να ομαδοποιηθούν με βάση ορισμένες παραμέτρους
        ώς προς το πρόβλημα και τα φυσικά χαρακτηριστικά του συστήματος.
        
        Στην έρευνα των Guozheng Zhao, Kaiqiang Lin, Tong Hao \cite{zhao_feasibility_2023}
        χρησιμοποιήθηκαν τερματικά που ξεκινούν να δέχονται κατερχόμενη σύνδεση αμέσως μετά 
        από μετάδοση και για ένα ορισμένο χρονικό διάστημα σε ένα LoRa-WAN δίκτυο.
        Για την μοντελοποίηση του καναλιού Above Ground to Underground (AG2UG) \& 
        Underground to Above Ground (UG2AG) λήφθηκε υπ'όψη και εξασθένηση λόγω πλευρικών
        κυμάτων, έτσι η λαμβανόμενη ισχύς υπολογίζεται από τον τύπο:\\
        \begin{multline} \label{eq:1}
            P_r = P_t + G_t + G_r + \\ 
                [L_{ug}(d_{ug}) + aL_{ag}(d_{ag}) + bL_{surface}(d_{surface}) + L_R - 10\log\chi^2]
        \end{multline}
        
        Και ανάλογα με τις φυσικές παραμέτρους της τοπολογίας (π.χ. αποστάσεις μεταξύ
        τερματικών, αποστάσεις μεταξύ κόμβων και τερματικών, κ.α.) επιλέγονται οι 
        σταθερές εξασθένησης a \& b, συγκεκριμένα εξαρτώνται από την διηλεκτρική 
        σταθερά του εδάφους.

        Αντίστοιχα στην έρευνα για την διάδοση UHF σε υπόγεια σπήλαια \cite{rak_uhf_2007}
        προτάθηκε μια πειραματική μελέτη κατά την οποία η τηλεμετρία εκμεταλλεύεται την φυσική
        μορφολογία του υπόγειου περιβάλλοντος. Πιο συγκεκριμένα οι ερευνητές χρησιμοποίησαν
        κινητούς πομποδέκτες και κεραίες για να εκτελέσουν τα πειράματα τους, σε συχνότητες 
        446 και 860 MHz. Σε πέντε τοποθεσίες αναζήτησαν το μέγιστο επίπεδο σήματος σε όλο το
        προφίλ της γκαλερί σε κάθε σημείο μέτρησης.

        Οι συγγραφείς έβγαλαν ένα εμπειρικό γραμμικό μοντέλο απώλειας διαδρομής ως συνάρτηση
        της απόστασης και το συνέκριναν με ένα θεωρητικό μοντέλο κυματοδηγού που βασίζεται στις
        γεωμετρικές και ηλεκτρικές παραμέτρους του περιβάλλοντος. Βρήκαν ότι το μοντέλο κυματοδηγού
        ήταν κατάλληλο μόνο για κανονικές στοές και όχι για ακανόνιστα προφίλ που εμφανίζονται
        συχνά σε σπηλιές.
        
        Στην έρευνα για διερεύνηση τραχιών επιφανειών για διάδοση μοντελοποίηση σε σπηλιές
        \cite{soo_investigation_2018} επιδιώκει την παροχή ασύρματων επικοινωνιών σε σπήλαια
        για διάφορες δραστηριότητες, ενίσχυση της ασφάλειας σε σπηλιές για τουριστικό σκοπό
        και βοήθεια σε επιχειρήσεις έρευνας και διάσωσης. Με άλλα λόγια είναι ένα θεμέλιο για
        περαιτέρω έρευνα σε πραγματικά περιβάλλοντα σπηλαίων. Ωστόσο ο περιορισμός κατά την
        παρούσα ανάλυση είναι ότι ο συντελεστής εξασθένησης υπολογίζει εσφαλμένα την απώλεια
        σκέδασης για ορισμένες μετρήσεις και έτσι χρειάζεται να εισαχθεί ο μέσος συντελεστής
        ανάκλασης εισάγεται για να ληφθούν υπόψη οι τραχιές επιφάνειες.

        % ~~~~~~~~~~~~~~~~~~~~~~~~~~~~~~~~~~~~~~~~~~~~~~~~~~~~~~~~~~~~~~~~~~~~~~~~~~~~~~~~~~        
        Ακόμη ο Muhammed Enes Bayrakdar έδειξε έναν τρόπο για
        συσκευές του δικτύου των πραγμάτων να επικοινωνούν σε ένα δίκτυο πλέγματος
        \cite{bayrakdar_rule_2019}. Οι Mark Hedley και Ian Gipps ανέδειξαν έναν τρόπο για
        ακριβή προσδιορισμό θέσης σε υπόγεια ορυχεία \cite{hedley_accurate_2013}. Οι
        Manoja D. Weiss και Kevin Moore ανέδειξαν έναν τρόπο για αυτόνομη κινητή
        τηλεπικοινωνία και ασύρματη πρόσδεση σε τούνελ \cite{weiss_autonomous_2009}. Οι
        William Walsh και Jay Gao ανέλυσαν την δυνατότητα χρήσης wifi σε περιβάλλον
        σπηλαίου \cite{walsh_communications_2018}. Ο Philip Branch συνέταξε την
        διπλωματική του για κυψελωδη δίκτυα πέμπτης γενιάς σε υπόγεια ορυχεία βαρύτιτας
        \cite{branch_fifth_2021}. Οι M.I. Martínez-Garrido και R. Fort ανέπτυξαν υλισμικό
        για πειράματα σε ανθρωπογενής και φυσική κληρονομιά
        \cite{martinez-garrido_experimental_2016}.
        % ~~~~~~~~~~~~~~~~~~~~~~~~~~~~~~~~~~~~~~~~~~~~~~~~~~~~~~~~~~~~~~~~~~~~~~~~~~~~~~~~~~

        Ο βασικός σκοπός του In-Cave Wireless
        Sensor Networks (ICWCS) είναι να παρέχει ένα αξιόπιστο κανάλι φωνητικής επικοινωνίας
        μέσω ενός δικτύου πολυμέσων μεταξύ των μελών μιας ομάδας σε μια σπηλιά. Αποτελείται
        από δύο διαφορετικούς τύπους κόμβων. 
        \begin{itemize}
            \item «κόμβους κορμού» που είναι τοποθετημένο στα τοιχώματα του σπηλαίου.
            \item «Κινητοί κόμβοι», οι οποίοι έχουν παρόμοια λειτουργικότητα όπως τα κινητά
            τηλέφωνα, φέρονται από τα μέλη της ομάδας.
        \end{itemize}

        Οι κόμβοι κορμού είναι ακίνητοι
        κόμβοι και ο σχεδιασμός τους βασίζεται στον ασύρματο αισθητήρα γενικής
        χρήσης «VF1A» \cite{walsh_communications_2018}.
        Αυτοί οι κόμβοι είναι υπεύθυνοι για διατήρηση της συνδεσιμότητας κορμού μέσα 
        στο σπήλαιο, τις δραστηριότητες
        εγγραφής και περιαγωγής κινητών κόμβων, δρομολόγηση πληροφοριών κλήσεων μεταξύ
        κινητών κόμβων και μεταφορά των δεδομένων ελέγχου και φωνής μέσω του ασύρματου
        δικτύου.

        Άλλες πρακτικές εφαρμογές είναι η δημιουργία αξιόπιστου δικτύου φωνητικής επικοινωνίας
        σε προκλητικά περιβάλλοντα ραδιοσυχνοτήτων όπως στην έρευνα των A. Gokhan Yavuz, 
        Z. Cihan Taysi και Esra Celik \cite{yavuz_-cave_2009}. Επιπλέον, σχεδιάστηκαν
        ασύρματοι κόμβοι κορμού και κινητά τερματικά πολυμέσων (π.χ κείμενο, εικόνας) 
        αφιερώνοντας παραπάνω χρόνο σε αμφίδρομες φωνητικές επικοινωνίες με περαιτέρω
        δυνατότητες (multicast \& broadcast).

        Οι πρακτικές εφαρμογές της ανάλυσης πλήρους κύματος RF της εξασθένησης σήματος σε σπήλαιο
        με απώλειες χρησιμοποιώντας μέθοδο διακριτών στοιχείων υψηλής τάξης διανυσματικού τομέα
        χρόνου \cite{pingenot_full_2005} είναι η ενίσχυση της ασύρματης επικοινωνίας σε σπηλιές
        και σήραγγες και μειώνει τους κινδύνους για το στρατιωτικό και το προσωπικό διάσωσης σε τέτοια
        περιβάλλοντα.

        Ακόμη φάνηκε στην έρευνα για ακανόνιστο μοντέλο \cite{soo_measurement_2019} ότι ελάχιστα
        είναι γνωστά για τους σταλαγμίτες και τους σταλακτίτες και τις επιπτώσεις τους στη 
        συσσώρευση νερού στα σπήλαια.

        \subsection{ \textsf{Συγκριτική ανάλυση}}
        Οι παράμετροι με βάση τις οποίες θα συγκριθούν οι έρευνες είναι οι εξής:
            \begin{itemize}
                \item Pathloss
                \item Received Signal Power
                \item Data Extraction Rate (DER)
                \item Goodput
                \item Energy per Packet (E/P)
                \item Volumetric Water Content (VWC) of the soil
                \item Transmit Power
                \item Spreading Factor
                \item Coding Rate (the proportion of the useful
                parts of the data stream)
                \item Bandwidth
            \end{itemize}

        Ακόμη είναι σημαντικό στην μέτρηση της απόδοσης να επικρατεί κοινός
        σχεδιασμός δικτύου σε όλα τα πειράματα προκειμένου να αποφευχθεί προκατάληψη
        των μετρήσεων αν σε περίπτωση σύγκρουσης των πακέτων.Όπως στην έρευνα για την
        σκοπιμότητα των LoRaWAN σε υπόγεια περιβάλλοντα \cite{zhao_feasibility_2023}
        οι δέκτες όταν λαμβάνουν ισχύ (\ref{eq:1}) μεγαλύτερη από την ισχύ ευαισθησίας,
        αυτόματα αναγνωρίζουν το πακέτο ώς χαμένο καθώς έχει γίνει σύγκρουση. Ωστόσο έχει
        αναπτυχθεί αλγόριθμος για την αναγνώριση αν όντως έχει γίνει σύγκρουση.
        Οι παράμετροι φυσικού επιπέδου στους δέκτες (gateways) να είναι:
        \begin{itemize}
            \item Ισχύς Μετάδοσης: 14dBm
            \item Παράγοντας Εξάπλωσης: 12
            \item Coding Rate: 4/8 (50\%)
            \item Εύρος Ζώνης: 125kHz
        \end{itemize}

        Ως παράμετρο προσομοίωσης έχει οριστεί το Goodput το οποίο είναι ένα μέγεθος
        που περιγράφει την ταχύτητα των δεδομένων από ένα τερματικό στον λήπτη και 
        ορίζεται ώς: 
        \begin{equation}\label{eq:2}
            Goodput = \frac{N_{arrived} * PL}{T_{simulation}}
        \end{equation}\\
        Όπου PL είναι το μήκος του πακέτου.
        Ακόμη μια παράμετρος που χρησιμοποιήθηκε είναι το DER, δηλαδή τον ρυθμό με τον οποίο 
        Τέλος η παράμετρος με την οποία αξιολογήθηκε το μοντέλο είναι η ενέργεια ανά πακέτο. 

        Αντίστοιχα στην έρευνα για την διάδοση των UHF \cite{rak_uhf_2007} παρατηρήθηκαν
        διαφορετικές απώλειες σήματος σε καθένα από τα πέντε διαφορετικά μέρη σύμφωνα με το
        κάθε προφίλ. Πιο συγκεκριμένα στην πρώτη τοποθεσία ενός συστήματος ημικυκλικών στοών
        σε ξηρό ψαμμίτη (sandstone) με λεία τοιχώματα ο ειδικός ρυθμός εξασθένησης ήταν
        0,15 dB/m στα 446 MHz και 0,18 dB/m στα 860 MHz.
        Στην δεύτερη και τρίτη τοποθεσία με ορθογώνιες στοές από ασβεστόλιθο (limestone) με πιο
        τραχείς τοίχους και πήλινο δάπεδο, η δεύτερη τοποθεσία ήταν ξηρή, ενώ η τρίτη τοποθεσία
        ήταν υγρή με μικρές λίμνες νερού, ο ειδικός ρυθμός εξασθένησης κυμαινόταν από 0,14 έως
        0,22 dB/m στα 446 MHz και από 0,17 έως 0,28 dB/m στα 860 MHz. Στην τέταρτη τοποθεσία
        μιας στενή στοά με νερό καλυμμένο δάπεδο και υγρούς τοίχους λήφθηκε ο ειδικός ρυθμός
        εξασθένησης 0,16 dB/m στα 446 MHz και 0,19 dB/m στα 860 MHz. Τέλος στην πέμπτη τοποθεσία
        μια πολύ ακανόνιστη στοά με δάπεδο καλυμμένο με νερό και υγρούς τοίχους ο ειδικός ρυθμός
        εξασθένησης ήταν 0,25 dB/m στα 446 MHz και 0,29 dB/m στα 860 MHz.

        Πολλοί ερευνητές έχουν στραφεί σε κατευθυνόμενες μορφές τηλεμετρίας. Πιο συγκεκριμένα 
        στην διπλωματική έρευνα για κυψελωτά δίκτυα πέμπτης γενιάς του Philip Branch
        \cite{branch_fifth_2021} παρουσιάζεται μια επισκόπηση της χρήσης των δικτύων αυτών σε
        ορυχεία εξόρυξης βαρύτητας (Block Cave Mining), ακόμη παρατίθενται οι περιορισμοί του
        802.11 για την κάλυψη αναγκών στο εν λόγω περιβάλλον ενώ υπογραμμίζει τον επανασχεδιασμό
        ενός δικτύου ραδιοπρόσβασης 5G για τις απαιτήσεις.

        Μια από τις παραμέτρους που εξετάστηκαν σχεδόν σε κάθε έρευνα που παρουσιάζεται στην εργασία
        είναι η λαμβανόμενη ισχύς σήματος (RSSI). Πιο συγκεκριμένα στην έρευνα για το δίκτυο των
        πραγμάτων σε ορυχεία \cite{ming_study_2019} η RSSI χρησιμοποιείται ως μέτρηση για την 
        εκτίμηση της απόστασης μεταξύ ενός drone και ενός επίγειου σταθμού σε ένα σύστημα 
        UAV (Unmanned Aerial Vehicle). Αυτό επιτυγχάνεται με τριγωνικό εντοπισμό ενώ προτείνονται
        και μοντέλα παλινδρόμησης (regression) και νευρωνικά δίκτυα.

        Αρκετά ενδιαφέρουσα ήταν και η έρευνα του Donald G. Dudley για ασύρματη διάδοση σε κυκλικά
        τούνελ \cite{dudley_wireless_2005} κατά την οποία χρησιμοποιήθηκαν ηλεκτρικές και μαγνητικές
        πηγές έντασης προκειμένου να διεγερθούν τα πεδία και να παραμετροποιηθούν ανάλογα \cdot
        συχνότητα, πλάτος, πόλωση. Η μελέτη στοχεύει στην κατανόηση της διάδοσης, της σκέδασης και
        της απορρόφησης ηλεκτρομαγνητικών κυμάτων σε διαφορετικά περιβάλλοντα, τα οποία μπορεί να 
        έχουν επιπτώσεις σε πεδία όπως η ασύρματη επικοινωνία, τα συστήματα ραντάρ και η 
        ηλεκτρομαγνητική συμβατότητα. Πιο συγκεκριμένα τα ΤΕ \& ΤΜ διεγείρονται ξεχωριστά ενώ
        λαμβάνονται υπόψη οι: συχνότητα, ακτίνα, αγωγιμότητα.

        Τα ηλ/κά κύματα προφανώς ανακλώνται στις επιφάνειες τών τοιχωμάτων ενός σπηλαίου και
        είναι σημαντικό να μετρηθεί η διάδοση όταν στο μονοπάτι υπάρχουν τραχείς επιφάνειες.
        Έτσι στην έρευνα \cite{soo_investigation_2018} χρησιμοποιήθηκε μια τεχνική ανίχνευσης 
        ακτίνων και εξέτασαν την ανάκλαση με τρεις τεχνικές:
        \begin{itemize}
            \item Συμβατικός συντελεστής ανάκλασης Fresnel
            \item Τροποποιημένος συντελεστής ανάκλασης Fresnel με συντελεστή εξασθένησης
            \item Προσομοίωση τυχαίας τραχιάς επιφάνειας
            \item Ανάλυση ισχύος σήματος έναντι θέσεων δέκτη
        \end{itemize}

        Στην έρευνα για ασύρματο σύστημα επικοινωνίας \cite{yavuz_-cave_2009} η μέθοδος που
        προτείνεται είναι βασισμένη σε ασύρματο δίκτυο αισθητήρων πολυμέσων (WMSN). Πιο
        συγκεκριμένα υπάρχουν κινητοί κόμβοι με παρόμοια λειτουργικότητα όπως τα κινητά τηλέφωνα,
        που μεταφέρονται από μέλη της ομάδας, ένας διακομιστής καταλόγου ICWCS που αποθηκεύει
        πληροφορίες για την πραγματοποίηση επικοινωνίας και παρακολούθησης σε πραγματικό χρόνο. Ακόμη
        στο δίκτυο υπάρχει μηχανισμός προσωρινής αποθήκευσης για να Επιταχύνει την πραγματοποίηση
        μελλοντικών κλήσεων σε προσωρινά αποθηκευμένους κινητούς κόμβους.
        
        Πιο πειραματικά σε ένα σπήλαιο που μετατράπηκε σε κελάρι \cite{soo_propagation_2018}
        έγιναν πειράματα μέτρησης πεδίου σε συχνότητες 900 MHz, 2.4 GHz και 5.8 GHz. Καθώς
        τα 900MHz είναι η μικρότερη από της τρεις συχνότητες, και άρα λιγότερο ευάλωτη σε 
        απώλειες διαδρομής έγινε η σύγκριση συν- και σταυροπολώσεων (π.χ. VV \& VH) για 900 MHz.
        
        Οι μέθοδοι που εφαρμόστηκαν στην \cite{pingenot_full_2005} είναι οι εξής: υψηλής τάξης
        διακριτοποίηση πεπερασμένων στοιχείων, προσομοίωση πεδίου χρόνου, άλμα-βάτραχου και
        κατά συνέπεια αλγόριθμοι FFT για μετατροπή στον τομέα της συχνότητας.

        Προκειμενου να επιτευχθεί η μέτρηση μοντέλου ανώμαλου εδάφους για πιθανές εφαρμογές
        σε σπήλαια, πραγματοποιήθηκε μέτρηση πεδίου στο μοντέλο ανώμαλου εδάφους στα 2,4 GHz
        έγινε σύγκριση του μετρούμενου αποτελέσματος με δύο αποτελέσματα ανίχνευσης ακτίνων
        καθώς επίσης και ενσωμάτωση παράγοντα σκέδασης στην προσομοίωση ιχνηλάτησης ακτίνων
        για βελτιωμένη εφαρμογή
\documentclass[12pt]{article}
\usepackage{geometry}
\geometry{ b4paper, total={220mm,320mm}, left=20mm, top=15mm }

%\usepackage[parfill]{parskip}  % Activate to begin paragraphs with an empty line rather
%than an indent

\usepackage{mathtools}
\usepackage{blindtext}
\usepackage{multicol}
\usepackage{listings}
\usepackage{tikz}
\usepackage{booktabs, caption, colortbl}
\usepackage{pdflscape}

\usepackage{fancyhdr, lastpage, setspace}

\usepackage{pgfplots}
\pgfplotsset{compat=1.18}

\usepackage{hyperref}
\hypersetup{
    bookmarksopen=true,
    colorlinks=true,
    linkcolor=black,
    filecolor=magenta,      
    urlcolor=black,
    pdftitle={Cave Communications}
}

\usepackage{graphicx}
\usepackage{amssymb}
\usepackage{enumitem}
\usepackage{amsmath}

\usepackage{polyglossia}
\setdefaultlanguage{greek} 
\setotherlanguages{english}

\usepackage[acronym]{glossaries}
\usepackage{fontspec}
\usepackage{microtype}

\usepackage[ backend=biber, bibencoding=utf8, style=ieee]{biblatex}
\addbibresource{ref.bib}


\listfiles

%\newfontfamily\greekfont[Script=Greek]{Georgia}
%\newfontfamily\greekfontsans[Script=Greek]{FreeSans}

% \setmainlanguage{greek} \setotherlanguage{english}
% \setTransitionsForGreek{\selectlanguage{greek}}{\selectlanguage{english}}

% \defaultfontfeatures{Mapping=tex-text}


%\setmainfont[Kerning=On,Mapping=tex-text]{Linux Libertine}
\setmainfont[Numbers=Lining]{Georgia}
\setsansfont[Script=Greek]{FreeSans}
\setmonofont{Corbel}

%\DeclareTextFontCommand{\maintxt}{\greekfont}
%\DeclareTextFontCommand{\titletxt}{\greekfontsans}

\renewcommand{\thesection}{\Roman{section}}
\renewcommand{\thesubsection}{\thesection.\Roman{subsection}}

\newcommand{\ra}[1]{\renewcommand{\arraystretch}{#1}}

\onehalfspacing

\AddToHook{env/landscape/begin}
 {%
  \clearpage
  \pagestyle{empty}
  \AddToHook{shipout/background}[sven/page]
    {
     \put(0.9\paperwidth,-0.5\paperheight)%adapt values
      {\rotatebox{90}{\thepage}}
    }%
 }     

\AddToHook{env/landscape/after}
 {\RemoveFromHook{shipout/background}[sven/page]}


\lhead{Μάρκος Δελαπόρτας} \rhead{ece01316@uowm.gr} \cfoot{Page \thepage\ of
\pageref{lastpage}}

\title{ \textsf{ Cave Communications}\\
    \textsf{Δίκτυα Νέας Γενιάς \& Επικοινωνίες}\\
    \textsf{\Large Τμήμα Ηλεκτρολόγων Μηχανικών \& Μηχανικών Υπολογιστών}\\
    \textsf{\large Πανεπιστήμιο Δυτικής Μακεδονίας}
} \author{\textsf{Μάρκος Δελαπόρτας} \footnote{E-mail: ece01316@uowm.gr}}
\date{\textsf{Ιανουάριος 2024}}

\begin{document}
\maketitle
\tableofcontents

\begin{multicols*}{2}
    \textbf{ \textit{Abstract} -- Τα ασύρματα δίκτυα επικοινωνίας αισθητήρων
        έχουν γίνει πανταχού παρόντα, τόσο στην καθημερινή ζωή, όσο και σε πολλές
        βιομηχανίες. Ωστόσο η αποτελεσματικότητά τους μπορεί να παρεμποδιστεί σε ορισμένα
        περιβάλλοντα όπως για παράδειγμα στην εξερεύνηση σπηλαίων, στην διάσωση ανθρώπων
        από κατεστραμμένα κτήρια ή στον τομέα της γεωργίας ακριβείας όπου τα σπαρτά
        βρίσκονται κάτω από την επιφάνεια του εδάφους. Με άλλα λόγια υπάρχει ανάγκη για
        ασύρματη επικοινωνία σε υπόγειο περιβάλλον. Στο παρόν άρθρο εξετάζονται διάφορες
        τεχνικές ασύρματης επικοινωνίας, οι περιορισμοί και οι στερήσεις που έχουν γίνει
        προκειμένου να μοντελοποιηθεί σε περιβάλλον σπηλαίων από διάφορες τεχνολογίες που
        έχουν προταθεί. Καθώς επίσης και πως με αλλαγές στην ισχύ και την συχνότητα μπορεί
        να μειωθεί η απώλεια μονοπατιού και η ισχύς του σήματος λήψης. }
    

    \section{\textsf{Εισαγωγή}}
    % what has not been done in the sector, like there are technologies in lace but not
    % for a specific matter discussed. \textit{\textbf{Ιστορικό}}: 
        Από την εμφάνιση των πρώτων κινητών επικοινωνιών, πολλοί εξερευνητές, όπως και
        πολλοί διασώστες επιχειρούν να τις χρησιμοποιούν προς όφελός τους· πράγμα το οποίο
        αποδεικνύεται δύσκολο λόγω της πρόκλησης που παρουσιάζεται κατά την διάδοση
        ηλεκτρομαγνητικών κυμάτων σε υπόγεια περιβάλλοντα και γενικότερα σε περιβάλλοντα
        με περίπλοκη μορφολογία. Αξίζει ακόμη να σημειωθεί οτι ενώ δεν υπάρχει πολύ
        βιβλιογραφία για συνήθεις σπήλαια και ερείπια, έχει γίνει αρκετή έρευνα για
        ορυχεία ανθρακα, παρότι έχουν διαφορετικά χαρακτηριστικά.
    
    % \maintxt{
        Η τρέχουσα γενιά ασύρματων δικτύων, όπως το 4G και το 5G, έχουν κατασκευαστεί
        βασισμένα σε αρχές υψηλού εύρους ζώνης, χαμηλής καθυστέρησης και υψηλής
        διαθεσιμότητας για να ανταποκριθούν στις απαιτήσεις μεγάλου αριθμού πελατών. Αυτά
        τα δίκτυα έχουν σχεδιαστεί για να υποστηρίζουν μεταφορά δεδομένων υψηλής
        ταχύτητας, επικοινωνία χαμηλής καθυστέρησης και εφαρμογές σε πραγματικό χρόνο.
    % }
    % \maintxt{ %\textit{\textbf{Στόχοι}}: 
        Ωστόσο, οι στόχοι των υπόγειων ασύρματων δικτύων επικοινωνίας είναι διαφορετικοί
        από εκείνους των υπαρχόντων ασύρματων δικτύων. Λόγω του μικρού αριθμού πελατών
        (κινητών τερματικών), του μικρού εύρους ζώνης και της υψηλής ανοχής καθυστέρησης,
        αυτά τα δίκτυα θα πρέπει να σχεδιάζονται με διαφορετικές αρχές κατά νου. Σε
        υπόγεια περιβάλλοντα, τα ασύρματα δίκτυα πρέπει να είναι αξιόπιστα, αποτελεσματικά
        και ανεκτικά σε υψηλή εξασθένηση και απώλεια σήματος.
    % }
    % \maintxt{ %\textit{\textbf{Σκοπιμότητα}}: 
        Αυτή η εργασία παρουσιάζει μια μελέτη σκοπιμότητας για τη χρήση υπόγειων ασύρματων
        δικτύων επικοινωνίας σε περιβάλλοντα σπηλαίων. Η μελέτη περιλαμβάνει την ανάλυση
        των υπαρχόντων τεχνολογιών που έχουν προταθεί και υλοποιηθεί, τα χαρακτηριστικά
        διάδοσης του σήματος σε υπόγεια περιβάλλοντα καθώς και το σχεδιασμό πρωτοκόλλων
        επικοινωνίας για την επίτευξη αξιόπιστης και αποτελεσματικής επικοινωνίας.
    % }

    \section{\textsf{Το πρόβλημα}} Το μεγαλύτερο πρόβλημα στις υπόγειες
        επικοινωνίες εγγυάται στο υψηλό pathloss λόγο της πυκνότητας του μέσου διαδοσης. Με
        άλλα λόγια το μεγαλύτερο μέρος της έρευνας που έχει λάβει τόπο στις ασύρματες
        επικοινωνίες λαμβάνει ώς μέσο διάδοσης τον κενό χώρος (δηλ. τον αέρα), αυτό διότι
        υπάρχει προβλεψιμότητα στην μοντελοποίηση και επειδή επιτρέπει διάδοση σε
        υψηλότερες συχνότητες μεγαλύτερο εύρος ζώνης και άλλα.\\
        Όταν ωστόσο όταν το αντικείμενο της μελέτης διαφέρει από την μεγιστοποίηση της
        διαδιδόμενης πληροφορίας αλλά εγγυάται στην αξιοπιστία και στην μεγιστοποιηση της
        θωράκισης σε μέσα με υψηλή απώλεια σήματος, προκύπτει ανάγκη για έρευνα σημάτων σε
        διαφορετικές συχνότητες και κωδικοποιήσεις.\\
        Ο σκοπός της παρούσας μελέτης είναι η -- σε περιβάλλον σπηλαίου , συντρίμμια
        φυσικές καταστροφές και επικοινωνία με το εξωτερικό περιβάλλον ή/και με προσωπικό
        διάσωσης.

        Η προτεινόμενη λύση με κινητούς αναμεταδότες που προτείνεται είναι πολύ χρήσιμη
        για επιχειρήσεις διάσωσης σε περιπτώσεις φυσικών καταστροφών ή/και επικοινωνίας σε
        υπόγειο βάθος.

    \section{\textsf{Σχετικές Έρευνες}} Στην βιβλιογραφία έχουν συνταχθεί
        αρκετές εργασίες σχετικά με την ασύρματη επικοινωνία σε σπήλαια και υπόγεια
        περιβάλλοντα. Ωστόσο πρέπει να σημειωθεί οτι δεδομένου του εύρους όλων των 
        σεναρίων/περιπτώσεων που ενέχει η έρευνα υπόγειων επικοινωνιών στην βιβλιογραφία
        μπορεί να βρεθεί μια πληθώρα από μοντέλα που διαφέρουν σχετικά με τα προβλήματα που
        καλούνται να επιλύσουν. Με άλλα λόγια ενώ παρατίθενται διάφορες έρευνες για το 
        συγκεκριμένο ζήτημα κρίνεται απαραίτητο να ομαδοποιηθούν με βάση ορισμένες παραμέτρους
        ώς προς το πρόβλημα και τα φυσικά χαρακτηριστικά του συστήματος.
        
        Στην έρευνα των Guozheng Zhao, Kaiqiang Lin, Tong Hao \cite{zhao_feasibility_2023}
        χρησιμοποιήθηκαν τερματικά που ξεκινούν να δέχονται κατερχόμενη σύνδεση αμέσως μετά 
        από μετάδοση και για ένα ορισμένο χρονικό διάστημα σε ένα LoRa-WAN δίκτυο.
        Για την μοντελοποίηση του καναλιού Above Ground to Underground (AG2UG) \& 
        Underground to Above Ground (UG2AG) λήφθηκε υπ'όψη και εξασθένηση λόγω πλευρικών
        κυμάτων, έτσι η λαμβανόμενη ισχύς υπολογίζεται από τον τύπο:\\
        \begin{multline} \label{eq:1}
            P_r = P_t + G_t + G_r + \\ 
                [L_{ug}(d_{ug}) + aL_{ag}(d_{ag}) + bL_{surface}(d_{surface}) + L_R - 10\log\chi^2]
        \end{multline}
        
        Και ανάλογα με τις φυσικές παραμέτρους της τοπολογίας (π.χ. αποστάσεις μεταξύ
        τερματικών, αποστάσεις μεταξύ κόμβων και τερματικών, κ.α.) επιλέγονται οι 
        σταθερές εξασθένησης a \& b, συγκεκριμένα εξαρτώνται από την διηλεκτρική 
        σταθερά του εδάφους.

        Αντίστοιχα στην έρευνα για την διάδοση UHF σε υπόγεια σπήλαια \cite{rak_uhf_2007}
        προτάθηκε μια πειραματική μελέτη κατά την οποία η τηλεμετρία εκμεταλλεύεται την φυσική
        μορφολογία του υπόγειου περιβάλλοντος. Πιο συγκεκριμένα οι ερευνητές χρησιμοποίησαν
        κινητούς πομποδέκτες και κεραίες για να εκτελέσουν τα πειράματα τους, σε συχνότητες 
        446 και 860 MHz. Σε πέντε τοποθεσίες αναζήτησαν το μέγιστο επίπεδο σήματος σε όλο το
        προφίλ της γκαλερί σε κάθε σημείο μέτρησης.

        Οι συγγραφείς έβγαλαν ένα εμπειρικό γραμμικό μοντέλο απώλειας διαδρομής ως συνάρτηση
        της απόστασης και το συνέκριναν με ένα θεωρητικό μοντέλο κυματοδηγού που βασίζεται στις
        γεωμετρικές και ηλεκτρικές παραμέτρους του περιβάλλοντος. Βρήκαν ότι το μοντέλο κυματοδηγού
        ήταν κατάλληλο μόνο για κανονικές στοές και όχι για ακανόνιστα προφίλ που εμφανίζονται
        συχνά σε σπηλιές.
        
        Στην έρευνα για διερεύνηση τραχιών επιφανειών για διάδοση μοντελοποίηση σε σπηλιές
        \cite{soo_investigation_2018} επιδιώκει την παροχή ασύρματων επικοινωνιών σε σπήλαια
        για διάφορες δραστηριότητες, ενίσχυση της ασφάλειας σε σπηλιές για τουριστικό σκοπό
        και βοήθεια σε επιχειρήσεις έρευνας και διάσωσης. Με άλλα λόγια είναι ένα θεμέλιο για
        περαιτέρω έρευνα σε πραγματικά περιβάλλοντα σπηλαίων. Ωστόσο ο περιορισμός κατά την
        παρούσα ανάλυση είναι ότι ο συντελεστής εξασθένησης υπολογίζει εσφαλμένα την απώλεια
        σκέδασης για ορισμένες μετρήσεις και έτσι χρειάζεται να εισαχθεί ο μέσος συντελεστής
        ανάκλασης εισάγεται για να ληφθούν υπόψη οι τραχιές επιφάνειες.

        % ~~~~~~~~~~~~~~~~~~~~~~~~~~~~~~~~~~~~~~~~~~~~~~~~~~~~~~~~~~~~~~~~~~~~~~~~~~~~~~~~~~        
        Ακόμη ο Muhammed Enes Bayrakdar έδειξε έναν τρόπο για
        συσκευές του δικτύου των πραγμάτων να επικοινωνούν σε ένα δίκτυο πλέγματος
        \cite{bayrakdar_rule_2019}. Οι Mark Hedley και Ian Gipps ανέδειξαν έναν τρόπο για
        ακριβή προσδιορισμό θέσης σε υπόγεια ορυχεία \cite{hedley_accurate_2013}. Οι
        Manoja D. Weiss και Kevin Moore ανέδειξαν έναν τρόπο για αυτόνομη κινητή
        τηλεπικοινωνία και ασύρματη πρόσδεση σε τούνελ \cite{weiss_autonomous_2009}. Οι
        William Walsh και Jay Gao ανέλυσαν την δυνατότητα χρήσης wifi σε περιβάλλον
        σπηλαίου \cite{walsh_communications_2018}. Ο Philip Branch συνέταξε την
        διπλωματική του για κυψελωδη δίκτυα πέμπτης γενιάς σε υπόγεια ορυχεία βαρύτιτας
        \cite{branch_fifth_2021}. Οι M.I. Martínez-Garrido και R. Fort ανέπτυξαν υλισμικό
        για πειράματα σε ανθρωπογενής και φυσική κληρονομιά
        \cite{martinez-garrido_experimental_2016}.
        % ~~~~~~~~~~~~~~~~~~~~~~~~~~~~~~~~~~~~~~~~~~~~~~~~~~~~~~~~~~~~~~~~~~~~~~~~~~~~~~~~~~

        Ο βασικός σκοπός του In-Cave Wireless
        Sensor Networks (ICWCS) είναι να παρέχει ένα αξιόπιστο κανάλι φωνητικής επικοινωνίας
        μέσω ενός δικτύου πολυμέσων μεταξύ των μελών μιας ομάδας σε μια σπηλιά. Αποτελείται
        από δύο διαφορετικούς τύπους κόμβων. 
        \begin{itemize}
            \item «κόμβους κορμού» που είναι τοποθετημένο στα τοιχώματα του σπηλαίου.
            \item «Κινητοί κόμβοι», οι οποίοι έχουν παρόμοια λειτουργικότητα όπως τα κινητά
            τηλέφωνα, φέρονται από τα μέλη της ομάδας.
        \end{itemize}

        Οι κόμβοι κορμού είναι ακίνητοι
        κόμβοι και ο σχεδιασμός τους βασίζεται στον ασύρματο αισθητήρα γενικής
        χρήσης «VF1A» \cite{walsh_communications_2018}.
        Αυτοί οι κόμβοι είναι υπεύθυνοι για διατήρηση της συνδεσιμότητας κορμού μέσα 
        στο σπήλαιο, τις δραστηριότητες
        εγγραφής και περιαγωγής κινητών κόμβων, δρομολόγηση πληροφοριών κλήσεων μεταξύ
        κινητών κόμβων και μεταφορά των δεδομένων ελέγχου και φωνής μέσω του ασύρματου
        δικτύου.

        Άλλες πρακτικές εφαρμογές είναι η δημιουργία αξιόπιστου δικτύου φωνητικής επικοινωνίας
        σε προκλητικά περιβάλλοντα ραδιοσυχνοτήτων όπως στην έρευνα των A. Gokhan Yavuz, 
        Z. Cihan Taysi και Esra Celik \cite{yavuz_-cave_2009}. Επιπλέον, σχεδιάστηκαν
        ασύρματοι κόμβοι κορμού και κινητά τερματικά πολυμέσων (π.χ κείμενο, εικόνας) 
        αφιερώνοντας παραπάνω χρόνο σε αμφίδρομες φωνητικές επικοινωνίες με περαιτέρω
        δυνατότητες (multicast \& broadcast).

        Οι πρακτικές εφαρμογές της ανάλυσης πλήρους κύματος RF της εξασθένησης σήματος σε σπήλαιο
        με απώλειες χρησιμοποιώντας μέθοδο διακριτών στοιχείων υψηλής τάξης διανυσματικού τομέα
        χρόνου \cite{pingenot_full_2005} είναι η ενίσχυση της ασύρματης επικοινωνίας σε σπηλιές
        και σήραγγες και μειώνει τους κινδύνους για το στρατιωτικό και το προσωπικό διάσωσης σε τέτοια
        περιβάλλοντα.

        Ακόμη φάνηκε στην έρευνα για ακανόνιστο μοντέλο \cite{soo_measurement_2019} ότι ελάχιστα
        είναι γνωστά για τους σταλαγμίτες και τους σταλακτίτες και τις επιπτώσεις τους στη 
        συσσώρευση νερού στα σπήλαια.
\end{multicols*}
\begin{landscape}
    \begin{table}
        \caption{Σύγκριση τεχνολογιών που παρουσιάζονται στην βιβλιογραφία}
        \label{table:comparison}
        \centering
        \begin{tabular}{cp{0.1\linewidth}p{0.15\linewidth}p{0.20\linewidth}p{0.1\linewidth}p{0.3\linewidth}}
        \toprule
        Τεχνολογία & Ερευνητές & Σκοπός Έρευνας & Μέθοδοι & Τύπος Έρευνας & Κύρια Πορίσματα - Συμπεράσματα \\
        \midrule
        LoRA WAN & Guozheng Zhao, Kaiqiang Lin, Tong Hao \cite{zhao_feasibility_2023} 
        & Ανάπτυξη προσομοίωσης για να αξιολογηθεί η σκοπιμότητα της υπόγειας παρακολούθησης χρησιμοποιώντας
        υπόγειο ασύρματο δύκτιο αισθητήρων (WUSN) που βασίζεται στο LoRaWAN & 
        Ο προσομοιωτής έκανε είκοσι προσομοιώσεις, η καθεμία με ποικίλες παραμέτρους δικτύου όπως
        η ανάπτυξη ακριανών συσκευών και η μετάδοση δεδομένων. Διάφορες μετρήσεις όπως η καλή απόδοση,
        το DER (Λόγος σφάλματος καθυστέρησης) και το EPP (Αποτελεσματικότητα ελέγχου ισχύος) 
        υπολογίστηκαν χρησιμοποιώντας αλγόριθμους φιλτραρίσματος μέσου όρου. 
        Επιπρόσθετα, ο προσομοιωτής εντόπισε πιθανές συγκρούσεις μεταξύ των WUSN.
        & Πείραμα μέσω προσομοίωσης 
        & Μέσω πειραμάτων προσομοίωσης, αυτοί οι παράγοντες διερευνήθηκαν ποσοτικά και τα αποτελέσματα
        έδειξαν ότι τα WUSN που βασίζονται στο LoRaWAN επέτρεψαν βαθύτερη και ευρύτερη υπόγεια
        παρακολούθηση με καλή απόδοση δικτύου σε σύγκριση με τα παραδοσιακά WUSN.\\
        UHF & Milan Rak and Pavel Pechac \cite{rak_uhf_2007} 
        & Διερεύνηση της διάδοσης ραδιοκυμάτων σε σπήλαια για συχνότητες, 446 MHz και 860 MHz. &
        Τα συνεχή κύματα δημιουργήθηκαν από φορητούς πομπούς εξοπλισμένους με οριζόντια πανκατευθυντικές 
        μονοπολικές κεραίες που πολώθηκαν κατακόρυφα. Η κεραία λήψης μετακινήθηκε κατά μήκος του
        προφίλ της συλλογής αναζητώντας το μέγιστο επίπεδο σήματος, εξαλείφοντας τα τοπικά ελάχιστα
        που προκαλούνται από τη διάδοση πολλαπλών διαδρομών. & Πειράματα και Μαθηματική Ανάλυση &
        Η μετρηθείσα ειδική εξασθένηση στις πρώτες τέσσερις θέσεις ήταν σε καλή συμφωνία με τις 
        θεωρητικές προβλέψεις αλλά στην τοποθεσία Ε, τα αποτελέσματα διέφεραν σημαντικά. 
        Αυτή η απόκλιση αποδόθηκε σε λανθασμένη εκτίμηση των περιβαλλοντικών παραμέτρων.
        Παρατηρήθηκε συνδυασμός πολλαπλών τρόπων λειτουργίας. Αυτό αποδόθηκε στην αυξημένη απόσταση,
        η οποία προκάλεσε σημαντική διασπορά. Η σκέδαση είχε ως αποτέλεσμα την εκπόλωση των κυμάτων
        και παρόμοιες απώλειες διαδρομής και για τις δύο ορθογώνιες πολώσεις. Η μελέτη 
        διαπίστωσε ότι το διάσπαρτο πεδίο συνέβαλε σημαντικά στη διάδοση του σήματος στις στροφές.\\

        \bottomrule
        \end{tabular}
    \end{table}      
\end{landscape}
\begin{multicols*}{2}
    \subsection{ \textsf{Συγκριτική ανάλυση}}
        Οι παράμετροι με βάση τις οποίες θα συγκριθούν οι έρευνες είναι οι εξής:
            \begin{itemize}
                \item Pathloss
                \item Received Signal Power
                \item Data Extraction Rate (DER)
                \item Goodput
                \item Energy per Packet (E/P)
                \item Volumetric Water Content (VWC) of the soil
                \item Transmit Power
                \item Spreading Factor
                \item Coding Rate (the proportion of the useful
                parts of the data stream)
                \item Bandwidth
            \end{itemize}

        Ακόμη είναι σημαντικό στην μέτρηση της απόδοσης να επικρατεί κοινός
        σχεδιασμός δικτύου σε όλα τα πειράματα προκειμένου να αποφευχθεί προκατάληψη
        των μετρήσεων αν σε περίπτωση σύγκρουσης των πακέτων.Όπως στην έρευνα για την
        σκοπιμότητα των LoRaWAN σε υπόγεια περιβάλλοντα \cite{zhao_feasibility_2023}
        οι δέκτες όταν λαμβάνουν ισχύ (\ref{eq:1}) μεγαλύτερη από την ισχύ ευαισθησίας,
        αυτόματα αναγνωρίζουν το πακέτο ώς χαμένο καθώς έχει γίνει σύγκρουση. Ωστόσο έχει
        αναπτυχθεί αλγόριθμος για την αναγνώριση αν όντως έχει γίνει σύγκρουση.
        Οι παράμετροι φυσικού επιπέδου στους δέκτες (gateways) να είναι:
        \begin{itemize}
            \item Ισχύς Μετάδοσης: 14dBm
            \item Παράγοντας Εξάπλωσης: 12
            \item Coding Rate: 4/8 (50\%)
            \item Εύρος Ζώνης: 125kHz
        \end{itemize}

        Ως παράμετρο προσομοίωσης έχει οριστεί το Goodput το οποίο είναι ένα μέγεθος
        που περιγράφει την ταχύτητα των δεδομένων από ένα τερματικό στον λήπτη και 
        ορίζεται ώς: 
        \begin{equation}\label{eq:2}
            Goodput = \frac{N_{arrived} * PL}{T_{simulation}}
        \end{equation}\\
        Όπου PL είναι το μήκος του πακέτου.
        Ακόμη μια παράμετρος που χρησιμοποιήθηκε είναι το DER, δηλαδή τον ρυθμό με τον οποίο 
        Τέλος η παράμετρος με την οποία αξιολογήθηκε το μοντέλο είναι η ενέργεια ανά πακέτο. 

        Αντίστοιχα στην έρευνα για την διάδοση των UHF \cite{rak_uhf_2007} παρατηρήθηκαν
        διαφορετικές απώλειες σήματος σε καθένα από τα πέντε διαφορετικά μέρη σύμφωνα με το
        κάθε προφίλ. Πιο συγκεκριμένα στην πρώτη τοποθεσία ενός συστήματος ημικυκλικών στοών
        σε ξηρό ψαμμίτη (sandstone) με λεία τοιχώματα ο ειδικός ρυθμός εξασθένησης ήταν
        0,15 dB/m στα 446 MHz και 0,18 dB/m στα 860 MHz.
        Στην δεύτερη και τρίτη τοποθεσία με ορθογώνιες στοές από ασβεστόλιθο (limestone) με πιο
        τραχείς τοίχους και πήλινο δάπεδο, η δεύτερη τοποθεσία ήταν ξηρή, ενώ η τρίτη τοποθεσία
        ήταν υγρή με μικρές λίμνες νερού, ο ειδικός ρυθμός εξασθένησης κυμαινόταν από 0,14 έως
        0,22 dB/m στα 446 MHz και από 0,17 έως 0,28 dB/m στα 860 MHz. Στην τέταρτη τοποθεσία
        μιας στενή στοά με νερό καλυμμένο δάπεδο και υγρούς τοίχους λήφθηκε ο ειδικός ρυθμός
        εξασθένησης 0,16 dB/m στα 446 MHz και 0,19 dB/m στα 860 MHz. Τέλος στην πέμπτη τοποθεσία
        μια πολύ ακανόνιστη στοά με δάπεδο καλυμμένο με νερό και υγρούς τοίχους ο ειδικός ρυθμός
        εξασθένησης ήταν 0,25 dB/m στα 446 MHz και 0,29 dB/m στα 860 MHz.

        Πολλοί ερευνητές έχουν στραφεί σε κατευθυνόμενες μορφές τηλεμετρίας. Πιο συγκεκριμένα 
        στην διπλωματική έρευνα για κυψελωτά δίκτυα πέμπτης γενιάς του Philip Branch
        \cite{branch_fifth_2021} παρουσιάζεται μια επισκόπηση της χρήσης των δικτύων αυτών σε
        ορυχεία εξόρυξης βαρύτητας (Block Cave Mining), ακόμη παρατίθενται οι περιορισμοί του
        802.11 για την κάλυψη αναγκών στο εν λόγω περιβάλλον ενώ υπογραμμίζει τον επανασχεδιασμό
        ενός δικτύου ραδιοπρόσβασης 5G για τις απαιτήσεις.

        Μια από τις παραμέτρους που εξετάστηκαν σχεδόν σε κάθε έρευνα που παρουσιάζεται στην εργασία
        είναι η λαμβανόμενη ισχύς σήματος (RSSI). Πιο συγκεκριμένα στην έρευνα για το δίκτυο των
        πραγμάτων σε ορυχεία \cite{ming_study_2019} η RSSI χρησιμοποιείται ως μέτρηση για την 
        εκτίμηση της απόστασης μεταξύ ενός drone και ενός επίγειου σταθμού σε ένα σύστημα 
        UAV (Unmanned Aerial Vehicle). Αυτό επιτυγχάνεται με τριγωνικό εντοπισμό ενώ προτείνονται
        και μοντέλα παλινδρόμησης (regression) και νευρωνικά δίκτυα.

        Αρκετά ενδιαφέρουσα ήταν και η έρευνα του Donald G. Dudley για ασύρματη διάδοση σε κυκλικά
        τούνελ \cite{dudley_wireless_2005} κατά την οποία χρησιμοποιήθηκαν ηλεκτρικές και μαγνητικές
        πηγές έντασης προκειμένου να διεγερθούν τα πεδία και να παραμετροποιηθούν ανάλογα \cdot
        συχνότητα, πλάτος, πόλωση. Η μελέτη στοχεύει στην κατανόηση της διάδοσης, της σκέδασης και
        της απορρόφησης ηλεκτρομαγνητικών κυμάτων σε διαφορετικά περιβάλλοντα, τα οποία μπορεί να 
        έχουν επιπτώσεις σε πεδία όπως η ασύρματη επικοινωνία, τα συστήματα ραντάρ και η 
        ηλεκτρομαγνητική συμβατότητα. Πιο συγκεκριμένα τα ΤΕ \& ΤΜ διεγείρονται ξεχωριστά ενώ
        λαμβάνονται υπόψη οι: συχνότητα, ακτίνα, αγωγιμότητα.

        Τα ηλ/κά κύματα προφανώς ανακλώνται στις επιφάνειες τών τοιχωμάτων ενός σπηλαίου και
        είναι σημαντικό να μετρηθεί η διάδοση όταν στο μονοπάτι υπάρχουν τραχείς επιφάνειες.
        Έτσι στην έρευνα \cite{soo_investigation_2018} χρησιμοποιήθηκε μια τεχνική ανίχνευσης 
        ακτίνων και εξέτασαν την ανάκλαση με τρεις τεχνικές:
        \begin{itemize}
            \item Συμβατικός συντελεστής ανάκλασης Fresnel
            \item Τροποποιημένος συντελεστής ανάκλασης Fresnel με συντελεστή εξασθένησης
            \item Προσομοίωση τυχαίας τραχιάς επιφάνειας
            \item Ανάλυση ισχύος σήματος έναντι θέσεων δέκτη
        \end{itemize}

        Στην έρευνα για ασύρματο σύστημα επικοινωνίας \cite{yavuz_-cave_2009} η μέθοδος που
        προτείνεται είναι βασισμένη σε ασύρματο δίκτυο αισθητήρων πολυμέσων (WMSN). Πιο
        συγκεκριμένα υπάρχουν κινητοί κόμβοι με παρόμοια λειτουργικότητα όπως τα κινητά τηλέφωνα,
        που μεταφέρονται από μέλη της ομάδας, ένας διακομιστής καταλόγου ICWCS που αποθηκεύει
        πληροφορίες για την πραγματοποίηση επικοινωνίας και παρακολούθησης σε πραγματικό χρόνο. Ακόμη
        στο δίκτυο υπάρχει μηχανισμός προσωρινής αποθήκευσης για να Επιταχύνει την πραγματοποίηση
        μελλοντικών κλήσεων σε προσωρινά αποθηκευμένους κινητούς κόμβους.
        
        Πιο πειραματικά σε ένα σπήλαιο που μετατράπηκε σε κελάρι \cite{soo_propagation_2018}
        έγιναν πειράματα μέτρησης πεδίου σε συχνότητες 900 MHz, 2.4 GHz και 5.8 GHz. Καθώς
        τα 900MHz είναι η μικρότερη από της τρεις συχνότητες, και άρα λιγότερο ευάλωτη σε 
        απώλειες διαδρομής έγινε η σύγκριση συν- και σταυροπολώσεων (π.χ. VV \& VH) για 900 MHz.
        
        Οι μέθοδοι που εφαρμόστηκαν στην \cite{pingenot_full_2005} είναι οι εξής: υψηλής τάξης
        διακριτοποίηση πεπερασμένων στοιχείων, προσομοίωση πεδίου χρόνου, άλμα-βάτραχου και
        κατά συνέπεια αλγόριθμοι FFT για μετατροπή στον τομέα της συχνότητας.

        Προκειμενου να επιτευχθεί η μέτρηση μοντέλου ανώμαλου εδάφους για πιθανές εφαρμογές
        σε σπήλαια, πραγματοποιήθηκε μέτρηση πεδίου στο μοντέλο ανώμαλου εδάφους στα 2,4 GHz
        έγινε σύγκριση του μετρούμενου αποτελέσματος με δύο αποτελέσματα ανίχνευσης ακτίνων
        καθώς επίσης και ενσωμάτωση παράγοντα σκέδασης στην προσομοίωση ιχνηλάτησης ακτίνων
        για βελτιωμένη εφαρμογή


    \section{\textsf{Συμπεράσματα}}
        Συμπερασματικά, η εργασία καταδεικνύει τη σκοπιμότητα των
        υπόγειων ασύρματων δικτύων επικοινωνίας σε υπόγεια περιβάλλοντα. Αυτά τα δίκτυα
        μπορούν να υποστηρίξουν μικρό αριθμό πελατών, με μικρά εύρη ζώνης και υψηλή ανοχή
        καθυστέρησης, και μπορούν να σχεδιαστούν ώστε να είναι αξιόπιστα και αποτελεσματικά σε
        τέτοια απαιτητικά περιβάλλοντα. Η μελέτη παρέχει ένα σημείο εκκίνησης για μελλοντική
        έρευνα σε αυτόν τον τομέα.

        Από την έρευνα για την σκοπιμότητα των LoRaWAN σε υπόγεια περιβάλλοντα 
        \cite{zhao_feasibility_2023} οι συντάκτες παραθέτουν τα αποτελέσματα τις προσομοίωσης
        και τις διαφορές που έχουν στο σύστημα οι μεταβλητές περιβάλλοντος. Πιο συγκεκριμένα
        φαίνεται ότι το LoRaWAN είναι πιο συμπαγές σε σχέση με άλλα υπόγεια ασύρματα συστήματα
        αισθητήρων (WUSNs). Πιο συγκεκριμένα φάνηκε ότι παρά ενδεχόμενες κακές συνθήκες 
        περιβάλλοντος (π.χ. Volumetric Water Content = 50\%) το σύστημα λειτουργούσε αξιόπιστα.

        Στην έρευνα για τα UHF \cite{rak_uhf_2007} οι συγγραφείς παρουσίασαν βασικές οδηγίες
        για την εκτίμηση της ειδικής εξασθένησης και της μέγιστης εμβέλειας για ασύρματες
        επικοινωνίες σε υπόγειες γκαλερί διαφόρων προφίλ και τύπων τοίχων. Πρότειναν επίσης 
        ότι τα αποτελέσματά τους θα μπορούσαν να χρησιμοποιηθούν ως πειραματική βάση για 
        περαιτέρω θεωρητικές εργασίες.

        Στα αποτελέσματα της διπλωματικής εργασίας για τα δίκτυα πέμπτης γενιάς \cite{branch_fifth_2021}
        φαίνεται η υπεροχή των 5G σε σύγκριση με το wifi αλλά και πως μπορεί να συμπληρώσει
        την παραδοσιακή VHF ραδιοφωνία σε Peer to Peer (P2P) επιτρέποντας την μετάδοση βίντεο.
        Τέλος αναφέρει ότι το περιβάλλον διάδοσης των ορυχείων είναι περίπλοκο με κάποιες ενδείξεις
        οτι η διάδοση είναι λιγότερο δριμεία οταν το μονοπάτι περιλαμβάνει άξονες εξαγωγής με
        επένδυση από χάλυβα.

        Η έρευνα που μοντελοποίησε δίκτυα IoT σε ορυχεία \cite{ming_study_2019} ανέδειξε οτι 
        προκειμένου να υλοποιηθεί αποδοτικά η επικοινωνία με μέσο διάδοσης την πέτρα, δε μπορεί
        να υπερβαίνει τα έξι μέτρα. Ακόμη φάνηκε οτι ένα νευρωνικό δίκτυο προς τα πίσω διάδοσης
        επιστρέφει πιο ακριβή αποτελέσματα από απλό τριγωνικό εντοπισμό.

        Όπως φάνηκε από την έρευνα για ασύρματη διάδοση σε κυκλικό τούνελ \cite{dudley_wireless_2005}
        13 λειτουργίες είναι πάνω από την συχνότητα αποκοπής για τέλεια αγώγιμα τοιχώματα. Ακόμη
        χρησιμοποιούνται 16 λειτουργίες για τη διασφάλιση της σύγκλισης στην περίπτωση τοίχου 
        με απώλειες.

        Τα αποτελέσματα της προσομοίωσης ανίχνευσης ακτίνων \cite{soo_investigation_2018} δείχνουν
        τη διαφορά στην ανάκλαση του εδάφους μεταξύ λείων και τραχιών επιφανειών. Αλλά επίσης και
        πως ο τροποποιημένος, με συντελεστή εξασθένησης, συντελεστής ανάκλασης Fresnel χρησιμοποιείται
        για τραχιές επιφάνειες και πώς η ισχύς του σήματος επηρεάζεται από την απόσταση και είναι 
        χαμηλότερη σε ανώμαλες συνθήκες επιφάνειας.

        Όπως και σε άλλες έρευνες οι κινητοί κόμβοι έχουν παρόμοια λειτουργία με τα κινητά τερματικά
        ενός κυψελωτού δικτύου. Πιο συγκεκριμένα στην έρευνα για ασύρματο δίκτυο επικοινωνίας σε
        σπήλαια \cite{yavuz_-cave_2009} υλοποιείται ένα αξιόπιστο δίκτυο φωνητικής επικοινωνίας
        χρησιμοποιώντας ασύρματους κόμβους αισθητήρων πολυμέσων σε εσωτερικά περιβάλλοντα με δύο
        τύπους ασύρματων κόμβων αισθητήρων: κόμβοι κορμού και κινητοί κόμβοι. Η τρέχουσα υλοποίηση
        χειρίζεται μια ενεργή φωνητική επικοινωνία από σημείο σε σημείο.

        Πιο πειραματικά σε ένα σπήλαιο που μετατράπηκε σε κελάρι \cite{soo_propagation_2018}
        παρατηρήθηκε ότι τα σήματα χαμηλότερης συχνότητας είναι πιο πρακτικά για ασύρματη
        επικοινωνία σε σπηλιές. Πιο συγκεκριμένα η κάθετη συν-πόλωση (VV) έχει την καλύτερη
        λαμβανόμενη ισχύ στα περισσότερα σενάρια.
        Είναι σημαντικό να γίνει ο διαχωρισμός μεταξύ σπηλαίων τουριστικού σκοπού και των σπηλαίων
        εξερευνητικού/ορειβατικού σκοπού. Αρχικά επειδή οι τουριστικές σπηλιές έχουν κίνηση ανθρώπων
        και μεγαλύτερες διαστάσεις σε σύγκριση με τις άγριες σπηλιές \cite{soo_propagation_2018}.
        Ακόμη η επιφάνεια του εδάφους των τουριστικών σπηλαίων είναι πιο λεία ενώ αντίθετα τα φυσικά
        περάσματα των σπηλαίων έχουν τραχιές επιφάνειες και ανομοιομορφίες διαστάσεων.

        Τα αποτελέσματα μέτρησης πεδίου που λαμβάνονται από το εσωτερικό του κελαριού του
        Jeff \cite{soo_propagation_2018} σε τρεις συχνότητες είναι τα εξής: η ισχύς του σήματος
        μειώνεται καθώς ο δέκτης απομακρύνεται από τον πομπό, το σήμα 900 MHz μπορεί να διαδοθεί
        ισχυρότερα για μεγαλύτερη απόσταση ενώ το σήμα 5,8 GHz εξασθενεί πιο γρήγορα και φτάνει στο 
        επίπεδο θορύβου.

        Στην πλήρη ανάλυση ραδιοσυχνοτήτων \cite{pingenot_full_2005} συλλέχθηκαν στατιστικά στοιχεία
        για τις ιδιότητες διάδοσης και εξασθένησης του περιβάλλοντος του σπηλαίου. Ακόμη η φασματική
        πυκνότητα ισχύος και η φάση των συντελεστών του διανυσματικού ηλεκτρικού πεδίου. Δεν βρέθηκε
        κάποια σημαντική διαφοροποίηση στα αποτελέσματα σε όλο το φάσμα. Για το πρωτεύον διαδιδόμενο
        πεδίο ($E_z$) το διάσπαρτο πεδίο συμπληρώνει μηδενικά (nulls) που στην ομαλή περίπτωση 
        δημιουργούνται από καταστροφικές παρεμβολές. Τέλος η διασπορά πόλωσης αυξάνει την ενέργεια
        των $E_x$ και $E_y$ περαιτέρω μέσα στο σπήλαιο, πράγμα αναμενόμενο.

        Η μέτρηση μοντέλου ανώμαλου εδάφους για πιθανές εφαρμογές σε σπήλαια \cite{soo_measurement_2019}
        ανέδειξε ότι η ισχύς του σήματος πέφτει καθώς ο δέκτης απομακρύνεται από τον πομπό και πιο
        συγκεκριμένα η δραστική πτώση του σήματος σε τοποθεσία μεταξύ 18 και 22 m μετά το σήμα LOS
        που στρίβει σε γωνία. Ακόμη η διαδρομή σήματος NLOS μπορεί να ταξιδέψει μόνο μέσω ανάκλασης,
        περίθλασης και διάθλασης. Τα προσομοιωμένα και μετρημένα αποτελέσματα δείχνουν καλύτερη συμφωνία
        με τον παράγοντα σκέδασης, δηλαδή βελτιωμένη τυπική απόκλιση από 6,84 dB σε 4,48 dB με συντελεστή
        σκέδασης.

    \section{\textsf{Συζήτηση}}
        Σε θεωρητικό επίπεδο εχουν προταθεί τεχνολογίες extremely low frequency
        (elf) οι οποίες είναι ήδη σε χρήση σε υποβρύχια επειδή τα ραδιοκύματα ELF μπορούν να
        διεισδύσουν στο θαλασσινό νερό σε πολύ μεγαλύτερο βάθος από τα ραδιοκύματα υψηλότερης
        συχνότητας, τα οποία απορροφώνται από το νερό. Τα κύματα ELF (3 - 30Hz) έχουν πολύ
        μεγάλα μήκη κύματος, που κυμαίνονται από εκατοντάδες έως χιλιάδες χιλιόμετρα, και
        παράγονται από μια μεγάλη κεραία που ονομάζεται «δίπολο εδάφους» ή «δίπολο γης». Οι
        κεραίες που χρησιμοποιούνται για την επικοινωνία ELF έχουν συνήθως μήκος πολλών
        χιλιομέτρων και συχνά βρίσκονται σε απομακρυσμένες περιοχές με χαμηλά επίπεδα
        ανθρωπογενών παρεμβολών.

        Ωστόσο απ'οτι φάνηκε στην έρευνα για LoRaWAN \cite{zhao_feasibility_2023}, μεγάλες 
        αποστάσεις μπορούν να καλυφθούν αλλά συγκεκριμένες περιπτώσεις όπως η γεωργία ακριβείας.
        Αυτό διότι το LoRaWAN είναι μια τεχνολογία που ελαχιστοποιεί την κατανάλωση ενέργειας και
        μεγιστοποιεί στην εμβέλεια, με αντάλλαγμα τον μειωμένο ρυθμό δεδομένων, δηλαδή τον χαμηλό
        ρυθμό διάδοσης (ενδεικτικά κάθε 30 λεπτά σε βάθος 40 εκατοστών με VWC=40\%). Με άλλα
        λόγια το LoRaWAN είναι μια πολύ αξιόπιστη επιλογή, που ωστόσο υστερεί σε περιπτώσεις 
        χρήσης που χρειάζεται γρήγορη επικοινωνία σε υπόγειο περιβάλλον βαθύτερο των τριών μέτρων.

        Αξίζει επίσης να σημειωθεί οτι τα δίκτυα πέμπτης γενιάς προσφέρουν υψηλή ρυθμοαπόδοση,
        χαμηλή καθυστέρηση και μεγάλο εύρος ζώνης - πράγμα χρήσιμο για περιβάλλοντα ορυχείων.
        Ωστόσο κρίνεται σημαντικό να γίνει η διαφοροποιηση των ορυχείων με των συνήθων σπηλαίων,
        ερειπίων καθώς τα ορυχεία έχουν λιγότερο ακανόνιστη μορφολογία και έτσι μεγαλύτερο βαθμό
        ελευθερίας και εκμετάλλευσης των χαρακτηριστικών από την τηλεμετρία. Αντίστοιχα στην έρευνα
        για την ανάκλαση ακτίνων σε τραχείς επιφάνειες \cite{soo_investigation_2018} δημιουργήθηκε
        ένα τυχαίο προφίλ τραχιάς επιφάνειας που μοιάζει με φυσικές συνθήκες σπηλαίων, εξετάστηκε
        η επίδραση της περιεκτικότητας σε νερό στο βράχο στη διάδοση ηλεκτρομαγνητικών κυμάτων και
        τέλος εισήλθε ο μέσος συντελεστής ανάκλασης για να ληφθούν υπόψη τα φαινόμενα σκέδασης.

        Όπως φάνηκε από την ανάλυση των IoT σε ορυχεία \cite{ming_study_2019} η λαμβανόμενη ισχύς
        σήματος μπορεί να χρησιμοποιηθεί για την παρακολούθηση της προόδου σε ένα ορυχείο, ακόμη 
        η καμπύλη προσαρμογής προέκυψε πειραματικά για την υλοποιηση του νευρωνικού δικτύου.

        Αντίθετα από την έρευνα για ασύρματες επικοινωνίες σε σπήλαια \cite{yavuz_-cave_2009},
        φάνηκε οτι λόγω των περιοριστικών συνθηκών ραδιοσυχνοτήτων εντός των σπηλαίων καθιστούν
        ανεπαρκείς τις ασύρματες λύσεις. Επίσης το ICWCS χειρίζεται επί του παρόντος μόνο μία
        ενεργή φωνητική επικοινωνία από σημείο σε σημείο. Επιπρόσθετα η υπηρεσία καταλόγου
        εκτελείται συνήθως έξω από το σπήλαιο, αλλά οι κόμβοι κορμού χρησιμοποιούν μηχανισμό
        προσωρινής αποθήκευσης για να επιταχύνουν τις λειτουργίες καταλόγου. Τέλος πρέπει να
        εκτελεστούν αρκετά βήματα, συμπεριλαμβανομένης της ανάπτυξης και της αρχικής εγκατάστασης,
        προτού τεθεί σε λειτουργία το ICWCS. Όλα τα παραπάνω φανερώνουν κάποιους από τους
        περιορισμούς των ασύρματων δικτύων επικοινωνίας σε υπόγεια περιβάλλοντα.

        Οι σταλαγμίτες και οι σταλακτίτες στα περάσματα των σπηλαίων χρειάζονται περαιτέρω
        εξερεύνηση όπως φάνηκε στην \cite{soo_propagation_2018}.
        Τα ασύρματα σήματα στα σπήλαια είναι ευεργετικά για την επιστημονική έρευνα και τη 
        διαχείριση του τουρισμού

        Στην πλήρη ανάλυση ραδιοσυχνοτήτων \cite{pingenot_full_2005} εγινε υπολογιστική μελέτη
        διάδοσης και εξασθένησης σήματος σε περιβάλλον σπηλαίων με απώλειες. Οι εξισώσεις Maxwell
        πλήρους κύματος επιλύθηκαν απευθείας στο πεδίο του χρόνου. Στατιστικά δεδομένα που παράγονται
        για τη φασματική πυκνότητα ισχύος και τη φάση του ηλεκτρικού πεδίου συλλέχθηκαν.

        Στην μέτρηση μοντέλου ανώμαλου εδάφους για πιθανές εφαρμογές σε σπήλαια \cite{soo_measurement_2019}
        κατασκευάστηκε ένα μοντέλο ανώμαλου εδάφους για πιθανές εφαρμογές σε σπήλαια. Η μέτρηση πεδίου
        πραγματοποιήθηκε στα 2,4 GHz και φάνηκε ότι προσομοίωση ιχνηλάτησης ακτίνων με συντελεστή 
        σκέδασης ταιριάζει καλύτερα με το μετρημένο αποτέλεσμα.

    \subsection{}
        Σε πειραματικό επίπεδο έχουν υλοποιηθεί και δοκιμαστεί ορισμένα πρότυπα επικοινωνίας
        τα οποία βασίζονται σε κάποιες από τις αρχές που έχουν προταθεί παραπάνω. Πιο
        συγκεκριμένα το HeyPhone είναι ένα φορητό ραδιόφωνο που χρησιμοποιεί μια κεραία βρόχου
        με διάμετρο ένα μέτρο και έχει την ικανότητα να διαπεράσει το έδαφος σε βάθος μέχρι
        και 500 περίπου μέτρα, χρησιμοποιεί διαμόρφωση 87KHz Single Side Band (SSB).

    \subsection{}
        Ακόμη μια τεχνολογία που χρησιμοποιείται ευρέως στην βιομηχανία είναι το τηλέφωνο
        ενός καλωδίου (Single Wire Telephone). Οι συσκευές είναι πολύ απλές, χρησιμοποιώντας
        ένα μόνο op-amp τόσο για να στείλουν όσο και να λάβουν ένα ηχητικό σήμα (φωνή) κατά
        μήκος ενός μόνο μονωμένου καλωδίου, χρησιμοποιώντας τη χωρητικότητα σύζευξης χειριστών
        στη γείωση για την επιστροφή. Ο δέκτης έχει πολύ υψηλή αντίσταση εισόδου. Η ζήτηση
        ισχύος είναι πολύ χαμηλή, με πολλές ώρες (ή ημέρες) λειτουργίας από μπαταρία 9 volt.
        Απαιτείται πολύ μικρή σύζευξη στο έδαφος στο τέλος της λήψης. Το άκρο μετάδοσης οδηγεί
        κυρίως στην κατανεμημένη χωρητικότητα του καλωδίου στη γείωση. Τα πολύ μεγάλα καλώδια
        (χιλιόμετρα) θα απαιτήσουν αρκετά καλή σύζευξη στη γείωση στο άκρο μετάδοσης.
        Προφανώς, οι αμφίδρομες επικοινωνίες πάνω από ένα πολύ μακρύ σύρμα θα απαιτήσουν
        αρκετά καλούς λόγους και στα δύο άκρα, όπως ένα μικρό πείρο σε βρωμιά ή ένα γυμνό
        σύρμα στο νερό.

    \printbibliography
\end{multicols*}
\end{document}

\section{\textsf{Συζήτηση}}
    Σε θεωρητικό επίπεδο εχουν προταθεί τεχνολογίες extremely low frequency
    (elf) οι οποίες είναι ήδη σε χρήση σε υποβρύχια επειδή τα ραδιοκύματα ELF μπορούν να
    διεισδύσουν στο θαλασσινό νερό σε πολύ μεγαλύτερο βάθος από τα ραδιοκύματα υψηλότερης
    συχνότητας, τα οποία απορροφώνται από το νερό. Τα κύματα ELF (3 - 30Hz) έχουν πολύ
    μεγάλα μήκη κύματος, που κυμαίνονται από εκατοντάδες έως χιλιάδες χιλιόμετρα, και
    παράγονται από μια μεγάλη κεραία που ονομάζεται «δίπολο εδάφους» ή «δίπολο γης». Οι
    κεραίες που χρησιμοποιούνται για την επικοινωνία ELF έχουν συνήθως μήκος πολλών
    χιλιομέτρων και συχνά βρίσκονται σε απομακρυσμένες περιοχές με χαμηλά επίπεδα
    ανθρωπογενών παρεμβολών.


    Σε πειραματικό επίπεδο έχουν υλοποιηθεί και δοκιμαστεί ορισμένα πρότυπα επικοινωνίας
    τα οποία βασίζονται σε κάποιες από τις αρχές που έχουν προταθεί παραπάνω. Πιο
    συγκεκριμένα το HeyPhone είναι ένα φορητό ραδιόφωνο που χρησιμοποιεί μια κεραία βρόχου
    με διάμετρο ένα μέτρο και έχει την ικανότητα να διαπεράσει το έδαφος σε βάθος μέχρι
    και 500 περίπου μέτρα, χρησιμοποιεί διαμόρφωση 87KHz Single Side Band (SSB).

    Ακόμη μια τεχνολογία που χρησιμοποιείται ευρέως στην βιομηχανία είναι το τηλέφωνο
    ενός καλωδίου (Single Wire Telephone). Οι συσκευές είναι πολύ απλές, χρησιμοποιώντας
    ένα μόνο op-amp τόσο για να στείλουν όσο και να λάβουν ένα ηχητικό σήμα (φωνή) κατά
    μήκος ενός μόνο μονωμένου καλωδίου, χρησιμοποιώντας τη χωρητικότητα σύζευξης χειριστών
    στη γείωση για την επιστροφή. Ο δέκτης έχει πολύ υψηλή αντίσταση εισόδου. Η ζήτηση
    ισχύος είναι πολύ χαμηλή, με πολλές ώρες (ή ημέρες) λειτουργίας από μπαταρία 9 volt.
    Απαιτείται πολύ μικρή σύζευξη στο έδαφος στο τέλος της λήψης. Το άκρο μετάδοσης οδηγεί
    κυρίως στην κατανεμημένη χωρητικότητα του καλωδίου στη γείωση. Τα πολύ μεγάλα καλώδια
    (χιλιόμετρα) θα απαιτήσουν αρκετά καλή σύζευξη στη γείωση στο άκρο μετάδοσης.
    Προφανώς, οι αμφίδρομες επικοινωνίες πάνω από ένα πολύ μακρύ σύρμα θα απαιτήσουν
    αρκετά καλούς λόγους και στα δύο άκρα, όπως ένα μικρό πείρο σε βρωμιά ή ένα γυμνό
    σύρμα στο νερό.
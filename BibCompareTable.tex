\begin{landscape}
    \begin{longtable}{cp{0.1\linewidth}p{0.15\linewidth}p{0.20\linewidth}p{0.1\linewidth}p{0.3\linewidth}}
        \caption{Σύγκριση τεχνολογιών που παρουσιάζονται στην βιβλιογραφία}
        \label{table:comparison}\\
        
        Τεχνολογία & Ερευνητές & Σκοπός Έρευνας & Μέθοδοι & Τύπος Έρευνας & Κύρια Πορίσματα - Συμπεράσματα \\
        \hline

        LoRA WAN & Guozheng Zhao, Kaiqiang Lin, Tong Hao \cite{zhao_feasibility_2023} 
        & Ανάπτυξη προσομοίωσης για να αξιολογηθεί η σκοπιμότητα της υπόγειας παρακολούθησης χρησιμοποιώντας
        υπόγειο ασύρματο δύκτιο αισθητήρων (WUSN) που βασίζεται στο LoRaWAN & 
        Ο προσομοιωτής έκανε 20 προσομοιώσεις με διαφορετικές ρυθμίσεις δικτύου και υπολόγισε πολλές
        μετρήσεις για κάθε προσομοίωση, όπως: Goodput (bps), αναλογία σφάλματος καθυστέρησης (DER)
        και απόδοση ελέγχου ισχύος (EPP). Ο προσομοιωτής εντόπισε επίσης πιθανές συγκρούσεις μεταξύ
        συσκευών στο ασύρματο δίκτυο αισθητήρων χρησιμοποιώντας τέσσερις διαφορετικές λειτουργίες:
        σύγκρουση συχνότητας, Sf(sps), ισχύος, χρονική σύγκρουση
        & Πείραμα μέσω προσομοίωσης 
        & Μέσω πειραμάτων προσομοίωσης, αυτοί οι παράγοντες διερευνήθηκαν ποσοτικά και τα αποτελέσματα
        έδειξαν ότι τα WUSN που βασίζονται στο LoRaWAN επέτρεψαν βαθύτερη και ευρύτερη υπόγεια
        παρακολούθηση με καλή απόδοση δικτύου σε σύγκριση με τα παραδοσιακά WUSN.\\
        \hline
        UHF & Milan Rak and Pavel Pechac \cite{rak_uhf_2007} 
        & Διερεύνηση της διάδοσης ραδιοκυμάτων σε σπήλαια για συχνότητες, 446 MHz και 860 MHz. &
        Τα συνεχή κύματα δημιουργήθηκαν από φορητούς πομπούς εξοπλισμένους με οριζόντια πανκατευθυντικές 
        μονοπολικές κεραίες που πολώθηκαν κατακόρυφα. Η κεραία λήψης μετακινήθηκε κατά μήκος του
        προφίλ της συλλογής αναζητώντας το μέγιστο επίπεδο σήματος, εξαλείφοντας τα τοπικά ελάχιστα
        που προκαλούνται από τη διάδοση πολλαπλών διαδρομών. & Πειράματα και Μαθηματική Ανάλυση &
        Η μετρηθείσα ειδική εξασθένηση στις πρώτες τέσσερις θέσεις ήταν σε καλή συμφωνία με τις 
        θεωρητικές προβλέψεις αλλά στην τοποθεσία Ε, τα αποτελέσματα διέφεραν σημαντικά. 
        Αυτή η απόκλιση αποδόθηκε σε λανθασμένη εκτίμηση των περιβαλλοντικών παραμέτρων.
        Παρατηρήθηκε συνδυασμός πολλαπλών τρόπων λειτουργίας. Αυτό αποδόθηκε στην αυξημένη απόσταση,
        η οποία προκάλεσε σημαντική διασπορά. Η σκέδαση είχε ως αποτέλεσμα την εκπόλωση των κυμάτων
        και παρόμοιες απώλειες διαδρομής και για τις δύο ορθογώνιες πολώσεις. Η μελέτη 
        διαπίστωσε ότι το διάσπαρτο πεδίο συνέβαλε σημαντικά στη διάδοση του σήματος στις στροφές.\\
        
        ZigBee (802.15.4g) & M.I. Martínez-Garrido, R. Fort \cite{martinez-garrido_experimental_2016} &
        Αξιολόγηση μιας πλατφόρμας ασύρματης επικοινωνίας για χρήση σε διάφορα περιβάλλοντα, 
        συμπεριλαμβανομένων εσωτερικών και εξωτερικών που σχεδιάστηκε για να ανιχνεύει αυτόματα νέες
        κινήσεις και να ενσωματώνει τα αντίστοιχα δεδομένα.&
        Οι παράμετροι που μετρώνται μπορούν να χωριστούν σε δύο
        κύριες κατηγορίες: αφενός, τις μετρήσεις που λαμβάνονται από τους ίδιους τους αισθητήρες 
        (σε αυτή τη μελέτη θερμοκρασία και σχετική υγρασία), και αφετέρου την ποιότητα των επικοινωνιών
        και την απαίτηση ενέργειας.& Πειράματα &
        Ποσοτικοποίησε την επίδραση των εμποδίων στη διαδρομή μεταξύ της κίνησης και του σταθμού βάσης, 
        την ενέργεια που καταναλώνεται από τα τερματικά, μετρούμενη ως προς την ισχύ της μπαταρίας. 
        Ο πιο επιζήμιος παράγοντας βρέθηκε ότι ήταν η πλήρης απόφραξη των κεραιών εκπομπής από 
        παρακείμενα μεταλλικά υλικά.\\
        \hline
        WiFi & Pablo F. Miaja, Fermin Navarro-Medina, Daniel G. Aller, Germán León,
        Alejandro Camanzo, Carlos Manuel Suarez, Francisco G. Alonso, Diego Nodar,
        Francesco Sauro, Massimo Bandecchi, Loredana Bessone, Fernando Aguado-Agelet, Manuel Arias
        & Χρήση ενός πρωτοκόλλου IEEE 802.11 για τη δημιουργία ραδιοζεύξης μεταξύ των μείων πρόσβασης (CH)
        και κινητών τερματικών (CE).
        & Οποιοδήποτε από τα κινητά τερματικά (CE) μπορεί να λειτουργήσει ως «κινητά» σημεία πρόσβασης 
        που διευκολύνουν την επικοινωνία απευθείας οπτικής επαφής (LOS). &
        Mηχανική συστημάτων που βασίζονται σε μοντέλα &
        Το κύριο πλεονέκτημα του 802.11ah έγκειται στη χαμηλότερη συχνότητα που χρησιμοποιείται,
        κάτω από 1 GHz, που ικανοποιεί απαιτήσεις μεγάλης εμβέλειας ή ισοδύναμα μειωμένες ανάγκες
        ισχύος μετάδοσης, καθώς και βελτιωμένη διείσδυση εμποδίων, σε βάρος της διαθέσιμης απόδοσης.\\
        % \bottomrule
    \end{longtable}
\end{landscape}

%!TEX TS-program = xelatex
%!TEX encoding = UTF-8 Unicode

\documentclass[12pt]{article}
\usepackage{geometry}           % See geometry.pdf to learn the layout options. There are lots.
\geometry{letterpaper}          % ... or a4paper or a5paper or ... 
%\geometry{landscape}           % Activate for for rotated page geometry
%\usepackage[parfill]{parskip}  % Activate to begin paragraphs with an empty line rather than an indent

\usepackage{blindtext}
\usepackage{multicol}

\usepackage{graphicx}
\usepackage{amssymb}
\usepackage{enumitem}
\usepackage{amsmath}

\usepackage{fontspec,xltxtra,xunicode}
\usepackage[Latin,Greek]{ucharclasses}
%\usepackage{microtype}

\usepackage{polyglossia}
\setdefaultlanguage{greek}
\setotherlanguages{english}

\setmainlanguage{greek}
\setotherlanguage{english}
\setTransitionsForGreek{\selectlanguage{greek}}{\selectlanguage{english}}

\defaultfontfeatures{Mapping=tex-text}

\setmainfont[Kerning=On,Mapping=tex-text]{EB Garamond}
\setmainfont{Literata}
\setsansfont[Ligatures=TeX]{Linux Libertine}
\setmonofont[Scale=MatchLowercase]{Ubuntu}

\newfontfamily\greekfont[Script=Greek]{Ubuntu}
\newfontfamily\greekfontsf[Script=Greek]{Literata}

\renewcommand{\thesection}{\Roman{section}} 
\renewcommand{\thesubsection}{\thesection.\Roman{subsection}}

\title{\texttt{Cave Communications}}
\author{\texttt{Μάρκος Δελαπόρτας 1316}}
\date{}

\begin{document}
\maketitle

%\texttt{loρεμ} %? Lato
%\textsf{λοrem} %? Serif
%\newpage

\begin{multicols}{2}
    \scriptsize { %\textbf
        Abstract--Τα ασύρματα δίκτυα επικοι-νωνίας αισθητήρων έχουν γίνει
        πανταχού παρόντα, τόσο στην καθημερινή ζωή, όσο και σε πολλές βιομηχανίες. 
        Ωστόσο η αποτελεσματικότητά τους μπορεί να παρεμποδιστεί σε
        ορισμένα περιβάλλοντα όπως για παράδειγμα στην εξερεύνηση σπηλαίων, 
        στην διάσωση ανθρώπων από κατεστραμέννα κτήρια ή στον τομέα της γεωργίας ακριβείας 
        όπου τα σπάρτα βρίσκονται κάτω από την επιφάνεια του εδάφους. 
        Με άλλα λόγια υπάρχει ανάγκη για ασύρματη επικοινωνία σε υπόγειο περιβάλλον. 
        Στο παρόν άρθρο εξετάζονται διάφορες τεχνικές ασύρματης επικοινωνίας, 
        οι περιορισμοί και οι στερήσεις που έχουν γίνει προκειμένου να μοντελοποιηθεί σε περιβάλλον σπηλαίων από διάφορες τεχνολογίες που έχουν προταθεί. 
        Καθώς επίσης και πως με αλλαγές στην ισχύ και την συχνότητα μπορεί
        να μειωθεί η απώλεια
        μονοπατιού και η ισχύς του σήματος λήψης.
    }
    
    \section{\textit{\normalsize Εισαγωγή}}:
    % todo: note what has not been done in the sector, like there are technoligies in lace but not for a specific matter discussed.
    %\textit{\textbf{Ιστορικό}}: 
    Since the appearence of the first wireless mobile communications,
    many adventurers as well as rescue personel has been trying to utilize them into their advantage. This is difficult as for the environments that challenge the propagation of electromagnetic waves within them\dots It is also worth noting that while not very much reaserch has been conducted for regular caves there is much ground work for coal mines, even though they have different charachteristics.
    
    
    Η τρέχουσα γενιά ασύρματων δικτύων, όπως το 4G και το 5G, 
    έχουν κατασκευαστεί βασισμένα σε αρχές υψηλού εύρους ζώνης, 
    χαμηλής καθυστέρησης και υψηλής διαθεσιμότητας για να ανταποκριθούν
    στις απαιτήσεις μεγάλου αριθμού πελατών. 
    Αυτά τα δίκτυα έχουν σχεδιαστεί για να υποστηρίζουν μεταφορά δεδομένων υψηλής ταχύτητας, 
    επικοινωνία χαμηλής καθυστέρησης και εφαρμογές σε πραγματικό χρόνο.
    
    %\textit{\textbf{Στόχοι}}: 
    Ωστόσο, οι στόχοι των υπόγειων ασύρματων δικτύων επικοινωνίας είναι
    διαφορετικοί από εκείνους των υπαρχόντων ασύρματων δικτύων.
    Λόγω του μικρού αριθμού πελατών, του μικρού εύρους ζώνης και της
    υψηλής ανοχής καθυστέρησης, 
    αυτά τα δίκτυα θα πρέπει να σχεδιάζονται με διαφορετικές αρχές κατά νου. 
    Σε υπόγεια περιβάλλοντα, τα ασύρματα δίκτυα πρέπει να είναι αξιόπιστα, αποτελεσματικά και ανεκτικά σε υψηλή εξασθένηση και απώλεια σήματος.
    
    %\textit{\textbf{Σκοπιμότητα}}: 
    Αυτή η εργασία παρουσιάζει μια μελέτη σκοπιμότητας για τη χρήση υπόγειων ασύρματων δικτύων επικοινωνίας σε περιβάλλοντα σπηλαίων. Η μελέτη περιλαμβάνει την ανάλυση των υπαρχόντων τεχνολογιών που έχουν προταθεί και υλοποιηθεί, τα χαρακτηριστικά διάδοσης του σήματος σε υπόγεια περιβάλλοντα καθώς και το σχεδιασμό πρωτοκόλλων επικοινωνίας για την επίτευξη αξιόπιστης και αποτελεσματικής επικοινωνίας.
    
    \section{\textit{\normalsize Βασικοί Τρόποι Προσέγγισης}}
    Σε αυτό το κομμάτι παρουσιάζονται διάφοροι τρόποι προσέγγισης του προβλήματος. Με άλλα λόγια τι τοπολογίες μπορούν να χρισημοποιηθούν για εποικοινωνίες σε σπήλαια και άλλα περιβάλλοντα που παρουσιάζεται υψηλή εξασθένηση.
    Για αυτό δύο τεχνικές μπορούν να εφαρμοστούν:
    \begin{itemize}
        \item peer to peer
        \item mesh
    \end{itemize}

    Παρακάτω παρατείθενται διάφορες τοπολογίες και τεχνικές με συνδιασμό ενός εκ των παραπάνω τρόπων προσσέγισης και αστερισμού προκειμένουν να επιτευχθεί επικοινωνία μεταξύ κινητών συσκευών μέσα σε περιβάλλον σπηλαίου αλλά και μεταξύ ενός σταθμού βάσης σε εξωτερικό χώρο.

    \section{\textit{\normalsize Βασική Τοπολογία}}
    Ο βασικός σκοπός του In-Cave Wireless Sensor Networks (ICWCS) είναι να παρέχει ένα αξιόπιστο κανάλι φωνητικής επικοινωνίας μέσω ενός δικτύου πολυμέσων μεταξύ των μελών
    μιας ομάδας σε μια σπηλιά. Αποτελείται από δύο διαφορετικούς τύπους κόμβων. 
    \begin{itemize}
        \item «κόμβους κορμού» που είναι τοποθετημένο στα τοιχώματα του σπηλαίου.
        \item «Κινητοί κόμβοι», οι οποίοι έχουν παρόμοια λειτουργικότητα όπως τα κινητά τηλέφωνα, φέρονται από τα μέλη της ομάδας.
    \end{itemize}

    \subsection{\textit{\small Κόμβοι ραχοκοκαλιάς}}
        Οι κόμβοι κορμού είναι ακίνητοι κόμβοι και ο σχεδιασμός τους
        βασίζεται στον προηγούμενο ασύρματο αισθητήρα γενικής χρήσης
        κόμβος «VF1A» [4]. Έχουν μια μονάδα επεξεργαστή, ένα RF
        μονάδα πομποδέκτη, μονάδα μνήμης και τροφοδοσία
        ενότητα διαχείρισης. Αυτοί οι κόμβοι είναι υπεύθυνοι για
        διατήρηση της συνδεσιμότητας κορμού μέσα στο σπήλαιο, το
        δραστηριότητες εγγραφής και περιαγωγής κινητών κόμβων,
        δρομολόγηση πληροφοριών κλήσεων μεταξύ κινητών κόμβων και
        μεταφορά των δεδομένων ελέγχου και φωνής μέσω του ασύρματου
        δίκτυο.
    \subsection{\textit{\small Κινητοί κόμβοι}}
        Οι κινητοί κόμβοι μεταφέρονται από σπηλαιολόγους και περιλαμβάνουν
        χαρακτηριστικά πέρα από τα τυπικά ενός ασύρματου αισθητήρα
        κόμβος για την αλληλεπίδραση με τους σπηλαιολόγους. Αυτά τα χαρακτηριστικά
        περιλαμβάνει μια γραφική οθόνη, ένα μικρόφωνο, ένα ακουστικό
        και ένα πληκτρολόγιο ελέγχου.
        \includegraphics*[scale=.4]{img/ICWMSN-WMSNpng.png}

    \section{\textit{\normalsize Τεχνολογίες}}
    Παρακάτω παρουσιάζονται διάφορες τεχνικές διαμόρφωσης προκειμένου να μπορεί να διαδοθεί σήμα μέσα σε σκληρά υλικά ή το ενδεχόμενο εκμετάλευσης των ρογμών (π.χ. μεταξύ βράχων) αυτών.

    \subsection{} Σε θεωριτικό επίπεδο εχουν προταθεί τεχνολογίες extrimely low frequency (elf) οι οποίες είναι ήδη σε χρήση σε υποβρύχια επειδή τα ραδιοκύματα ELF μπορούν να διεισδύσουν στο θαλασσινό νερό σε πολύ μεγαλύτερο βάθος από τα ραδιοκύματα υψηλότερης συχνότητας, τα οποία απορροφώνται από το νερό. Τα κύματα ELF (3 - 30Hz) έχουν πολύ μεγάλα μήκη κύματος, που κυμαίνονται από εκατοντάδες έως χιλιάδες χιλιόμετρα, και παράγονται από μια μεγάλη κεραία που ονομάζεται «δίπολο εδάφους» ή «δίπολο γης». Οι κεραίες που χρησιμοποιούνται για την επικοινωνία ELF έχουν συνήθως μήκος πολλών χιλιομέτρων και συχνά βρίσκονται σε απομακρυσμένες περιοχές με χαμηλά επίπεδα ανθρωπογενών παρεμβολών.

    \subsection{}
    Σε πειραματικό επίπεδο έχουν υλοποιηθεί και δοκιμαστεί ορισμένα πρότυπα επικοινωνίας τα οποία βασίζονται σε κάποιες από τις αρχές που έχουν προταθεί παραπάνω. Πιο συγκεκριμένα το HeyPhone είναι ένα φοριτό ραδιόφωνο που χρισημοποιεί μια κεραία βρόχου με διάμετρο ένα μέτρο και έχει την ικανότητα να διαπεράσει το έδοαφος σε βάθος μέχρι και 500 περίπου μέτρα, χρισημοποιεί διαμόρφωση 87KHz Single Side Band (SSB).
    \subsection{}
    Ακόμη μια τεχνολογία που χρισημοποιείται ευραίως στην βιομηχανία είναι το τηλέφωνο ενός καλωδίου (Single Wire Telephone). Οι συσκευές είναι πολύ απλές, χρησιμοποιώντας ένα μόνο op-amp τόσο για να στείλουν όσο και να λάβουν ένα ηχητικό σήμα (φωνή) κατά μήκος ενός μόνο μονωμένου καλωδίου, χρησιμοποιώντας τη χωρητικότητα σύζευξης χειριστών στη γείωση για την επιστροφή. Ο δέκτης έχει πολύ υψηλή αντίσταση εισόδου. Η ζήτηση ισχύος είναι πολύ χαμηλή, με πολλές ώρες (ή ημέρες) λειτουργίας από μπαταρία 9 volt. Απαιτείται πολύ μικρή σύζευξη στο έδαφος στο τέλος της λήψης. Το άκρο μετάδοσης οδηγεί κυρίως στην κατανεμημένη χωρητικότητα του καλωδίου στη γείωση. Τα πολύ μεγάλα καλώδια (χιλιόμετρα) θα απαιτήσουν αρκετά καλή σύζευξη στη γείωση στο άκρο μετάδοσης. Προφανώς, οι αμφίδρομες επικοινωνίες πάνω από ένα πολύ μακρύ σύρμα θα απαιτήσουν αρκετά καλούς λόγους και στα δύο άκρα, όπως ένα μικρό πείρο σε βρωμιά ή ένα γυμνό σύρμα στο νερό.

    \textit{\textbf{Συμπέρασμα}}: Συμπερασματικά, η εργασία καταδεικνύει τη σκοπιμότητα των υπόγειων ασύρματων δικτύων επικοινωνίας σε υπόγεια περιβάλλοντα. Αυτά τα δίκτυα μπορούν να υποστηρίξουν μικρό αριθμό πελατών, με μικρά εύρη ζώνης και υψηλή ανοχή καθυστέρησης, και μπορούν να σχεδιαστούν ώστε να είναι αξιόπιστα και αποτελεσματικά σε τέτοια απαιτητικά περιβάλλοντα. Η μελέτη παρέχει ένα σημείο εκκίνησης για μελλοντική έρευνα σε αυτόν τον τομέα.
\end{multicols}

\end{document}
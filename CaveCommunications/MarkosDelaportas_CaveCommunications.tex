\documentclass[12pt]{article}
\usepackage{geometry}
\geometry{ b4paper, total={220mm,320mm}, left=20mm, top=15mm }

%\usepackage[parfill]{parskip}  % Activate to begin paragraphs with an empty line rather
%than an indent

\usepackage{mathtools}
\usepackage{blindtext}
\usepackage{multicol}
\usepackage{listings}
\usepackage{tikz}

\usepackage{fancyhdr, lastpage, setspace}

\usepackage{pgfplots}
\usepackage{graphicx}
\usepackage{amssymb}
\usepackage{enumitem}
\usepackage{amsmath}

\usepackage{fontspec}
% \usepackage[Latin,Greek]{ucharclasses}
\usepackage{microtype}

\usepackage{polyglossia}
\usepackage[ backend=biber, style=ieee, sorting=ynt ]{biblatex}
\addbibresource{ref.bib}

%\setdefaultlanguage{greek} \setotherlanguages{english}

% \setmainlanguage{greek} \setotherlanguage{english}
% \setTransitionsForGreek{\selectlanguage{greek}}{\selectlanguage{english}}

% \defaultfontfeatures{Mapping=tex-text}


%\setmainfont[Kerning=On,Mapping=tex-text]{Linux Libertine}
\setmainfont{Georgia}
\setsansfont{Segoe UI Variable}
\setmonofont{Corbel}

%\newfontfamily\greekfont[Script=Greek]{Linux Libertine}
%\newfontfamily\greekfontsans[Script=Greek]{Fira Code}

%\DeclareTextFontCommand{\maintxt}{\greekfont}
%\DeclareTextFontCommand{\titletxt}{\greekfontsans}

\renewcommand{\thesection}{\Roman{section}}
\renewcommand{\thesubsection}{\thesection.\Roman{subsection}}

\onehalfspacing

\lhead{Μάρκος Δελαπόρτας} \rhead{ece01316@uowm.gr} \cfoot{Page \thepage\ of
\pageref{lastpage}}

\title{ \textsf{ Cave Communications}\\
    \textsf{Δίκτυα Νέας Γενιάς \& Επικοινωνίες}\\
    \textsf{\Large Τμήμα Ηλεκτρολόγων Μηχανικών \& Μηχανικών Υπολογιστών}\\
    \textsf{\large Πανεπιστήμοιο Δυτικής Μακεδονίας}
} \author{\textsf{Μάρκος Δελαπόρτας} \footnote{E-mail: ece01316@uowm.gr}}
\date{\textsf{Νοέμβριος 2023}}

\begin{document}
\maketitle

\begin{multicols*}{2}
    \scriptsize \textbf{ \textit{Abstract} -- Τα ασύρματα δίκτυα επικοινωνίας αισθητήρων
        έχουν γίνει πανταχού παρόντα, τόσο στην καθημερινή ζωή, όσο και σε πολλές
        βιομηχανίες. Ωστόσο η αποτελεσματικότητά τους μπορεί να παρεμποδιστεί σε ορισμένα
        περιβάλλοντα όπως για παράδειγμα στην εξερεύνηση σπηλαίων, στην διάσωση ανθρώπων
        από κατεστραμέννα κτήρια ή στον τομέα της γεωργίας ακριβείας όπου τα σπαρτά
        βρίσκονται κάτω από την επιφάνεια του εδάφους. Με άλλα λόγια υπάρχει ανάγκη για
        ασύρματη επικοινωνία σε υπόγειο περιβάλλον. Στο παρόν άρθρο εξετάζονται διάφορες
        τεχνικές ασύρματης επικοινωνίας, οι περιορισμοί και οι στερήσεις που έχουν γίνει
        προκειμένου να μοντελοποιηθεί σε περιβάλλον σπηλαίων από διάφορες τεχνολογίες που
        έχουν προταθεί. Καθώς επίσης και πως με αλλαγές στην ισχύ και την συχνότητα μπορεί
        να μειωθεί η απώλεια μονοπατιού και η ισχύς του σήματος λήψης. }

    \section{\normalsize \textsf{Εισαγωγή}}
    % what has not been done in the sector, like there are technoligies in lace but not
    % for a specific matter discussed. \textit{\textbf{Ιστορικό}}: 
        Από την εμφάνηση των πρότων κινητών επικοινωνιών, πολλοί εξερευνητές, όπως και
        πολλοί διασώστες επιχειρούν να τις χρησιμοποιούν προς όφελός τους· πράγμα το οποίο
        αποδυκνύεται δύσκολο λόγω της πρόκλησης που παρουσιάζεται κατά την διάδωση
        ηλεκτρομαγνητικών κυμάτων σε υπόγεια περιβάλλοντα και γενικότερα σε περιβάλλοντα
        με περίπλοκη μορφωλογία. Αξίζει ακόμη να σημειωθεί οτι ενώ δεν υπάρχει πολύ
        βιβλιογραφία για συνήθεις σπήλαια και ερήπεια, έχει γίνει αρκετή έρευνα για
        ορυχεία ανθρακα, παρότι έχουν διαφορετικά χαρακτηριστικά.
    
    % \maintxt{
        Η τρέχουσα γενιά ασύρματων δικτύων, όπως το 4G και το 5G, έχουν κατασκευαστεί
        βασισμένα σε αρχές υψηλού εύρους ζώνης, χαμηλής καθυστέρησης και υψηλής
        διαθεσιμότητας για να ανταποκριθούν στις απαιτήσεις μεγάλου αριθμού πελατών. Αυτά
        τα δίκτυα έχουν σχεδιαστεί για να υποστηρίζουν μεταφορά δεδομένων υψηλής
        ταχύτητας, επικοινωνία χαμηλής καθυστέρησης και εφαρμογές σε πραγματικό χρόνο.
    % }
    % \maintxt{ %\textit{\textbf{Στόχοι}}: 
        Ωστόσο, οι στόχοι των υπόγειων ασύρματων δικτύων επικοινωνίας είναι διαφορετικοί
        από εκείνους των υπαρχόντων ασύρματων δικτύων. Λόγω του μικρού αριθμού πελατών
        (κινητών τερματικών), του μικρού εύρους ζώνης και της υψηλής ανοχής καθυστέρησης,
        αυτά τα δίκτυα θα πρέπει να σχεδιάζονται με διαφορετικές αρχές κατά νου. Σε
        υπόγεια περιβάλλοντα, τα ασύρματα δίκτυα πρέπει να είναι αξιόπιστα, αποτελεσματικά
        και ανεκτικά σε υψηλή εξασθένηση και απώλεια σήματος.
    % }
    % \maintxt{ %\textit{\textbf{Σκοπιμότητα}}: 
        Αυτή η εργασία παρουσιάζει μια μελέτη σκοπιμότητας για τη χρήση υπόγειων ασύρματων
        δικτύων επικοινωνίας σε περιβάλλοντα σπηλαίων. Η μελέτη περιλαμβάνει την ανάλυση
        των υπαρχόντων τεχνολογιών που έχουν προταθεί και υλοποιηθεί, τα χαρακτηριστικά
        διάδοσης του σήματος σε υπόγεια περιβάλλοντα καθώς και το σχεδιασμό πρωτοκόλλων
        επικοινωνίας για την επίτευξη αξιόπιστης και αποτελεσματικής επικοινωνίας.
    % }

    \section{\normalsize \textsf{Το πρόβλημα}} Το μεγαλύτερο πρόβλημα στις υπόγειες
        επικοινωνίες έγγυται στο υψηλό pathloss λόγο της πυκνότητας του μέσου δίαδωσης. Με
        άλλα λόγια το μεγαλύτερο μέρος της έρευνας που έχει λάβει τόπο στις ασύρματες
        επικοινωνίες λαμβάνει ώς μέσο διάδοσης τον κενό χώρος (δηλ. τον αέρα), αυτό διότι
        υπάρχει προβλεψημότητα στην μοντελοποίηση και επειδή επιτρέπει διάδοση σε
        υψηλότερες συχνότητες μεγαλύτερο έυρος ζώνης και άλλα.\\
        Όταν ωστόσο όταν το αντικέιμενο της μελέτης διαφέρει από την μεγιστοποίηση της
        διαδιδόμενης πληροφορίας αλλά εγγυάται στην αξιοπιστία και στην μεγιστοποιηση της
        θωράκισης σε μέσα με υψηλή απώλεια σήματος, προκύπτει ανάγκη για έρευνα σημάτων σε
        διαφορετικές συχνότητες και κωδικοποιήσεις.\\
        Ο σκοπός της παρούσας μελέτης είναι η -- σε περιβάλλον σπηλάιου , συντρίμια
        φυσικές καταστροφές και επικοινωνία με το εξωτερικό περιβάλλον ή/και με προσωπικό
        διάσωσης.

        Η προτεινώμενη λύση με κινητούς αναμεταδότες που προτείνεται είναι πολύ χρίσημη
        για επιχειρήσεις διάσωσης σε περιπτώσεις φυσκών καταστροφών ή/και επικοινωνίας σε
        υπόγειο βάθος.

    \section{\normalsize \textsf{Σχετικές Έρευνες}} Στην βιβλιογραφία έχουν συνταχθεί
        αρκετές εργασίες σχετικά με την ασύρματη επικοινωνία σε σπήλεα και υπόγεια
        περιβάλλοντα. Ωστόσο πρέπει να σημειωθεί οτι δεδομένου του έυρους όλων των 
        σεναρίων/περιπτώσεων που ενέχει η έρευνα υπόγειων επικοινωνιών στην βιβλιογραφία
        μπορεί να βρεθεί μια πληθώρα από μοντέλα που διαφέρουν σχετικά με τα προβλήματα που
        καλούνται να επιλύσουν. Με άλλα λόγια ενώ παρατείθενται διάφορες έρευνες για το 
        συγκεκριμένο ζήτημα κρίνεται απαραίτητο να ομαδοποιηθούν με βάση ορισμένες παραμέτρους
        ώς προς το πρόβλημα και τα φυσικά χαρακτηριστικά του συστήματος.
        
        Πιο συγκεκριμένα ο Muhammed Enes Bayrakdar έδειξε έναν τρόπο για
        συσκευές του δικτύου των πραγμάτων να επικοινωνούν σε ένα δίκτυο πλέγματος
        \cite{bayrakdar_rule_2019}. Οι Mark Hedley και Ian Gipps ανέδειξαν έναν τρόπο για
        ακριβή προσδιορισμό θέσης σε υπόγεια ορυχεία \cite{hedley_accurate_2013}. Οι
        Manoja D. Weiss και Kevin Moore ανέδειξαν έναν τρόπο για αυτόνομη κινητή
        τηλεπικοινωνία και ασύρματη πρόσδεση σε τούνελ \cite{weiss_autonomous_2009}. Οι
        William Walsh και Jay Gao ανέλυσαν την δυνατότητα χρήσης wifi σε περιβάλλον
        σπηλαίου \cite{walsh_communications_2018}. Ο Philip Branch συνέταξε την
        διπλωματική του για κυψελωτά δίκτυα πέμπτης γενιάς σε υπόγεια ορυχεία βαρύτιτας
        \cite{branch_fifth_2021}. Οι M.I. Martínez-Garrido και R. Fort ανέπτυξαν υλισμικό
        για πειράματα σε ανθρωπογενής και φυσική κληρονομιά
        \cite{martinez-garrido_experimental_2016}.

    \section{\normalsize \textsf{Συγκριτική ανάλυση}}
        Οι παράμετροι με βάση τις οποίες θα συγκριθούν οι έρευνες είναι οι εξής:
            \begin{itemize}
                \item Pathloss
                \item Received Signal Power
                \item Data Extraction Rate (DER)
                \item Energy per Packet (E/P)
                \item Volumetric Water Content (VWC) of the soil
            \end{itemize}

        Ακόμη είναι σημαντικό στην μέτρηση της απόδοσης δικτύου να χρησιμοποιείται μια 
        
        
    \section{\normalsize \textsf{Βασικοί Τρόποι Προσέγγισης}} Σε αυτό το κομμάτι
        παρουσιάζονται διάφοροι τρόποι προσέγγισης του προβλήματος. Με άλλα λόγια τι
        τοπολογίες μπορούν να χρισημοποιηθούν για εποικοινωνίες σε σπήλαια και άλλα
        περιβάλλοντα που παρουσιάζεται υψηλή εξασθένηση. Για αυτό δύο τεχνικές μπορούν να
        εφαρμοστούν:
    \begin{itemize}
        \item peer to peer
        \item mesh
    \end{itemize}

    Παρακάτω παρατείθενται διάφορες τοπολογίες και τεχνικές με συνδιασμό ενός εκ των
    παραπάνω τρόπων προσσέγισης και αστερισμού προκειμένουν να επιτευχθεί επικοινωνία
    μεταξύ κινητών συσκευών μέσα σε περιβάλλον σπηλαίου αλλά και μεταξύ ενός σταθμού βάσης
    σε εξωτερικό χώρο.

    \section{\normalsize  \textsf{Βασική Τοπολογία}} Ο βασικός σκοπός του In-Cave Wireless
    Sensor Networks (ICWCS) είναι να παρέχει ένα αξιόπιστο κανάλι φωνητικής επικοινωνίας
    μέσω ενός δικτύου πολυμέσων μεταξύ των μελών μιας ομάδας σε μια σπηλιά. Αποτελείται
    από δύο διαφορετικούς τύπους κόμβων. 
    \begin{itemize}
        \item «κόμβους κορμού» που είναι τοποθετημένο στα τοιχώματα του σπηλαίου.
        \item «Κινητοί κόμβοι», οι οποίοι έχουν παρόμοια λειτουργικότητα όπως τα κινητά
        τηλέφωνα, φέρονται από τα μέλη της ομάδας.
    \end{itemize}

    Οι συσκευές είναι πολύ απλές, χρησιμοποιώντας έναν ενιαίο op-amp για να στέλνουν και
    να λαμβάνουν ένα ηχητικό σήμα (φωνή) κατά μήκος ενός μόνο μονωμένου καλωδίου,
    χρησιμοποιώντας τους χειριστές που συνδέουν τη χωρητικότητα στη γείωση για την
    επιστροφή. Ο δέκτης έχει πολύ υψηλή σύνθετη αντίσταση εισόδου. Η ζήτηση ισχύος είναι
    πολύ χαμηλή, με πολλές ώρες (ή ημέρες) λειτουργίας από μπαταρία 9 Volt. Απαιτείται
    πολύ μικρή σύζευξη με τη γείωση στο άκρο λήψης. Το άκρο μετάδοσης οδηγεί κυρίως την
    κατανεμημένη χωρητικότητα του καλωδίου στη γείωση. Τα πολύ μακριά καλώδια (χιλιόμετρα)
    θα απαιτήσουν αρκετά καλή σύζευξη με τη γείωση στο άκρο εκπομπής. Προφανώς, οι
    αμφίδρομες επικοινωνίες σε ένα πολύ μακρύ καλώδιο θα απαιτούν αρκετά καλή βάση και στα
    δύο άκρα, όπως ένα μικρό μανταλάκι στη βρωμιά ή ένα γυμνό καλώδιο στο νερό.

    \subsection{\small \textsf{Κόμβοι ραχοκοκαλιάς}} Οι κόμβοι κορμού είναι ακίνητοι
        κόμβοι και ο σχεδιασμός τους βασίζεται στον προηγούμενο ασύρματο αισθητήρα γενικής
        χρήσης κόμβος «VF1A» [4]. Έχουν μια μονάδα επεξεργαστή, ένα RF μονάδα πομποδέκτη,
        μονάδα μνήμης και τροφοδοσία ενότητα διαχείρισης. Αυτοί οι κόμβοι είναι υπεύθυνοι
        για διατήρηση της συνδεσιμότητας κορμού μέσα στο σπήλαιο, το δραστηριότητες
        εγγραφής και περιαγωγής κινητών κόμβων, δρομολόγηση πληροφοριών κλήσεων μεταξύ
        κινητών κόμβων και μεταφορά των δεδομένων ελέγχου και φωνής μέσω του ασύρματου
        δίκτυο.
    
        \subsection{\small \textsf{Κινητοί κόμβοι}}  Οι κινητοί κόμβοι μεταφέρονται από
        σπηλαιολόγους και περιλαμβάνουν χαρακτηριστικά πέρα από τα τυπικά ενός ασύρματου
        αισθητήρα κόμβος για την αλληλεπίδραση με τους σπηλαιολόγους. Αυτά τα
        χαρακτηριστικά περιλαμβάνει μια γραφική οθόνη, ένα μικρόφωνο, ένα ακουστικό και
        ένα πληκτρολόγιο ελέγχου.
        %\includegraphics*[scale=.4]{img/ICWMSN-WMSNpng.png}

    \section{\normalsize  \textsf{Τεχνολογίες}} Παρακάτω παρουσιάζονται διάφορες τεχνικές
    διαμόρφωσης προκειμένου να μπορεί να διαδοθεί σήμα μέσα σε σκληρά υλικά ή το
    ενδεχόμενο εκμετάλευσης των ρογμών (π.χ. μεταξύ βράχων) αυτών.

    \subsection{} Σε θεωριτικό επίπεδο εχουν προταθεί τεχνολογίες extrimely low frequency
    (elf) οι οποίες είναι ήδη σε χρήση σε υποβρύχια επειδή τα ραδιοκύματα ELF μπορούν να
    διεισδύσουν στο θαλασσινό νερό σε πολύ μεγαλύτερο βάθος από τα ραδιοκύματα υψηλότερης
    συχνότητας, τα οποία απορροφώνται από το νερό. Τα κύματα ELF (3 - 30Hz) έχουν πολύ
    μεγάλα μήκη κύματος, που κυμαίνονται από εκατοντάδες έως χιλιάδες χιλιόμετρα, και
    παράγονται από μια μεγάλη κεραία που ονομάζεται «δίπολο εδάφους» ή «δίπολο γης». Οι
    κεραίες που χρησιμοποιούνται για την επικοινωνία ELF έχουν συνήθως μήκος πολλών
    χιλιομέτρων και συχνά βρίσκονται σε απομακρυσμένες περιοχές με χαμηλά επίπεδα
    ανθρωπογενών παρεμβολών.

    \subsection{}
    Σε πειραματικό επίπεδο έχουν υλοποιηθεί και δοκιμαστεί ορισμένα πρότυπα επικοινωνίας
    τα οποία βασίζονται σε κάποιες από τις αρχές που έχουν προταθεί παραπάνω. Πιο
    συγκεκριμένα το HeyPhone είναι ένα φοριτό ραδιόφωνο που χρισημοποιεί μια κεραία βρόχου
    με διάμετρο ένα μέτρο και έχει την ικανότητα να διαπεράσει το έδοαφος σε βάθος μέχρι
    και 500 περίπου μέτρα, χρισημοποιεί διαμόρφωση 87KHz Single Side Band (SSB).

    \subsection{}
    Ακόμη μια τεχνολογία που χρισημοποιείται ευραίως στην βιομηχανία είναι το τηλέφωνο
    ενός καλωδίου (Single Wire Telephone). Οι συσκευές είναι πολύ απλές, χρησιμοποιώντας
    ένα μόνο op-amp τόσο για να στείλουν όσο και να λάβουν ένα ηχητικό σήμα (φωνή) κατά
    μήκος ενός μόνο μονωμένου καλωδίου, χρησιμοποιώντας τη χωρητικότητα σύζευξης χειριστών
    στη γείωση για την επιστροφή. Ο δέκτης έχει πολύ υψηλή αντίσταση εισόδου. Η ζήτηση
    ισχύος είναι πολύ χαμηλή, με πολλές ώρες (ή ημέρες) λειτουργίας από μπαταρία 9 volt.
    Απαιτείται πολύ μικρή σύζευξη στο έδαφος στο τέλος της λήψης. Το άκρο μετάδοσης οδηγεί
    κυρίως στην κατανεμημένη χωρητικότητα του καλωδίου στη γείωση. Τα πολύ μεγάλα καλώδια
    (χιλιόμετρα) θα απαιτήσουν αρκετά καλή σύζευξη στη γείωση στο άκρο μετάδοσης.
    Προφανώς, οι αμφίδρομες επικοινωνίες πάνω από ένα πολύ μακρύ σύρμα θα απαιτήσουν
    αρκετά καλούς λόγους και στα δύο άκρα, όπως ένα μικρό πείρο σε βρωμιά ή ένα γυμνό
    σύρμα στο νερό.

    \textbf{Συμπέρασμα}: Συμπερασματικά, η εργασία καταδεικνύει τη σκοπιμότητα των
    υπόγειων ασύρματων δικτύων επικοινωνίας σε υπόγεια περιβάλλοντα. Αυτά τα δίκτυα
    μπορούν να υποστηρίξουν μικρό αριθμό πελατών, με μικρά εύρη ζώνης και υψηλή ανοχή
    καθυστέρησης, και μπορούν να σχεδιαστούν ώστε να είναι αξιόπιστα και αποτελεσματικά σε
    τέτοια απαιτητικά περιβάλλοντα. Η μελέτη παρέχει ένα σημείο εκκίνησης για μελλοντική
    έρευνα σε αυτόν τον τομέα.

    \printbibliography
\end{multicols*}
\end{document}

\begin{landscape}
    \begin{longtable}{p{0.1\linewidth}p{0.1\linewidth}p{0.15\linewidth}p{0.20\linewidth}p{0.1\linewidth}p{0.3\linewidth}}
        \caption{Σύγκριση τεχνολογιών που παρουσιάζονται στην βιβλιογραφία}
        \label{table:comparison}\\
        
        Τεχνολογία & Ερευνητές & Σκοπός Έρευνας & Μέθοδοι & Τύπος Έρευνας & Κύρια Πορίσματα - Συμπεράσματα \\
        \hline

        LoRA WAN & Guozheng Zhao, Kaiqiang Lin, Tong Hao \cite{zhao_feasibility_2023} 
    & Ανάπτυξη προσομοίωσης για να αξιολογηθεί η σκοπιμότητα της υπόγειας παρακολούθησης χρησιμοποιώντας
        υπόγειο ασύρματο δίκτυο αισθητήρων (WUSN) που βασίζεται στο LoRaWAN & 
        Ο προσομοιωτής έκανε 20 προσομοιώσεις με διαφορετικές ρυθμίσεις δικτύου και υπολόγισε πολλές
        μετρήσεις για κάθε προσομοίωση, όπως: Goodput (bps), αναλογία σφάλματος καθυστέρησης (DER)
        και απόδοση ελέγχου ισχύος (EPP). Ο προσομοιωτής εντόπισε επίσης πιθανές συγκρούσεις μεταξύ
        συσκευών στο ασύρματο δίκτυο αισθητήρων χρησιμοποιώντας τέσσερις διαφορετικές λειτουργίες:
        σύγκρουση συχνότητας, Sf(sps), ισχύος, χρονική σύγκρουση
        & Πείραμα μέσω προσομοίωσης 
        & Μέσω πειραμάτων προσομοίωσης, αυτοί οι παράγοντες διερευνήθηκαν ποσοτικά και τα αποτελέσματα
        έδειξαν ότι τα WUSN που βασίζονται στο LoRaWAN επέτρεψαν βαθύτερη και ευρύτερη υπόγεια
        παρακολούθηση με καλή απόδοση δικτύου σε σύγκριση με τα παραδοσιακά WUSN.\\
        \hline
        UHF & Milan Rak and Pavel Pechac \cite{rak_uhf_2007} 
        & Διερεύνηση της διάδοσης ραδιοκυμάτων σε σπήλαια για συχνότητες, 446 MHz και 860 MHz. &
        Τα συνεχή κύματα δημιουργήθηκαν από φορητούς πομπούς εξοπλισμένους με οριζόντια πανκατευθυντικές 
        μονοπολικές κεραίες που πολώθηκαν κατακόρυφα. Η κεραία λήψης μετακινήθηκε κατά μήκος του
        προφίλ της συλλογής αναζητώντας το μέγιστο επίπεδο σήματος, εξαλείφοντας τα τοπικά ελάχιστα
        που προκαλούνται από τη διάδοση πολλαπλών διαδρομών. & Πειράματα και Μαθηματική Ανάλυση &
        Η μετρηθείσα ειδική εξασθένηση στις πρώτες τέσσερις θέσεις ήταν σε καλή συμφωνία με τις 
        θεωρητικές προβλέψεις αλλά στην τοποθεσία Ε, τα αποτελέσματα διέφεραν σημαντικά. 
        Αυτή η απόκλιση αποδόθηκε σε λανθασμένη εκτίμηση των περιβαλλοντικών παραμέτρων.
        Παρατηρήθηκε συνδυασμός πολλαπλών τρόπων λειτουργίας. Αυτό αποδόθηκε στην αυξημένη απόσταση,
        η οποία προκάλεσε σημαντική διασπορά. Η σκέδαση είχε ως αποτέλεσμα την εκπόλωση των κυμάτων
        και παρόμοιες απώλειες διαδρομής και για τις δύο ορθογώνιες πολώσεις. Η μελέτη 
        διαπίστωσε ότι το διάσπαρτο πεδίο συνέβαλε σημαντικά στη διάδοση του σήματος στις στροφές.\\
        
        ZigBee (802.15.4g) & M.I. Martínez-Garrido, R. Fort \cite{martinez-garrido_experimental_2016} &
        Αξιολόγηση μιας πλατφόρμας ασύρματης επικοινωνίας για χρήση σε διάφορα περιβάλλοντα, 
        συμπεριλαμβανομένων εσωτερικών και εξωτερικών που σχεδιάστηκε για να ανιχνεύει αυτόματα νέες
        κινήσεις και να ενσωματώνει τα αντίστοιχα δεδομένα.&
        Οι παράμετροι που μετρώνται μπορούν να χωριστούν σε δύο
        κύριες κατηγορίες: αφενός, τις μετρήσεις που λαμβάνονται από τους ίδιους τους αισθητήρες 
        (σε αυτή τη μελέτη θερμοκρασία και σχετική υγρασία), και αφετέρου την ποιότητα των επικοινωνιών
        και την απαίτηση ενέργειας.& Πειράματα &
        Ποσοτικοποίησε την επίδραση των εμποδίων στη διαδρομή μεταξύ της κίνησης και του σταθμού βάσης, 
        την ενέργεια που καταναλώνεται από τα τερματικά, μετρούμενη ως προς την ισχύ της μπαταρίας. 
        Ο πιο επιζήμιος παράγοντας βρέθηκε ότι ήταν η πλήρης απόφραξη των κεραιών εκπομπής από 
        παρακείμενα μεταλλικά υλικά.\\
        \hline
        WiFi & Pablo F. Miaja, Fermin Navarro-Medina, Daniel G. Aller, Germán León,
        Alejandro Camanzo, Carlos Manuel Suarez, Francisco G. Alonso, Diego Nodar,
        Francesco Sauro, Massimo Bandecchi, Loredana Bessone, Fernando Aguado-Agelet, Manuel Arias
        & Χρήση ενός πρωτοκόλλου IEEE 802.11 για τη δημιουργία ραδιοδέσμης μεταξύ των σημείων πρόσβασης (CH)
        και κινητών τερματικών (CE).
        & Οποιοδήποτε από τα κινητά τερματικά (CE) μπορεί να λειτουργήσει ως «κινητά» σημεία πρόσβασης 
        που διευκολύνουν την επικοινωνία απευθείας οπτικής επαφής (LOS). &
        Μηχανική συστημάτων που βασίζονται σε μοντέλα &
        Το κύριο πλεονέκτημα του 802.11ah έγκειται στη χαμηλότερη συχνότητα που χρησιμοποιείται,
        κάτω από 1 GHz, που ικανοποιεί απαιτήσεις μεγάλης εμβέλειας ή ισοδύναμα μειωμένες ανάγκες
        ισχύος μετάδοσης, καθώς και βελτιωμένη διείσδυση εμποδίων, σε βάρος της διαθέσιμης απόδοσης.\\
        
        ELF (Extremely Low Frequency)& Jarred S. Glickstein, Jifu Liang, Seungdeog Choi, Arjuna Madanayake,
        and Soumyajit Mandal \cite{glickstein_power-efficient_2020}
        & Σχεδιασμός και τη δοκιμή ενός αποτελεσματικού συστήματος μετάδοσης δεδομένων.&
        Η πρωτότυπη μηχανική κεραία δοκιμάστηκε για να εξεταστεί η πρακτική αποτελεσματικότητά της.
        &Θεωρητική Ανάλυση \& Πειραματικός Έλεγχος&
        Ένας πομπός ELF που κατασκευάστηκε χρησιμοποιώντας ένα μηχανικά περιστρεφόμενο μαγνητικό
        δίπολο αξιολογήθηκε για να αποδειχθεί η χρήση του για ασύρματες επικοινωνίες μη οπτικής επαφής
        σε αγώγιμο μεσο.\\
        \hline
        WiFi & William Walsh, Jay Gao \cite{walsh_communications_2018}& Εξέταση ενός πρόχειρου μοντέλου
        μιας σπηλιάς και να συζήτηση σχετικά γενικών ηλεκτρομαγνητικών φαινομένων. &
        Τέσσερις περιοχές εξετάστηκαν κατά τη διάρκεια της μελέτης: το περιβάλλον επικοινωνίας στα
        πέντε μέτρα πίσω από το εμπόδιο, το περιβάλλον επικοινωνίας περίπου 18 μέτρα μπροστά από 
        το εμπόδιο, το σήμα κοντά στα τοιχώματα του σπηλαίου και το σήμα ως εισχωρεί σε μια 
        πλευρική δίοδο που αποκλείεται από τη θέα στο πίσω μέρος της σπηλιάς.&
        Μοντελοποίηση σε Η/Υ & Λαμβάνεται καλό σήμα WiFi σε όλο το σπήλαιο μήκους 100 μέτρων,
        αλλά σημειώνεται ότι η γεωμετρία κάθε σπηλαίου πρέπει να εξετάζεται ξεχωριστά. 
        Η επιλεγμένη γεωμετρία μπορεί να είναι αισιόδοξη καθώς συλλαμβάνει μεγάλη ποσότητα 
        της μεταδιδόμενης ενέργειας και παρέχει πολλές ευκαιρίες διασποράς.\\
        \hline
        Ασαφή Λογική \& BPSK (350MHz)& Muhammed Enes Bayrakdar \cite{bayrakdar_rule_2019}
        & Ανάπτυξη μιας μεθόδου που διασφαλίζει τη μετάδοση δεδομένων χωρίς απώλειες, 
        ελαχιστοποίηση της κατανάλωσης ενέργειας σε υπόγεια δίκτυα αισθητήρων, 
        βελτιστοποίηση της επιλογής των σταθμών συλλογής για τη βελτίωση της απόδοσης του δικτύου. 
        & Ασαφή (Fuzzy) λογική χρησιμοποιείται για τις λειτουργίες επιλογής βάσει κανόνων για τον
        προσδιορισμό της καταλληλότερης τοποθεσίας του σταθμού συλλογής. Βασικές παράμετροι όπως
        η καθυστέρηση, η απόδοση και η κατανάλωση ενέργειας διερευνώνται.
        & Simulation Modeling (Matlab \& Riverbed) &
        Το προτεινόμενο σχέδιο επιλέγει αποτελεσματικά τον καταλληλότερο σταθμό συλλογής, 
        ελαχιστοποιώντας έτσι την κατανάλωση ενέργειας. Τα αποτελέσματα της προσομοίωσης δείχνουν
        σημαντικές βελτιώσεις στη μέση καθυστέρηση και στη μέγιστη απόδοση. Η μελέτη επιβεβαιώνει
        ότι η προτεινόμενη προσέγγιση βασισμένη σε κανόνες, χρησιμοποιώντας ασαφή λογική για την 
        επιλογή σταθμών συλλογής, ενισχύει την απόδοση και την αποδοτικότητα των ασύρματων υπόγειων δικτύων αισθητήρων. \\

        BPSK (315-700MHz)& Muhammed Enes Bayrakdar \cite{bayrakdar_smart_2019}&
        Στόχος είναι η παροχή μιας αξιόπιστης μεθόδου για την ανίχνευση επιβλαβών υπόγειων 
        παρασίτων εντόμων που απειλούν την ανάπτυξη των λαχανικών&
        Ένα μαθηματικό μοντέλο περιγράφει τη δομή του δικτύου αισθητήρων και την επικοινωνία 
        τους με τους σταθμούς συλλογής. Το μοντέλο προσομοίωσης χρησιμοποιείται για να καταδείξει
        την αποτελεσματικότητα της προτεινόμενης τεχνικής σε ελεγχόμενο περιβάλλον.&
        Μαθηματική Ανάλυση \& Προσομοίωση σε Λογισμικό (Riverbed)&
        Η τεχνική που χρησιμοποιεί κατάλληλους ακουστικούς υπόγειους αισθητήρες εντοπίζει
        αποτελεσματικά τα παράσιτα με βάση το επίπεδο θορύβου που παράγουν. 
        Στη συνέχεια, οι πληροφορίες μεταδίδονται στο σταθμό συλλογής για περαιτέρω ενέργειες.
        Οι αισθητήρες εισέρχονται σε κατάσταση αναστολής λειτουργίας όταν δεν υπάρχει σημαντική
        υπόγεια δραστηριότητα.Τέλος τα δεδομένα μεταδίδονται στον σταθμό συλλογής είτε απευθείας
        είτε μέσω άλλων κόμβων (ad-hoc), εξασφαλίζοντας στιβαρή και ευέλικτη επικοινωνία ακόμη 
        και σε διαφορετικά βάθη.\\
        \hline
        WASP (Wireless Ad-hoc System for Positioning)& Mark Hedley, Ian Gipps \cite{hedley_accurate_2013}&
        Βελτίωση της ασφάλειας και της παραγωγικότητας των υπόγειων εργασιών εξόρυξης με την ακριβή
        παρακολούθηση των οχημάτων σε επιχειρήσεις σπηλαίων μπλοκ.&
        Πλακέτες κυκλωμάτων WASP για μέτρηση επικοινωνίας και ώρας άφιξης (TOA), κόμβους 
        τοποθετημένους σε οχήματα, κόμβους αγκύρωσης σε σταθερές θέσεις και υπολογιστές για 
        επεξεργασία και μετάδοση δεδομένων. Αλγόριθμος για τον υπολογισμό του εύρους που βασίζεται
        σε μετρήσεις TOA και ένας αλγόριθμος παρακολούθησης σχεδιασμένος να λειτουργεί με χαμηλή
        πυκνότητα κόμβων αναφοράς.&
        Μαθηματική Μοντελοποίηση \& Πειραματική Υλοποίηση&
        Το σύστημα απέδειξε την ικανότητα ακριβούς παρακολούθησης οχημάτων ορυχείων και διεργασιών
        εξόρυξης μεταλλεύματος σε ένα επιχειρησιακό υπόγειο ορυχείο. Κατάφερε να διατηρήσει 
        αξιόπιστη λειτουργία για μεγάλο χρονικό διάστημα (ένα έτος), αποδεικνύοντας την στιβαρότητα
        και την αποτελεσματικότητά του σε δύσκολα υπόγεια περιβάλλοντα. Τόνισε τα σημαντικά οφέλη
        της ασύρματης παρακολούθησης για τη βελτίωση της ασφάλειας των ανθρακωρύχων και τη βελτίωση
        της λειτουργικής απόδοσης με τη μείωση των συγκρούσεων και την καλύτερη παρακολούθηση
        της παραγωγής.\\
        \hline
        WiFi & Manoja D. Weiss Kevin L. Moore \cite{weiss_autonomous_2009}&
        Ανάπτυξη και αξιολόγηση ενός αλγορίθμου ασύρματης πρόσδεσης που βασίζεται στη λαμβανόμενη ισχύ
        για μια αλυσίδα αυτόνομων κινητών ραδιοφώνων (AMR) που εξερευνούν υπόγειες σήραγγες
        με πολλαπλές στροφές και γωνίες&
        Προσομοίωση για τη σύνδεση με RSS.Η κίνηση AMR εξασφαλίζει συνδέσμους μεγαλύτερους από ένα όριο RSS.
        & Προσομοίωση &
        Ο αλγόριθμος ασύρματης πρόσδεσης επιτρέπει στα AMR να διατηρούν ισορροπημένο εύρος ζώνης μεταξύ των
        πλησιέστερων γειτόνων, αποτρέποντας οποιαδήποτε μεμονωμένη σύνδεση να γίνει bottleneck. Ο αλγόριθμος 
        διατηρεί με επιτυχία το RSS πάνω από το απαραίτητο όριο για τη διασφάλιση συνεχούς επικοινωνίας, ακόμη
        και σε συνθήκες NLOS. Σε σύγκριση με τη σύνδεση LOS, η πρόσδεση βάσει NLOS με κεραίες μέτριας υψηλής 
        απολαβής σχεδόν διπλασιάζει το εύρος εξερεύνησης στα ανθρακωρυχεία.\\

        WLAN 2.4GHZ& Qi Ping Soo, Soo Yong Lim, David Wee Gin Lim, Nurhidayah Rusli, Ka Heng Chong,
        Kian Meng Yap, Sian Lun Lau \cite{soo_measurement_2019}&
        Διερεύνηση της ασύρματης επικοινωνίας σε περιβάλλοντα σπηλαίων, τα οποία χαρακτηρίζονται από ανώμαλα εδάφη&
        Οι ερευνητές πραγματοποίησαν μετρήσεις πεδίου σε ένα μοντέλο ανώμαλου εδάφους που κατασκεύασαν. Αυτό το
        μοντέλο προσομοιώνει τις ακανόνιστες συνθήκες εδάφους που βρίσκονται μέσα σε σπηλιές. Έπειτα συνέκριναν 
        τα αποτελέσματα με δύο προσομοιώσεις ανίχνευσης ακτίνων. Η μία προσομοίωση χρησιμοποιούσε το αρχικό μοντέλο,
        ενώ η άλλη ενσωμάτωσε έναν παράγοντα διασποράς για να ταιριάζει καλύτερα με τις πραγματικές συνθήκες&
        Μοντελοποίηση \& Προσομοίωση &
        Η προσομοίωση ανίχνευσης ακτίνων που περιελάμβανε τον παράγοντα σκέδασης έδειξε καλύτερη συμφωνία με τα
        μετρούμενα αποτελέσματα από το αρχικό μοντέλο, υποδεικνύοντας τη σημασία της εξέτασης των επιδράσεων σκέδασης
        σε τέτοια περιβάλλοντα.\\
        \hline
        RF 200MHz & J. Pingenot, R. Rieben, D. White \cite{pingenot_full_2005}&
        Διεξαγωγή υπολογιστικής ανάλυσης της διάδοσης του σήματος και της εξασθένησης μιας διπολικής κεραίας 200 MHz
        σε περιβάλλον σπηλιάς. &
        Η μελέτη λύνει τις εξισώσεις Maxwell πλήρους κύματος απευθείας στο πεδίο του χρόνου μέσω μιας
        μεθόδου διακριτοποίησης πεπερασμένων στοιχείων διανυσματικού τομέα χρόνου υψηλής τάξης. Ακόμη 
        το σπήλαιο έχει μοντελοποιηθεί ως ένας ευθύς και απωλεστικός τυχαίος τραχύς τοίχος. Επίσης 
        έγινε χρήση συναρτήσεων βάσης παρεμβολής H(curl) δεύτερης τάξης και προσαρμοσμένων κανόνων τετραγωνισμού
        για ακρίβεια.& Μαθηματική Μοντελοποίηση &
        Η μελέτη παρουσιάζει αποτελέσματα για τη φασματική πυκνότητα ισχύος (PSD) και τη φάση των συνιστωσών του 
        ηλεκτρικού πεδίου, που δείχνουν σημαντική υποβάθμιση του σήματος σε απόσταση. Ακόμη οι σπηλιές με τραχιά 
        τοιχώματα εισάγουν τυχαιότητα στη διάδοση του σήματος, γεμίζοντας τα κενά που προκαλούνται από καταστροφικές
        παρεμβολές που παρατηρούνται σε σπήλαια με λεία τοιχώματα. Τέλος αυξάνει την ενέργεια ορισμένων συστατικών 
        του ηλεκτρικού πεδίου (Ex, Ey) περαιτέρω μέσα στο σπήλαιο, υποδεικνύοντας ένα πιο περίπλοκο περιβάλλον διάδοσης
        από ένα με λείες επιφάνειες.\\

        900MHz, 2.4GHz, 5.8GHz & Qi Ping Soo, Soo Yong Lim, David Wee Gin Lim, Kian Meng Yap, Sian Lun Lau 
        \cite{soo_propagation_2018} & Μέτρηση και ανάλυση της διάδοσης των ηλεκτρομαγνητικών σημάτων σε μια φυσική
        σπηλιά που έγινε κελάρι κρασιού &
        Η μελέτη περιελάμβανε τη λήψη μετρήσεων εντός του κελαριού στις καθορισμένες συχνότητες. Εξετάστηκαν τόσο 
        σενάρια οπτικής επαφής (LOS) όσο και σενάρια μη οπτικής επαφής (NLOS). Έγιναν μετρήσεις και ανάλυση σε 
        διαφορετικές θέσεις μέσα στο σπήλαιο για να κατανοηθεί πώς το περιβάλλον του σπηλαίου επηρεάζει τη διάδοση
        του σήματος.&
        Ανάλυση μέσω Πειραμάτων&
        Η μελέτη διαπίστωσε σημαντική εξασθένηση σήματος στο σπήλαιο, με διαφορετικές συχνότητες να παρουσιάζουν 
        διαφορετικά επίπεδα εξασθένησης. Όπως αναμενόταν, η ισχύς του σήματος ήταν υψηλότερη στα σενάρια LOS σε 
        σύγκριση με τα σενάρια NLOS. Οι υψηλότερες συχνότητες (2,4 GHz και 5,8 GHz) παρουσίασαν μεγαλύτερη εξασθένηση
        σε σύγκριση με τις χαμηλότερες συχνότητες (900 MHz), κάτι που είναι σύμφωνο με τη γενική θεωρία διάδοσης.\\
        \hline        
        Ασύρματο Δίκτυο Αισθητήρων 868 MHz, 2.4 GHz \& 5 GHz & Tomás Laborra, Leire Azpilicueta, Peio Lopez Iturri, 
        Erik Aguirre, and Francisco Falcone \cite{laborra_estimation_2014} &
        Ανάλυση και να εκτίμηση της ασύρματης κάλυψης στο εσωτερικό των σπηλαίων για εφαρμογές στη σπηλαιολογία.&
        Η έρευνα χρησιμοποίησε έναν εσωτερικό κώδικα προσομοίωσης 3D Ray Launching σε μια σχηματική αναπαράσταση 
        ενός πραγματικού σπηλαίου στη Ναβάρρε. Εξέτασε την τοπολογία και τη μορφολογία του σπηλαίου, 
        συμπεριλαμβανομένων των ιδιοτήτων των υλικών, για να εκτιμήσει τα επίπεδα ισχύος που έλαβε μέσα στο σπήλαιο.
        Η εκτιμώμενη λαμβανόμενη ισχύς αναλύθηκε για πομπό τοποθετημένο στην είσοδο του σπηλαίου, σε ύψος 1,5 μέτρου.&
        Μοντελοποίηση Προσομοίωσης &
        Η απόδοση του ασύρματου δικτύου εξαρτάται σε μεγάλο βαθμό από την τοπολογία του σπηλαίου. Η προσομοίωση 3D Ray
        Launching παρείχε ακριβείς εκτιμήσεις των λαμβανόμενων επιπέδων ισχύος, επικυρώνοντας τη σκοπιμότητα χρήσης
        WSN σε σπηλιές. Διαφορετικές ζώνες συχνοτήτων (868 MHz, 2.4 GHz και 5 GHz) αξιολογήθηκαν ως προς την απόδοσή
        τους όσον αφορά τη διάδοση και την κάλυψη του σήματος. Μια συνολική αύξηση στην απόδοση του συστήματος μπορεί
        να επιτευχθεί με τη χρήση της ντετερμινιστικής μεθόδου προσομοίωσης που παρέχεται από τον κώδικα 3D Ray Launching.\\
    
        Ray Tracing & Qi Ping Soo, Soo Yong Lim, and David Wee Gin Lim \cite{soo_investigation_2018} &
        Μελέτη της συμπεριφοράς των ραδιοκυμάτων σε μια τυχαία τραχιά επιφάνεια χρησιμοποιώντας την τεχνική ανίχνευσης
        ακτίνων&
        Δημιουργία ενός τυχαίου προφίλ τραχιάς επιφάνειας που με ένδειξη της θέσης του πομπού (Tx) και του δέκτη(Rx). 
        Τοποθέτηση στο 1m μακριά από το Τx, ενώ οι διαδοχικές θέσεις Rx διασκορπίστηκαν 0.3m κατά μήκος του μονοπατιού
        προσομοίωσης συνολικού μήκους 58.78m. Χρήση του κριτηρίου Rayleigh και ενός συνδυασμού από λείες και τραχιές
        επιφάνειες για τον συντελεστή ανάκλασης Fresnel.&
        Μαθηματική Μοντελοποίηση \& Προσομοίωση &
        Παρατηρείται ότι η ισχύς του σήματος μεταξύ των Tx και Rx επηρεάζεται κυρίως από την απόσταση για την λεία 
        επιφάνεια καθώς και για την τυχαία τραχιά επιφάνεια εδάφους. Ακόμη παρατηρείται ότι σε αυτές τις δύο περιπτώσεις,
        το φαινόμενο της γωνίας Brewster εμφανίζεται σε απόσταση περίπου 11m από το Tx ή την 34η θέση του Rx.\\
        \hline
        ICWCS (In-Cave Wireless Communication System) & A. Gokhan Yavuz, Z. Cihan Taysi, Esra Celik 
        \cite{yavuz_-cave_2009} &
        Ανασκόπηση των Συστημάτων Επικοινωνίας σε Σπήλαια \& Παρουσίαση της Τοπολογίας τους &
        Παρουσίαση δύο τύπων κόμβων: τους κόμβους ραχοκοκαλιάς και τους κινητούς κόμβους, καθένας από τους οποίους 
        έχουν ένα αναγνωριστικό μήκους οκτώ χαρακτήρων και καταχωρούνται στον διακομιστή καταλόγου για παρακολούθηση.
        Οι κόμβοι ραχοκοκαλιάς έχουν έναν δείκτη RSS για τον μακρινότερο κόμβο και οι κινητοί κόμβοι που αναμεταδίδουν
        αιτήματα εγγραφής, εξετάζονται στους πίνακες εγγραφής των κόμβων ραχοκοκαλιάς και συνδέονται σε αυτόν με το 
        μεγαλύτερο RSS & Επισκόπηση Τεχνολογιών \& Αναφορά &
        Οι κόμβοι κορμού ICWCS τροφοδοτούνται με μπαταρίες 12V, 1.3Ah που παρέχουν λίγο περισσότερες από 7 ημέρες 
        συνεχούς λειτουργίας. Οι κινητοί κόμβοι ήταν εξοπλισμένοι με τρεις επαναφορτιζόμενες 1.2V, μπαταρίες μεγέθους
        700mAh AAA που παρέχουν τριάντα ώρες συνεχούς λειτουργίας. Ωστόσο η παρούσα υλοποίηση διαχειρίζεται μόνο μία 
        επικοινωνία από σημείο σε σημείο.\\

        Ασύρματη Διάδοση & Donald G. Dudley \cite{dudley_wireless_2005} & Μοντελοποίηση της Ασύρματης Διάδοσης 
        σε Κυκλική Σήραγγα με Απώλειες &
        Διεγείρονται φ-συμμετρικές λειτουργίες με ηλεκτρικούς/μαγνητικούς βρόχους σε κυκλικό βρόχο. Ο μετασχηματισμός
        Fourier και η ολοκλήρωση περιγράμματος χρησιμοποιούνται για την εύρεση τροπικών πεδίων. Αναλύονται οι 
        διακυμάνσεις του πεδίου και λαμβάνονται συχνότητες αποκοπής για διαφορετικές συχνότητες και αγωγιμότητα τοίχων. 
        Οι διακυμάνσεις πεδίου είναι γρήγορες λόγω πολλών τρόπων λειτουργίας στα f=1GHz. Στα f=2GHz, υπάρχουν 
        περισσότερες λειτουργίες και η πτώση είναι ταχύτερη (25dB/1000m).& Μαθηματική Ανάλυση &
        Φαίνεται το αποτέλεσμα για 1GHz και για 2GHz. Για 1GHz, η πτώση στο πεδίο είναι περίπου 
        170dB σε 1000m. Για 2 GHz, χρησιμοποιούνται 29 λειτουργίες και βρίσκεται ότι η πτώση είναι 
        περίπου 60dB. Σε σύγκριση με τις ίδιες παραμέτρους στην περίπτωση ΤΕz είναι σαφές ότι η 
        πτώση είναι πολύ πιο σοβαρή για τη διέγερση του μαγνητικού ρεύματος, 170 έναντι 40dB στο 
        1GHz και 60 έναντι 25dB στα 2GHz.\\
        \hline
        WiFi & Michael R. Souryal, Andreas Wapf, and Nader Moayeri \cite{souryal_rapidly-deployable_2009} &
        Ενίσχυση της επικοινωνιακής κάλυψης με την ανάπτυξη ασύρματων αναμεταδοτών για τη δημιουργία μιας 
        ραχοκοκαλιάς δικτύου.&
        Ανάπτυξη και δοκιμή ενός αυτοματοποιημένου αλγόριθμου για την επέκταση της κάλυψης επικοινωνίας σε 
        πραγματικό χρόνο. Το πείραμα χρησιμοποιούσε υπολογιστές εξοπλισμένους με ραδιόφωνα 
        IEEE 802.11b/g, επαναφορτιζόμενες μπαταρίες και κεραίες. Ο αλγόριθμος έδωσε οδηγίες στους χρήστες να 
        αναπτύξουν κόμβους με βάση αμφίδρομες μετρήσεις SNR. Οι μετρήσεις απόδοσης όπως η καθυστέρηση πακέτων, 
        η διεκπεραίωση και η απώλεια πακέτων μετρήθηκαν κατά τη διάρκεια και μετά την ανάπτυξη του δικτύου.&
        Εφαρμογή Πειράματος &
        Ένα δίκτυο εννιά αναμεταδοτών αναπτύχθηκε με επιτυχία με καθυστερήσεις μετ' επιστροφής κάτω από 80ms.
        Η χρήση του SNR για την αξιολόγηση της ποιότητας ζεύξης επέτρεψε την έγκαιρη ανίχνευση των οριακών ζεύξεων, 
        διευκολύνοντας την έγκαιρη ανάπτυξη αναμεταδοτών.
        Το δίκτυο παρείχε αξιόπιστη αμφίδρομη επικοινωνία φωνής και δεδομένων σε ένα περιβάλλον που προηγουμένως 
        θεωρούνταν νεκρή ζώνη για επικοινωνίες δημόσιας ασφάλειας.\\

        5G Cellular & Philip Branch \cite{branch_fifth_2021} & Αξιολόγηση της καταλληλότητας και των πιθανών
        πλεονεκτημάτων των κυψελοειδών δικτύων 5G έναντι του WiFi για επικοινωνίες σε επιχειρήσεις εξόρυξης 
        υπόγειων σπηλαίων. &
        Εξέταση των τεχνικών πτυχών του 5G, όπως ρυθμοαπόδοση, καθυστέρηση, εύρος ζώνης και ο επανασχεδιασμός 
        του Δικτύου Ραδιοπρόσβασης (Radio Access). Σύγκριση της απόδοσης και της οικονομικής αποδοτικότητας 
        του 5G με τις υπάρχουσες τεχνολογίες WiFi και LTE στις εργασίες εξόρυξης. Προσδιορισμός και συζήτηση 
        συγκεκριμένων περιπτώσεων οπως των δικτύων αισθητήρων και της βιντεοεπιτήρησης.&
        Οικονομική και τεχνική ανάλυση σκοπιμότητας.&
        Το 5G προσφέρει μεγαλύτερες ταχύτητες, χαμηλότερη καθυστέρηση και μεγαλύτερη κάλυψη σε σύγκριση με το WiFi,
        καθιστώντας το ιδανικό για εφαρμογές σε πραγματικό χρόνο και υπόγεια χρήση. Ο αποτελεσματικός σχεδιασμός 
        του μειώνει επίσης το κόστος εγκατάστασης και συντήρησης.\\
        \hline
        Beamforming & Sathish Chandran \cite{chandran_wideband_2005} &
        Βελτίωση της ισχύος των λαμβανόμενων σημάτων και καταστολή παρεμβολών σε συστοιχίες 
        προσαρμοστικής δέσμης ευρείας ζώνης (wideband adaptive beamforming) &
        Η μελέτη χρησιμοποιεί μια τεχνική προσαρμοστικής διάταξης διαμόρφωσης δέσμης υποζώνης (ABA) χρησιμοποιώντας 
        μια συστοιχία φίλτρου καθρέφτη τετραγωνικής όγδοης τάξης (Quadratic Residue Decomposition - QMF). Η QMF 
        χωρίζει το εύρος ζώνης του εισερχόμενου σήματος σε δύο υποζώνες. 
        Η μέθοδος Minimum Variance Distortionless Response (MVDR) χρησιμοποιείται για τη διαδικασία διαμόρφωσης 
        δέσμης, η οποία ελαχιστοποιεί τη διακύμανση εξόδου ενώ διατηρεί μια απόκριση χωρίς παραμόρφωση στην 
        επιθυμητή κατεύθυνση. Αυτή η μέθοδος υλοποιείται χρησιμοποιώντας την Τετραγωνική Αποσύνθεση Υπολειμμάτων 
        (QRD).&
        Μαθηματική Μοντελοποίηση - Επεξεργασία Σήματος &
        Η εισαγωγή της επεξεργασίας υποζώνης με χρήση συστοιχίας QMF όγδοης τάξης βελτιώνει σημαντικά το κέρδος 
        σήματος προς την κατεύθυνση κάθετη προς την προσαρμοστική διάταξη διαμόρφωσης δέσμης.
        Η προτεινόμενη μέθοδος επιτυγχάνει βαθιά καταστολή των παρεμβολών, βελτιώνοντας τη συνολική απόδοση του 
        συστήματος επικοινωνίας. Η QMF βοηθά στη μείωση της υπολογιστικής πολυπλοκότητας αποδεκατίζοντας τη ροή 
        δεδομένων, διατηρώντας έτσι την ποιότητα του σήματος, ελαχιστοποιώντας παράλληλα τα φαινόμενα aliasing.\\
    \end{longtable}
\end{landscape}
